Salut à tous, chers amis en pyjama blanc et jupette pour certains !

Jérôme nous avait parlé il y a quelques semaines d'un livre d'histoire
japonaise, écrit par un de ses amis~\incise~lequel donne d'ailleurs une
conférence fin juin traitant du même thème que son second bouquin, les
invasions mongoles au Japon. Il en avait parlé sur le forum du club, et
l'avait ramené un soir pour le prêter, mais il n'a pas eu le succès escompté.
Comme tout sujet, l'histoire japonaise peut paraître difficile d'accès au
début~\incise~des noms pas franchement faciles à se rappeler, des dates à
foison, d'autres considérations, d'autres valeurs morales\dots~\incise~et je
me suis dit que ça pourrait être intéressant d'en proposer une introduction.
Et puis, ça me permettra de réviser un peu, ce qui n'est pas plus mal.

Mon but est donc d'écrire quelques paragraphes que j'espère assez simples à
lire, à tout le moins pas chiants, qui survoleront 1500 ans d'histoire
japonaise. Je vais essayer de restreindre les dates et les noms, et de laisser
tomber certains détails, quitte à raconter l'histoire plutôt que l'Histoire. À
chacun ensuite, si le virus a pris, d'approfondir telle ou telle période. Je
vais également essayer de montrer, par les (r)évolutions techniques et
sociales successives, comment les batailles rangées de samouraïs ont donné
naissance aux arts martiaux que nous pratiquons chaque semaine. Il y aura six
chapitres à notre voyage. Vais essayer de tout faire de tête, histoire de voir
ce que ça donne\dots

Pour que ce soit plus facile à lire, je l'ai mis en forme, histoire d'éviter
de se ruiner les yeux sur l'écran. Bon, et maintenant que c'est en \LaTeX{},
je pourrais même rajouter des macrons sur les \={o}\dots

\textbf{EDIT, écrit à la fin :}

Comme je remarque que j'ai un peu dépassé la limite d'un paragraphe par
chapitre que je m'étais fixée, je vous propose donc une historiettette du
Japon :

\begin{enumerate}

\item 500 : des clans d'origine paysanne se regroupent, et forment un pouvoir
centré sur la figure de l'Empereur. Petit à petit, les gens à la cour se
détachent des affaires locales, tenues par les guerriers, qui récupèrent alors
le pouvoir. 1180 : un clan en particulier se révèle victorieux, et dépouille
l'Empereur de toute influence, sinon symbolique. Le pouvoir est alors tenu par
un \guil{généralissime}, le Shogun.

\item 1280 : le petit-fils de Genghis Khan essaie de débarquer au Japon, mais
n'y arrive pas. 1333 : l'Empereur du moment essaie de re-récupérer le pouvoir,
mais n'y arrive pas. Cependant, l'influence du shogun diminue de plus en plus,
et en 1450, le Japon se morcelle en une multitude de cité-états indépendantes,
qui se font constamment la guerre.

\item Tout fout l'camp ma bonne dame. 1450 $\rightarrow$ 1600, guerre civile
au Japon. Petit à petit cependant, le pouvoir se concentre dans les mains
d'unificateurs successifs, le dernier restant étant du clan Tokugawa. Aidé
notamment par l'introduction révolutionnaire des arquebuses, \emph{via}
l'arrivée fortuite d'un navire portugais en 1550, la dynastie des Tokugawa est
fondée en 1600.

\item Le pays se cloisonne et se ferme à toute influence
extérieure. Cela ouvre la porte à 250 ans de paix, permettant aux anciennes
casernes militaires de se transformer en écoles d'arts martiaux. Mais\dots

\item 1860 : des navires occidentaux, notamment ceux du Commodore Perry
(\textsc{USA}) forcent, par leur supériorité technologique, le Japon à sortir
de son isolement. Le shogun accepte les traités inégaux, et le Japon
s'industrialise en sortant de sa période féodale. Cela provoque du ressentiment
de la part des anciens clans guerriers, qui voient là une occasion de renverser
le Shogun. C'est l'Empereur qui cristallise cette rébellion, et, après une
guerre civile \guil{contrôlée}, les anciens samourais meurent avec le shogun,
et l'Empereur est de nouveau installé sur le Trône.

\item Face aux puissances occidentales, le Japon a besoin de symboles
nationaux forts. Poussée de l'impérialisme, avec les guerres contre la Chine
(1895) et la Russie (1905) qui sont deux victoires éclatantes pour le Japon.
Parallèlement, renouveau des arts martiaux traditionnels grâce à Jigoro Kano
(fondateur du judo). Quelques années plus tard, l'aikido d'O Sensei naît.

\end{enumerate}