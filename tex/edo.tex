\chapter{Ère d'Edo}

\guil{The secret of steel has always carried with it a mystery. You must learn
its riddle, Conan. You must learn its discipline. For no one, no one in this
world can you trust. Not men, not women, not beasts. [pointant l'épée du doigt]
This you can trust.}

Arrivé au pouvoir par une série de circonstances incertaines, Tokugawa Ieyasu,
et, par la suite, ses héritiers successifs, vont tout mettre en oeuvre pour
rendre cette situation impossible. Le shogunat va en effet cloisonner en castes
quasi-imperméables toute la société du Japon : les daimyos (environ 250) tout
d'abord, qui sont les anciens chefs de clan, alliés ou ennemis, qui règnent sur
leurs provinces avec une certaine autonomie ; les samurai ensuite (environ
400\,000, entre 5\% et 10\% de la population), les seuls à avoir le droit de
porter les armes, qui servent de police aux ordres du shogun et de leur clan,
auquel ils devaient fidélité, sous peine de mort ; les paysans ensuite (un bon
80\% de la population) qui, s'ils jouissent d'un certain prestige social, vont
être de plus en plus écrasés sous les impôts ; et puis, tout en bas, les
artisans et les marchants, qui pourtant vont accumuler des richesses dépassant
largement celles des autres strates. Tout cela était contrôlé de manière
étroite par des officiels, et maintenu par une série de cérémonials et une
étiquette rigoureusement définie, empêchant les gens d'un certain rang d'aller
fricoter avec ceux d'un autre rang.

Tout est prétexte à être codifié, pour accroître le cloisonnement formel de la
société : on écrit des lois limitant l'habillement de tel ou tel rang, les
coiffures et les accessoires deviennent des marques d'appartenance à un groupe,
certains mots sont même rayés du vocabulaire des castes inférieures.
Parallèlement, dès le début du XVII\up{e} siècle, les Tokugawa interdisent le
christianisme, afin de garantir la loyauté des paysans envers leur daimyo,
persécutant violemment les convertis. Craignant que les étrangers soient à
l'origine d'une conquête européenne du Japon, le shogunat instaure une
politique d'isolement de l'île, nommée \emph{Sakoku} (\guil{fermeture du
pays}). Cette politique interdit, sous peine de mort, à tout Japonais de sortir
ou ré-entrer du territoire, puis expulse tous les étrangers du sol japonais, à
l'exception ultra-régulée, de négociants Hollandais (qui réussirent à
convaincre le shogun qu'ils étaient protestants et pas catholiques comme les
Portugais) dans le port de Nagasaki.  C'est \emph{via} ce contact que le Japon
se tient au courant des avancées technologiques occidentales (planches de
médecine, astronomie, horlogerie, électricité, automates\dots).

Le shogunat met également très vite en place un système de mesures empêchant
les daimyos de se rebeller. Même dans ce groupe de 250 personnes, on trouve
trois rangs bien distincts : la famille étendue des Tokugawa ; les daimyo
\guil{fudai}, qui sont ceux qui étaient alliés à Ieyasu avant la bataille de
Sekigahara, et qui récupèrent des fiefs stratégiques, proches de la capitale et
sur le long des routes ; puis les daimyo \guil{tozama}, vaincus à la bataille
de Sekigahara, qui récupèrent les fiefs les plus éloignés d'Edo. Nous y
reviendrons au prochain chapitre, d'ailleurs.  Conformément à leur rang, tous
les daimyos se doivent de maintenir dans la capitale une grande résidence
richement décorée, dans laquelle doit séjourner une partie (souvent la femme et
les fils) de leur famille~\incise~une prise d'otages à peine déguisée.
Parallèlement, les daimyos sont tenus de venir présenter, six mois par an,
leurs respects au shogun, ce qui leur impose d'organiser de vastes processions
ruineuses, particulièrement pour les fiefs les plus éloignés (\emph{sankin
kotai}, \guil{service en alternance}). Louis XIV en son temps exigeait
d'ailleurs la même chose de la noblesse française.  Enfin, ils doivent veiller
à l'entretien des temples et des routes de leur fief, et doivent demander la
permission au shogun d'effectuer des réparations sur leur chateau.

Dans cette société de castes fermée, bétonnée, le Japon jouit pourtant d'une
paix durable. En 1860, plus d'un tiers de la population était lettrée, et ce
sans que l'éducation ne soit obligatoire. Le Japon publiait plus de livres que
toute l'Europe réunie. Un visiteur allemand dont le nom m'échappe s'étonne de
voir les gens, du plus noble au plus populaire, sortir un mouchoir en papier
pour se moucher, plutôt que de le faire dans sa manche comme il était coutume
chez lui. Les arts foisonnent, notamment la poésie et l'écriture de haikus, la
peinture avec les estampes (Hokusai, Hiroshige, Utamaro\dots), le théâtre avec
le kabuki et le bunraku (théâtre de marionnettes). À propos des estampes, c'est
notamment \emph{via} le système d'astreinte du \emph{sankin kotai} et tout le
monde qui gravite que Hiroshige peint ses \emph{53 Stations du Tokaido} ou
Hokusai ses \emph{36 Vues du mont Fuji} (contenant notamment la célèbre vague),
un peu comme un Guide du Routard à l'attention des voyageurs voyageant entre
Edo et Kyoto. Petite parenthèse personnelle, la finance aussi se met en place,
avec les instruments de crédit, et les premières options, utilisées par les
marchands autour d'Osaka.

Et côté armes ? Déjà, le shogun, connaissant la puissance des armes à feu,
interdit la création d'arquebuses et les fait saisir, prétextant avoir besoin
de bronze pour fondre une statue de Bouddha, afin de remercier les dieux de
cette période de paix. Mais à quoi peuvent servir la masse énorme de soldats
qui avaient servi dans ces batailles gigantesques de l'ère précédente ? Si la
majorité retourne à ses champs, un nombre conséquent d'anciens samourais,
devenus sans maître ni cause à servir, deviennent des \guil{ronin}, des
\guil{hommes (ballotés par les) vagues}. Parallèlement, sans guerre à mener,
les samourais encore employés comme soldats perdent leur fonction première, et
sont alors des bureaucrates au service de leur daimyo. La paire de sabres, le
grand katana et le petit wakizashi sert de marque de statut social, tout comme
le chignon relevé sur la tonsure. Certains cependant déplacent la recherche de
l'efficacité martiale brute, si désirée pendant la période de tumultes
précédente vers une recherche de connaissance et de développement de soi.
Ainsi, poussé par une logique néo-confucéenne, qui veut qu'une société prospère
quand les supérieurs donnent l'exemple vertueux à suivre aux inférieurs,
certains idéaux du guerrier, qui préfigurent le Bushido (la voie du samouraï,
si l'on veut), commencent à être formalisés. Les anciens forgerons, plutôt que
d'équiper en masse des armées qui n'existent plus, vont chercher l'excellence
de leur art. De même, privés de leur utilité première, certains anciens
guerriers vont utiliser les techniques martiales pour les transformer en
discipline de vie~\incise~et notamment avec ce qu'ils ont sous la main,
c'est-à-dire leur corps et leurs sabres~\incise~et donnent naissance aux arts
martiaux. Les \emph{jutsu}, les pratiques guerrières, deviennent des \emph{do},
des disciplines, des voies.

C'est dans cette atmosphère que Yamamoto Jocho écrit le \emph{Hagakure}, un des
livres fondateurs du mythe moderne du samourai. C'est ce livre que Jarmusch
fait lire à Forest Whitaker dans \emph{Ghost Dog}, par exemple, et c'est le
\emph{Hagakure} que Mishima considérait comme son livre de chevet, au point
d'en avoir écrit un commentaire. On notera (ironiquement ? Ce serait ne pas
avoir compris le principe d'un mythe\dots) que Jocho décrit l'idéal du samourai
en ayant vécu un bon siècle après la dernière bataille que le Japon ait connu.
C'est également dans cette atmosphère que Musashi, le bretteur légendaire,
écrit son \emph{Traité des Cinq Roues}. À ce sujet justement, pendant l'ère
d'Edo, les écoles de disciplines de combat divers se multiplient. Pour le sabre
par exemple, on en recense plus de 2\,000 différentes en 1750, avec chacune un
\guil{style} qui lui est propre. C'est vers cette période d'ailleurs que le
shinai (le sabre en bambou) et le bogu (l'armure), encore utilisés quasiment
tels quels en kendo, sont développés, histoire d'arrêter d'utiliser un bokken
pour s'entraîner au combat. Les écoles se défient fréquemment, afin de prouver
au public, et aux élites en particulier, la supériorité de leurs styles. Les
élèves sont d'ailleurs encouragés à mener un pèlerinage (\emph{musha shugyo}),
arpentant seuls le pays pour chercher à parfaire leurs connaissances.

Tout pourrait donc aller plus ou moins pour le mieux dans le meilleur des
mondes\dots\ mais le souci est double : d'une, ce monde est fermé, et
l'extérieur se développe sacrément vite (l'Europe entre en pleine révolution
industrielle) ; de deux, ce monde est dramatiquement mal géré. La strate des
samurai, interdite de cultiver ou, pire, de commercer, sous peine d'être
destituée, voit ses revenus, indexés aux mêmes niveaux de riz qu'à l'avènement
du shogunat, donc soumis à une inflation énorme depuis 200 ans, s'évaporer
comme rosée au soleil d'été. Résultat, ils s'endettent auprès des commerçants ;
tout le monde s'endette, même le shogunat, qui n'a jamais taxé le commerce, et
voilà qui donne aux marchands une importance capitale, qu'ils traduisent en
influence politique. La grogne monte, l'autorité du shogunat est remise en
question, certains villages se soulèvent\dots\ Et, que vois-je au loin ? Quatre
navires, chacun peut-être dix fois plus large que le plus gros de nos
\emph{senki bune}, avec des colonnes énormes qui crachent une fumée noire, et
battant un pavillon étoilé inconnu\dots\ Voilà qui n'augure rien de bon, et
annonce la fin de ce quatrième chapitre !
