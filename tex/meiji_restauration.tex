\chapter{Restauration Meiji}

\guil{Someday, when all your civilization and science are likewise swept away,
your kind will pray for a man with a sword.}

1853~\incise~le début de la fin pour les samurai. Préparez vos mouchoirs, c'est
le moment tragique. En même pas vingt ans, le pouvoir du shogun, et, par là,
les samurais, va s'éteindre, et l'Empereur va de nouveau être à la tête du
pays. C'est une période que je trouve proprement fascinante, j'espère que vous
m'excuserez si je ne résiste pas à la tentation de rentrer dans certains
détails. L'Histoire a retenu de cette transition qu'elle a été douce, sans
heurts et pacifique, louant les capacités d'adaptation des Japonais~\incise~le
nom même est \guil{Restauration Meiji}, pas \guil{Révolution} ou \guil{Guerre
civile}~\incise~et rien ne serait moins faux.

Un matin de juillet, le Commodore Perry, de la marine américaine, débarque avec
quatre cannonières à vapeur dans la baie d'Edo, la capitale. Il ouvre les
négociations visant à faire sortir le Japon de son isolation, limitant le
commerce avec l'extérieur (et, notamment, avec les puissances occidentales), en
faisant preuve de la puissance de son artillerie embarquée, qui terrorise les
Japonais. Autour du shogun, deux écoles s'opposent : \textcolor{blue}{ceux qui
refusent entièrement de traiter avec les étrangers}, au risque d'aller à la
guerre, et \textcolor{red}{ceux qui sont prêts à les accueillir}. Dans ce
camp-là d'ailleurs, il ne faut voir aucune forme d'amitié particulière. La
plupart, conscients du danger que pose la technologie étrangère supérieure sur
le pays, cherchent à compromettre afin d'apprendre pour mettre le Japon au même
niveau que le reste du monde. Des intellectuels synthétisent tout cela en
théorisant un Japon armé de \guil{connaissances de l'ouest} et de la
\guil{morale de l'est}, afin de \guil{contrôler les barbares avec leurs propres
méthodes}. Le \textcolor{red}{shogunat}, séduit par cette option, et contraint
par la \guil{politique de la canonnière} de Perry et de ses successeurs, signe
avec les États-Unis plusieurs traités de paix. Ces traités sont cependant dans
les faits des conventions très inégales, semblables d'ailleurs au traité de
Tianjin, signé entre la Grande-Bretagne et la Chine après la seconde guerre de
l'opium, pendant la même année. Le Japon signe bien vite d'autres traités
similaires avec les autres puissances (Angleterre, France, Russie,
Pays-Bas\dots). Ces traités ouvrent les grands ports japonais aux étrangers,
les laissent libres de commercer, avec des taxes à l'import et à l'export
extrêmement faibles.

L'ouverture du Japon à ce commerce extérieur incontrôlé entraîne une forte
instabilité économique. Parallèlement, des tremblements de terre et de
mauvaises récoltes à répétition font énormément augmenter le prix de la
nourriture, qui n'en avait pas besoin vu l'inflation effrayante qui avait
cours. Si le \textcolor{red}{shogunat} envoie de nombreuses missions,
notamment en Allemagne et en France (une délégation participe à l'Exposition
Universelle de 1867 à Paris) et tente de faire réviser les traités
inégalitaires, son autorité décroit rapidement. On assiste de nouveau à un
morcellement du Japon, les daimyos reprenant peu à peu leur indépendance.
Schématiquement, ceux qui étaient dans le camp des vainqueurs à la bataille de
Sekigahara en 1600, \textcolor{red}{les fudai-daimyo}, restent loyaux au
shogun ; les autres, \textcolor{blue}{les tozama-daimyo }qui avaient été
placés dans les fiefs les plus lointains de la capitale, s'opposent au
contraire résolument au shogunat. Les camps de \textcolor{blue}{Choshu}, au
sud-ouest de Honshu (l'île principale), \textcolor{blue}{Tosa} sur l'île de
Shikoku et \textcolor{blue}{Satsuma}, à l'extrémité sud de l'île de Kyushu,
sont les plus virulents dans leur opposition, et vont jouer un rôle
déterminant dans les événements qui vont suivre.

Comme d'hab', une crise de succession au shogun vient rajouter de l'huile sur
un feu crépitant. Si les \textcolor{red}{fudai daimyo} gagnent cette lutte de
pouvoir, une purge brutale et sanglante des \textcolor{blue}{opposants}
accentue le ressentiment contre les \textcolor{red}{Tokugawa}, qui en sortent
encore plus affaiblis. Le mouvement \emph{sonno joi}, \guil{Révérer l'Empereur,
expulser les barbares !}, devient un slogan de ralliement dans les provinces de
\textcolor{blue}{Choshu et de Satsuma}. Cette philosophie, outre d'être un
totale opposition avec la politique du \textcolor{red}{shogunat}, contient
également une idée nationaliste forte~\incise~le caractère pour \guil{barbare}
comportant une notion péjorative de \guil{race}. Elle rassemble autour d'elle
la plupart des castes guerrières, qui n'appréciaient pas de signer sous la
contrainte, et qui voyaient en outre leurs pouvoirs et leurs prérogatives
diminuer. La violence grandit contre les étrangers et ceux qui commercent avec
eux. L'ambassade britannique se fait attaquer, le \textcolor{red}{signataire}
qui avait signé le \guil{Traité d'amitié et de commerce} avec les américains se
fait assassiner en pleine rue, l'ambassade américaine est incendiée\dots\ C'est
également à cette occasion que les occidentaux découvrent le rituel du seppuku
: en 1868, onze marins d'une frégate française qui mouillait dans la baie
d'Osaka, sont tués par des guerriers de \textcolor{blue}{Tosa}. L'ambassadeur
de France au Japon proteste et exige un châtiment exemplaire : le daimyo de
\textcolor{blue}{Tosa}, défiant, présente dès le lendemain vingt samourais
jugés coupables, et condamnés à mort par seppuku. La magnanimité du capitaine
de vaisseau, assistant à l'exécution volontaire, permet à neuf d'entre eux
d'être graciés. D'autres témoignages insistent au contraire sur son effroi, les
samourais, dans une ultime bravade morbide, lui jetant leurs intestins à ses
pieds. On peut lire d'ailleurs ses mémoires décrivant la scène sur le net.

L'opposition armée à l'influence occidentale dégénère alors en réelle guerre
civile lorsque \textcolor{blue}{l'Empereur Komei}, rompant avec des siècles de
tradition~\incise~l'Empereur n'avait pas reigné effectivement sur le Japon
depuis 800 ans !~\incise~prend un rôle actif dans les affaires d'État. Il
reprend le slogan \guil{sonno joi} en proclamant en 1863 l'ordre (divin)
d'expulser les barbares, défiant par là ouvertement la politique du
\textcolor{red}{shogunat}. Le clan \textcolor{blue}{Choshu} suit cet ordre à la
lettre et fait tirer sans avertissement sur tous les navires étrangers qui
tentaient de traverser le détroit de Shimonoseki, entre l'île de Kyushu et
l'île d'Honshu. Petite anecdote : \textcolor{blue}{Choshu} tire avec d'antiques
canons (certains en bois), datant d'avant 1600, construits pendant l'ère
Sengoku. En quelques semaines, des navires américains, français, hollandais et
britanniques se font tirer dessus, avec une artillerie antédiluvienne. Quelques
jours plus tard, l'ordre est donné d'aller châtier ces insulaires têtus et
sous-développés : c'est la bataille de Shimonoseki, un combat peu
connu~\incise~et plutôt inégal~\incise~dans lequel des frégates d'un peu toutes
les nations vont bombarder les forteresses japonaises. La portée des canons
occidentaux dépasse de loin celle des pétoires japonaises vieilles de trois
siècles. Les français débarquent d'ailleurs quelques hommes pour sécuriser le
détroit, et capturent les pièces d'artillerie japonaise. Vous pouvez voir ces
canons exposés devant la porte nord des Invalides, devant le Musée de la
Guerre, et vous amuser à discerner le blason de Choshu gravé à côté de leurs
bouches. En passant, un diplomate britannique brillant, Sir Ernest Satow voyage
dans une des frégates anglaises. Il deviendra par la suite et jusqu'en 1900
l'ambassadeur de Londres au Japon, et, par son journal (\emph{A Diplomat in
Japan}), donne un témoignage de première main passionnant sur cette période
trouble de transition.

Le Japon est divisé en deux groupes : l'un, \textcolor{red}{pro-shogun},
centré sur la capitale Edo (l'actuelle Tokyo) et l'autre, \textcolor{blue}{anti-shogun},
donc \textcolor{blue}{pro-empereur}, unissant les clans de
l'ouest et du sud du Japon. Les deux partis prennent des mesures fortes pour
accroître leurs influences respectives. Le \textcolor{red}{bakufu} ainsi
envoie des marins étudier dans les écoles navales occidentales, et s'allie
avec la France (qui, l'histoire le montrera, aura parié sur le mauvais cheval)
pour construire notamment des arsenaux et moderniser son armée. Tiens, autre
anecdote\dots\ Vous souvenez-vous de cette purge du \emph{Dernier Samourai}
avec Tomtom Cruise ? La vérité n'est pas totalement loin de ce qui est montré
dans le flim, et le personnage joué par le beau gosse aux yeux bleus a
réellement existé~\incise~sauf qu'il était français. Il s'appelait
\textcolor{red}{Jules Brunet}, un officier artilleur de l'armée de Napoléon
III en charge de moderniser l'armée du shogun, dont nous parlerons un peu plus
bas. La \textcolor{blue}{coalition impériale} de son côté n'était pas en reste
: poussée par des visionnaires comme \textcolor{blue}{Sakamoto Ryoma} ou
\textcolor{blue}{Saigo Takamori}, des experts américains et britanniques
aident à former une nouvelle armée, issue de toutes les strates de la
population, et achètent des mitrailleuses, des obusiers et des navires de
guerre, notamment au marchant écossais Thomas Glover. C'est cet écossais qui a
aidé à fonder ce qui deviendra la compagnie Mitsubishi et la Brasserie
Japonaise~\incise~laquelle arborera en hommage à Glover une moustache sur ses
bouteilles de Kirin. C'est lui aussi qui, à la suite d'une affaire avec une
courtisane, a inspiré Puccini pour écrire \emph{Madame Butterfly}.

Un mot sur \textcolor{blue}{Sakamoto Ryoma}. Né dans le fief de Tosa, épéiste
confirmé, il s'engage vite dans le patriotisme anti-étrangers et choisit
d'assassiner \textcolor{red}{Katsu Kaishu} (dans quelle couleur l'écrire ?),
un officiel haut-gradé du shogunat, et fervent partisan de l'occidentalisation
et de la modernisation du Japon. Il entre dans sa maison au crépuscule armé
simplement d'un sabre court, et le surprend en train de lire un traité de
droit maritime néerlandais. \guil{Je ne suis pas armé, et j'aimerai finir ce
livre avant de mourir. Quelle différence cela fait-il pour toi, que je meure
maintenant ou dans quelques heures ? Je te jure, sur mes ancêtres, de ne pas
fuir quand le soleil se lèvera}, lui lance le conseiller. Si la légende dit
vrai, \textcolor{red}{Katsu Kaishu} réussit à persuader
\textcolor{blue}{Ryoma} de la futilité qu'il avait à combattre les pouvoirs
occidentaux au vu de l'état du Japon de l'époque, et de la nécessité pour le
pays d'avoir un projet à long terme. À l'aube, \textcolor{blue}{Ryoma} devient
l'assistant et le protégé de \textcolor{red}{Kai}\textcolor{blue}{shu} (qui
deviendra d'ailleurs, en 1873, le ministre de la Marine Impériale).
\textcolor{blue}{Ryoma} meurt hélas avant d'avoir vu l'impact de ses actions
et de ses convitions, à 31 ans, dans une auberge à Kyoto, assassiné par une
milice shogunale bien connue par les amateurs de flims de sabre et de mangas,
le \textcolor{red}{Shinsengumi}.

En effet, devant l'agitation dans sa capitale, créée notamment par d'anciens
samourais, le \textcolor{red}{shogunat} fonde de nombreuses milices
indépendantes, chargées de maintenir l'ordre. Ces milices, extrêmement
organisées et régies sur un code d'honneur strict, sont constituées de
samourais très compétents, l'entrée se faisant sur une évaluation du niveau de
\emph{kenjutsu} (de sabre). Dans les faits, ces épéistes d'élite se comportent
de la même manière qu'une mafia : assassinats, extorsions, chantages,
intimidations, actes terroristes\dots\ étaient fréquents. C'est cependant là
une autre preuve de la différence fondamentale qui existe entre les morales
japonaise et occidentale : la plus connue de ces milices, le
\textcolor{red}{Shinsengumi}, arborait en effet sur les vestes de chacun de
ses membres l'idéogramme \emph{makoto}, la \guil{sincérité} dont nous parlions
tantôt, écrit en traits d'or sur fond rouge. Une autre de ses milices, les
\emph{\textcolor{red}{hitokiri}} (mot-à-mot, \guil{coupeurs de gens}),
contenait quatre escrimeurs d'élite dont un \textcolor{red}{Tanaka Shinbei}
dont je raconte une anecdote captivante tout à la fin.

C'est \textcolor{blue}{Saigo Takamori}, un colosse taciturne et colérique, du
domaine de \textcolor{blue}{Satsuma}, qui prend la tête des troupes impériales
nouvellement formées. Menacé par une action militaire imminente, le shogun
abdique en 1867 et rend le pouvoir à \textcolor{blue}{l'Empereur Komei}. Dans
la foulée, l'Empereur Komei meurt (probablement assassiné d'ailleurs par des
extrémistes de \textcolor{blue}{Choshu}, craignant une alliance entre
l'Empereur et le shogun) et est remplacé par son jeune (15 ans) fils qui sera
connu sous le nom \textcolor{blue}{d'Empereur Meiji}. L'Empereur Meiji se
suffit de ce transfert d'autorité, mais \textcolor{blue}{Saigo} tempête et
exige que que le \textcolor{red}{shogun} soit privé de son armée, de ses terres
et de son titre, et devienne un simple daimyo. Le shogun refuse, et le Japon
plonge dans une guerre civile, la guerre du Boshin.

En janvier 1868, les forces du \textcolor{red}{shogunat} attaquent les forces
de \textcolor{blue}{Choshu et de Satsuma}, menées par \textcolor{blue}{Saigo},
à l'entrée de Kyoto. Même si l'armée shogunale compte 15\,000 hommes, certains
formés par des experts français, la majorité d'entre elle reste d'anciens
samurai. Les forces impériales, comptant moins de 5\,000 hommes, posséde
cependant des fusils, des mitrailleuses et une artillerie technologiquement
supérieurs, et une armée globalement mieux disciplinée~\incise~et moins
encline aux charges héroïques et honorables, quoique tactiquement peu
efficaces\dots\ C'est \textcolor{red}{Kastu Kaishu} (le \guil{Bismarck
japonais}) qui négocie avec \textcolor{blue}{Saigo} la capitulation d'Edo, en
mai. Désorganisée, démotivée et sans ressources, l'armée shogunale essuie les
défaites et fuit toujours plus au nord. Quelques centaines de soldats fondent
la \textcolor{red}{République d'Ezo }à Hokkaido (l'île tout au nord du Japon).
Parmi ces hommes, l'ancien commandant-en-chef du \textcolor{red}{Shinsengumi},
et cinq capitaines français, dont \textcolor{red}{Jules Brunet} dont nous
parlions plus haut. Anecdote amusante, au début du conflit, les Français
envoyèrent à Napoléon III une lettre. La France, comme les autres puissances
occidentales, s'étant déclarée comme neutre, ils devaient quitter l'armée
française pour y participer. Cette lettre de démission indique explicitement
qu'ils estimaient que les qualités d'honneur, de valeur et de dignité, si
importantes pour un militaires, étaient plus prononcées ici au Japon que dans
l'armée française. À un contre trois, les dernières forces moribondes rebelles
se font décimer, en dépit de la forteresse en étoile, construite sur des plans
de Vauban. Certains conseillers français restent au Japon, Brunet rentre et,
en dépit de la requète japonaise d'être jugé comme traître, le soutien
populaire dont il jouit force l'armée française de le cacher pendant quelques
années, le temps que l'affaire se tasse.

La victoire sur tout le territoire acquise, le nouveau gouvernement, avec
\textcolor{blue}{l'Empereur Meiji} à sa tête (quoiqu'à 15 ans, la tête était
plus symbolique qu'autre chose) s'efforce de moderniser le pays.
L'indépendance des domaines est petit à petit supprimée, et la classe des
samourai est abolie, par des édits successifs : interdiction de porter le
chignon, les sabres, suppression de leurs privilèges, autorisation des classes
populaires d'avoir un nom de famille\dots\ \textcolor{blue}{Saigo Takamori},
qui avait mené les troupes impériales pendant la guerre du Boshin, rejoint un
poste-clef dans le gouvernement, et demande la clémence pour les anciens
partisans au shogun. Le gouvernement impérial fait petit à petit glisser sa
politique initiale d'\guil{expulser les barbares !} en cherchant de plus en
plus à construire une nation forte, similaire aux puissances occidentales.
Mais la notion de \guil{nation forte} divise le gouvernement naissant :
certains considèrent qu'il faut moderniser le pays, avec des projets publics
de grande ampleur (télégraphe, chemin de fer\dots), d'autres, comme
\textcolor{blue}{Saigo}, insistent qu'il s'agit en premier lieu de posséder
une armée puissante, afin que le pays soit reconnu par les autres puissances
mondiales. Ce dernier propose d'ailleurs d'envahir la Corée, suite à son refus
de reconnaître \textcolor{blue}{l'Empereur Meiji} comme chef d'État. nt se
propose même d'aller en Corée en temps qu'ambassadeur, et de se comporter de
manière si outrancière que les Coréens n'aient d'autre choix que de le tuer,
provocant un \emph{casus belli} donnant au Japon une raison de lui déclarer la
guerre. Une telle entreprise aurait été cependant désastreuse, notamment sur
les finances balbutiantes du pays, et est refusée. \textcolor{green}{Saigo}
démissionne de toutes ses fonctions et rentre dans sa ville natale, Kagoshima.

Il fonde une école militaire privée, afin surtout de canaliser les énergies
des anciens samourais qui l'aidèrent à combattre le shogunat. Craignant une
rebellion de ces \textcolor{green}{forces nostalgiques}, le
\textcolor{blue}{gouvernement impérial} envoie un navire de guerre à Kagoshima
pour prendre toutes les armes et munitions des arsenaux. Ironiquement, cela
conduit les tensions déjà importantes (en cette même année de 1877, le
\textcolor{blue}{gouvernement} supprime les rentes des samourais) à éclater en
conflit armé. \textcolor{green}{Un millier d'étudiants} armés tout autant
d'armes à feu moderne que d'armes traditionnelles (sabres, lances, arcs)
attaquent les ports et arsenaux. Devant ce fait accompli,
\textcolor{green}{Saigo}, écartelé entre la manifestaion vivante de ses
morales qui disparaissent et son désir pacifique loyal au Japon, est poussé à
mener la rebellion. À 60 contre 1 environ, les mathématiques ne sont pas du
côté des insurgés. Quelques mois plus tard, \textcolor{green}{Saigo} lance une
charge héroïque et stupide, à la tête d'une \textcolor{blue}{armée bigarrée},
entre fusiliers sans munition et sabreurs anachroniques, contre une compagnie
de \textcolor{green}{mitrailleuses modernes}. Blessé à la cuisse, il se retire
du champ de bataille pour s'ouvrir le ventre comme l'exige son rang : ainsi
meurt le dernier samourai. 

Bien que le grand Saigo restera dans l'inconscient collectif comme étant un
grand héros, le gouvernement a maintenant toute liberté de révolutionner le
pays, et de le modeler comme une puissance occidentale. Il adopte une
constitution, inspirée de la constitution de l'Empire Germanique, et fait des
réformes de fond afin de moderniser le pays dans tous les domaines : sciences
modernes, langues, technologies, constructions, agriculture, éducation,
finance\dots\ et sports. Aux arts martiaux, associés à la tradition
rétrograde, sont préférés les sports occidentaux, comme le base-ball (encore
très en vogue au Japon aujourd'hui) et le rugby, par exemple. Mais alors, que
diable s'est-il donc passé pour que nous étudions l'aikido chaque semaine ?
Nous verrons cela dans notre prochain et dernier (ouf !) chapitre !
