\chapter{Ère des province en guerre}

\guil{I remember days like this when my father took me to the forest and we ate
wild blueberries. More than 20 years ago\dots\ I was just a boy of four or
five. The leaves were so dark and green then. The grass smelled sweet with the
spring wind\dots\ Almost twenty years of pitiless combat ! No rest, no sleep
like other men. And yet the spring wind blows, Subotai. But for us, there is no
spring. Just the wind that smells fresh before the storm.}

Nous nous étions quittés tantôt alors que l'influence du shogun déclinait
grandement au Japon.  Rappelons que, tout mandaté par les Cieux qu'il soit,
l'Empereur n'avait aucun pouvoir politique, et restait cantonné dans sa
capitale. Le conflit initial dont nous parlions, la Guerre d'Onin, annonce la
prochaine période de notre voyage : \emph{Sengoku jidai}, l'âge des provinces
en guerre. C'est cette période qui est largement décrite dans le premier livre
de Julien Peltier que Jérôme avait ramené au dojo (\emph{Le Crépuscule des
Samouraïs} si ma mémoire est bonne).

Cette période de conflits virtuellement ininterrompus peut se modéliser en deux
phases : une phase de morcellement du Japon, qui se transforme petit à petit en
une multitude de petits seigneurs de guerre locaux autonomes, les
\emph{daimyo}, qui cherchaient à annexer leurs voisins ; puis, une phase
d'unification, les daimyos restants devenant de plus en plus puissants, qui
culmine en 1600, quand Tokugawa Ieyasu, victorieux de tous ses adversaires,
fonde le shogunat qui porte son nom, et qui durera plus de 250 ans.

En passant, outre le livre de Peltier, et à condition d'avoir une babasse
conséquente pour le faire tourner, je vous conseille très chaudement le jeu de
stratégie / gestion \emph{Total War : Shogun 2}, sorti l'an dernier, qui vous
place dans le kimono d'un de ces daimyos. À vous de monter des alliances, de
lever des armées et de déferler sur un Japon en plein chaos pour vous hisser à
la tête du pays. À quelques détails près, le jeu est historiquement solide, et
permet d'apprendre sans trop de difficultés la multitude de noms des gens
d'importance de l'époque. Et, très honnêtement, lancer ses ashigarus à l'assaut
des pentes douces en pierre des châteaux japonais, pour les voir ensuite
échanger des passes d'armes martialement réalistes avec les sabreurs ennemis,
sashimonos flottant au vent dans leur dos, ça n'a pas de prix. De même, c'est
cette période qui est souvent mise en scène dans les films de sabre, comme
\emph{Les Sept Samurai}, \emph{Ran} ou \emph{Kagemusha} de Kurosawa, ou la
série-fleuve \emph{Shogun}, qui, à défaut d'être historiquement exacte, permet
de voir le château d'Osaka.

Une autre appellation de cette \guil{période des provinces en guerre} est
\emph{Gekokujo} : le monde à l'envers.  Effectivement, pendant près de 150 ans,
le Japon va être le théâtre d'une guerre permanente, pleine de trahisons, de
rivalités, d'alliances oubliées, de mariages arrangés pour consolider un
avantage stratégique ou enterrer des querelles, de familles qui se brisent, de
clans qui se scindent puis s'allient avec leurs anciens ennemis\dots\ Le plus
bas des fantassins peut devenir, en quelques années, le seigneur le plus
puissant du pays, puis être assassiné par son lieutenant ; les fils
s'entretuent à la mort du père, les vassaux oublient leurs v\oe{}ux de fidélité
en l'absence de leurs seigneurs, les moines fondent des ligues de soldats, et
se soulèvent contre leurs maîtres. Même l'Empereur est obligé de vendre des
estampes pour manger !

C'était donc une période ou l'efficacité brute prime sur la pureté morale. La
fin~\incise~principalement, la survie de son clan~\incise~justifie tout, ou
presque. J'en veux pour preuve le nombre de codes et préceptes, écrits par les
chefs de clan à l'adresse de leurs fils et subordonnés, les exhortant à la plus
grande fidélité~\incise~un peu comme les codes de chevalerie insistait sur les
vertus que devaient posséder ceux qui portaient l'armure. La figure du samourai
comme parangon de nobles valeurs n'existe pas encore\dots\ Pourtant, c'est dans
ce climat inconstant de trahisons que l'on peut trouver les plus grands actes
de bravoure et les plus beaux exemples de loyauté désespérée qui préfigurent
l'archétype vertueux du samouraï. Ainsi, en 1600, Torii Mototada, vassal de
Tokugawa Ieyasu, resta avec seulement 2\,000 hommes dans le chateau de Fushimi,
afin de permettre à son seigneur de regrouper ses forces, et stoppa une colonne
ennemie de 40\,000 hommes pendant une semaine~\incise~au bout de laquelle,
implacable arithmétique !, les assiégés auront tous péri. Cet acte permit à
Ieyasu de remporter la bataille décisive de Sekigahara (nous y reviendrons). La
veille de l'assaut, les deux hommes échangèrent une dernière coupe de sake, et
Mototada proposa même de réduire la garnison du château, pour accroître les
forces de Ieyasu, certain et lucide qu'il tomberait quel qu'en soit le nombre.

De même, Takeda Shingen (le chef de clan à la cavalerie puissante, mis en scène
dans \emph{Kagemusha}) et Uesugi Kenshin, nous offrent de jolies anecdotes. Ils
guerroyèrent l'un contre l'autre pendant plus de quatorze ans, au point de se
rencontrer sur les mêmes rives d'une rivière frontalière quatre fois durant
leur règne. Sur son lit de mort, Shingen aurait instruit son fils de s'appuyer
sur Kenshin ; ce même Kenshin qui, en apprenant la mort de son ennemi, aurait
pleuré la disparition d'un adversaire aussi valeureux. En 1568, une coalition
de daimyos coupent l'accès au sel à la province des Takeda~\incise~le sel était
alors une denrée précieuse, notamment pour préserver la nourriture. Uesugi
Kenshin envoya secrètement du sel aux Takeda, arguant qu'un tel acte n'était
pas honorable, ajoutant que \guil{les guerres doivent être gagnées au sabre et
à la lance, pas avec du riz et du sel}.

Les armes, armées et techniques de combat évoluent~\incise~simple
darwinisme~\incise~extrêmement vite. En grossissant un peu le trait, d'une
succession de duels d'archerie montée entre deux nobles, on passe petit à petit
à des batailles de plus en plus importantes, mobilisant de plus en plus
d'hommes. Les mentalités changent, et les généraux font appel, de plus en plus,
à des soldats issus du peuple, les \emph{ashigaru} (\guil{pieds légers}, comme
dans \emph{tsugi ashi}, les \guil{pieds chassés} que nous apprenons en aiki),
plaçant certains samouraïs à la tête de ces fantassins. Ceux-ci sont légèrement
protégés et armés principalement d'une lance. Simultanément, l'arc ne reste pas
l'apanage des aristocrates, mais les généraux comprennent bien vite l'avantage
de larges volées de flèches, quitte à ce qu'elles soient tirées par de la
piétaille. Le sabre sert là toujours d'arme d'apparat, ou
d'appoint~\incise~notamment lors du rituel, maintenant codifié, socialement
valorisé et largement utilisé, du seppuku. Les récompenses obtenues dépendant
directement de la valeur prouvée sur le champ de bataille, c'est aussi le temps
du flamboiement baroque des armures et des habits : étendards de couleurs,
bannières individuelles, décorations sur les casques\dots

Les châteaux évoluent également : l'impossibilité géographique (tremblements de
terre) et l'inutilité stratégique (pays trop montagneux pour que l'artillerie
se déplace) d'avoir des hautes murailles, alliée à la stratégie en vogue du
moment (se reposer sur une infanterie nombreuse) fait que les chateaux forts
japonais n'ont pas du tout la même tête que nos châteaux médiévaux. Plutôt que
des murailles larges et hautes, ils sont souvent bâtis en haut d'une colline
surplombant un intérêt stratégique, avec un système de forteresses secondaires,
permettant aux défenseurs de se replier au fur et à mesure, et d'enceintes
successives, formant un labyrinthe mortel pour les assaillants. Ainsi, un
attaquant devait parcourir plus de quatre kilomètres pour couvrir les cinq cent
mètres à vol d'oiseau séparant l'enceinte extérieure du donjon principal du
château de Himeji (si ma mémoire est bonne) !

Des conscrits à la lance, en masse ; des archers, au début du combat ; des
samourais de petit rang, qui se battaient parfois à pieds, à la lance ou au
glaive ; des samourais de haut rang, commandant tout ce petit monde\dots\ Et en
1543, un événement de taille va bouleverser tout ça. Sur l'île Tanegashima, au
sud de Kyushu, un bateau portugais, en route vers la Chine mais qui avait dévié
de sa route, s'échoue.  Il contenait deux choses révolutionnaires : un
missionaire catholique, et une arquebuse. Extrêmement vite, les armes à feu
envahissent tous les champs de bataille, aidé par le savoir-faire japonais de
fabrication de l'acier (cf., si besoin était, la qualité de leurs lames). Moins
de quinze ans plus tard, Date Masamune, le daimyo contrôlant la pointe nord du
Japon, à 3\,000 kilomètres de Tanegashima, en utilise. Le Japon a été tellement
enthousiaste de ces nouvelles armes qu'il aurait dépassé, en nombre, tous les
pays européens à la même date.  Et, bien que les fusils restaient primitifs et
encombrants, ils se révèlent bien souvent décisifs. Ainsi, en 1575, à la
bataille de Nagashino, opposant le fils de Takeda Shingen à Oda Nobunaga, et
mis en scène dramatiquement à la fin de \emph{Kagemusha}, c'est par une
tactique de tirs en série disciplinés que la charge effrayante de la cavalerie
lourde des Takeda fut cassée. On voit d'ailleurs là l'ambivalence morale
japonaise : d'un côté, une arme techniquement très efficace ; mais, bien peu
honorable, puisqu'un paysan boueux et crotté pouvait, en quelques jours de
formation, tuer un noble aristocrate, lettré et raffiné, qui s'entraînait aux
armes depuis trente ans, quel scandale ! Quant au christianisme, il s'est
également diffusé comme une trainée de poudre dans tous le pays. Les historiens
estiment qu'en une cinquantaine d'années, 300 000 Japonais étaient baptisés.
Nous y reviendrons dans notre prochain chapitre d'ailleurs. 

On distingue habituellement trois grandes figures unificatrices du Japon : Oda
Nobunaga, Toyotomi Hideyoshi puis Tokugawa Ieyasu. Un dicton prétend que
\guil{Nobunaga confectionne le gâteau de riz, Hideyoshi le pétrit, et Ieyasu
s'assoit et le mange}. Si Oda Nobunaga n'a pas pu, avant sa mort, unifier
toutes les provinces du Japon, c'est lui le premier qui décide de ré-établir un
semblant d'autorité dans le pays, en forçant l'Empereur à nommer un de ses
pantins comme shogun. En 1582, alors qu'il part rejoindre un allié en
difficulté~\incise~un certain Toyotomi Hideyoshi~\incise~il est trahi par un de
ses lieutenants, qui le force à se suicider. Hideyoshi apprend la mort de son
seigneur, et rentre d'urgence pour châtier le traître. De fait, il récupère
alors les troupes d'Oda et se retrouve catapulté à la tête d'un des clans les
plus puissants. Notons que Hideyoshi est fils de fermier, a commencé comme
domestique d'un vague lieutenant, et est laid comme un cul de cynocéphale (on
le surnomme, dans son dos bien sûr, \guil{face de singe}~\incise~d'où le titre
de sa biographie écrite par Shiba Ryotaryo, \emph{Seigneur-Singe}). Ambitieux,
il soumet clan par clan, bataille par bataille, chaque famille, débarque en
1587 sur l'île de Kyushu avec 150\,000 hommes, et assoit sa domination sur la
totalité du Japon en 1590. Dans une poussée d'orgueil, il pose même ses yeux
vers la Chine, et débarque, en 1592 en Corée, avec 200\,000 hommes. À la suite
d'une conquête facile, grâce notamment à des combattants entraînés et
disciplinés, équipés d'armes à feu, face à une armée prise par surprise et peu
nombreuse, la Corée parvient cependant à bouter les envahisseurs grâce à sa
flotte. Celle-ci est extrêmement moderne~\incise~le bateau-tortue, outre une
puissante artillerie embarquée, était le premier navire cuirassé au monde, 275
ans avant le premier cuirassé occidental !~\incise~et commandée par un maître
stratège, véritable héros en Corée, Yi Sun-Sin. Résultat, les Japonais font
marche arrière, évacuent en ne laissant guère que quelques fortifications
derrière eux~\incise~et un paquet d'exactions qui préfigurent celles qui
viendront trois cent ans plus tard.

En tous les cas, Toyotomi Hideyoshi meurt en 1598, et son ami, Tokugawa Ieyasu,
également un ancien lieutenant (et otage !) d'Oda qui prend la tête des
troupes. Mais tous les daimyos ne sont pas d'accord ! Le Japon est encore
divisé en deux clans, l'un favorisant Ieyasu, l'autre préférant le fils
héritier de Hideyoshi.  La titanesque bataille de Sekigahara, en plein centre
du pays, décide du vainqueur : plus de 24 heures de bataille, qui s'ouvrent sur
un brouillard d'octobre épais, 200\,000 combattants, équitablement répartis,
des charges héroïques, des trahisons en plein milieu, des arrivées
providentielles, des éclairs de génie tactique qui finalement ne marchent
pas\dots\ Y'aurait moyen d'en faire un flim sacrément épique. Un certain
Musashi Miyamoto, d'ailleurs, y participe~\incise~et est dans le camp des
vaincus. Ieyasu remporte de manière décisive, face aux autres forces féodales
qui s'opposaient à sa prise de pouvoir, se fait nommer shogun, place sa
capitale à Edo, un petit village (qui s'appellera Tokyo dans 250 ans), et fonde
une dynastie qui garantira enfin la paix à son pays. Nous verrons ça la
prochaine fois\dots
