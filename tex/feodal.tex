\chapter{Ère féodale}

\guil{Civilized men are more discourteous than savages because they know they
can be impolite without having their skulls split, as a general thing.}

Nous avions laissé notre voyage en 1185, alors que le Japon n'est plus gouverné
par un Empereur et sa cour, mais par un ancien chef de guerre, le shogun. Les
administrateurs et fonctionnaires ne comptent plus parmi les plus puissants ;
une société féodale se met en place, avec la caste du guerrier à sa proue. Ces
guerriers (les \emph{bushi}) jurent fidélité à leur clan, ce qui, par
transitivité de loyauté, revient à jurer fidélité à l'Empereur. Ils demandent
la même loyauté aux paysans, lesquels louent leurs terres à l'Empereur (donc
aux clans), et payaient leurs baux en nature.

Dans la pure tradition bien japonaise des régents et du pouvoir entre les mains
de celui qui n'est pas au pouvoir, à la mort du premier shogun, c'est le clan
de sa femme, les Hojo, qui règnent dans les faits, et occupent le rôle de
\emph{shikken}, comme régents du shogun. La capitale impériale reste à Kyoto,
et est considérée comme plus raffinée et cultivée que Kamakura, vue comme la
ville des gros bourrins de guerriers~\incise~mais, à tout le moins,
indépendante des intrigues bureaucratiques de la cour.

Un courant du bouddhisme, alors minoritaire, s'attire bientôt les faveurs des
samouraïs : le zen, avec l'importance donnée à l'ascèse martiale, à la
méditation disciplinée, au renoncement et à ce lien intime et mystérieux avec
la mort, séduit vite les guerriers. Cette importance du zen se voit encore dans
virtuellement tous les arts martiaux japonais~\incise~j'entends encore mon
premier prof de kendo me dire, après avoir enchaîné une dizaine de coupes sur
mon casque sans que je ne puisse ni les voir, ni encore moins faire grand-chose
pour les éviter : \guil{Évidemment que j'arrive à te toucher aussi
facilement\dots\ Tu es ailleurs ! Il faut être \emph{ici et maintenant} !}

Un événement exogène vient cependant perturber la situation actuelle : en 1274,
puis en 1281, Kubilai Khan, le petit-fils de Gengis Khan, pose ses yeux sur ces
petites îles. C'est le thème du second livre de Julien Peltier (\emph{Le Sabre
et le Typhon}), et je vous le conseille pour tout savoir de ces invasions.

En deux mots, cependant : les Mongols débarquent (un peu) par surprise, avec
une force supérieure en nombre, très disciplinée, mais surtout, totalement
étrangère aux Japonais. Aux défis lancés noblement par des cavaliers s'avançant
solitaires sur le champ de bataille, ces enflures répondent par des volées de
flèches lancées par de la piétaille populacière bigarrée. Scandale ! De même,
ils ne laissent pas tranquilles les écuyers chargés de récupérer les têtes des
ennemis tués par leur maître, afin de prouver, de fait, leur valeur, et de
demander une récompense en conséquence de ces trophées. Résultat, les Japonais
sont forcés de changer de tactique, et c'est une série d'escarmouches héroïques
qui leur permet de garder les envahisseurs bloqués dans leurs bateaux,
attendant des renforts. Là, un typhon providentiel ravage la flotte mongole qui
mouillait près des côtes japonaises : c'est le \emph{kamikaze}~\incise~mot à
mot, le \guil{vent divin}~\incise~qui reviendra d'ailleurs sept cent ans plus
tard, portant les chasseurs Zéro à Pearl Harbor.

Ces invasions mongoles, bien qu'une victoire japonaise, est un désastre pour le
shogun. D'une, la peur d'une troisième invasion subsistait, et le shogunat doit
racler le fond de ses caisses pour maintenir une armée prête à défendre les
côtes. De deux, les samouraïs n'ont rien conquis, et il est impossible pour le
shogunat de les récompenser avec des richesses ou des terres. De trois enfin,
un siècle après en 1278, à la mort de l'Empereur du moment, la famille
impériale tente de se rebiffer en se plaçant en désaccord sur les affaires de
succession avec le shogunat affaibli.

Ce sursaut de volonté politique cultmine en 1333, où le shogunat de Kamakura
est renversé par un coup d'état, mené par l'Empereur en date, Go-Daigo, et ses
alliés / vassaux Nitta Yoshisada, Kusunoki Masashige et Ashikaga Takauji. Je
reviendrai à la fin ces trois lascars, mais, pour la faire courte, le coup
d'état se révèle être un échec pour la lignée impériale. Le shogunat de
Kamakura, totalement dépassé par les événements, s'éteint avec lui, et c'est
Ashikaga Takauji qui fait volte-face pour prendre le pouvoir. Il écrase la
coalition de Go-Daigo et se fait proclamer shogun à la place du shogun. La
capitale shogunale revient à Kyoto, au même endroit que la cour impériale. Le
shogunat des Ashikaga est globalement similaire à celui des Minamoto d'avant, à
une exception de taille près : contrairement au bakufu de Kamakura, les
Ashikaga n'ont jamais réussi à centraliser leur pouvoir, reposant plutôt sur
une coalition vaguement stable d'une multitude pyramidale de seigneurs de
guerre régionaux. Ces seigneurs de guerre se font appeler les \emph{daimyo}
(les \guil{grands noms}), et vont avoir une importance prépondérante dans notre
prochain chapitre.

En tous les cas, la culture flamboie, le no se développe, la cérémonie du thé
se codifie, le zen explose (pour autant que le vide puisse exploser,
s'entend)~\incise~le petit fils de Takauji construit d'ailleurs le Pavillon
d'Or à Kyoto, et les shoguns, petit à petit, perdent leur influence unifiante,
notamment dans les contrées éloignées de la capitale. Les conflits de petite
envergure entre tel ou tel chef régional sont de plus en plus légion, et cette
situation atteint son apogée en 1467, à l'occasion, vous aurez deviné, d'une
dispute de succession. Le 8\up{e} shogun de la dynastie de Ashikaga n'a pas
d'héritier, demande à son frère cadet, alors moine, de quitter les ordres,
puis, merde, un fils lui naît. Deux clans ennemis sautent sur l'occasion, et en
viennent même à se battre dans les rues de la capitale, pendant dix
ans~\incise~c'est dire à quel point l'autorité du shogun avait décliné, lequel
shogun d'ailleurs, un peu comme Néron, était en train de dessiner les plans
d'un Pavillon d'Argent pendant que sa ville brûlait et que son pays plongeait
dans la guerre civile. C'est la Guerre d'Onin, qui sert classiquement de point
de départ de la prochaine étape de notre voyage, l'époque Sengoku : l'âge des
provinces en guerre.

J'ouvre une parenthèse en revenant sur la restauration ratée de l'Empereur en
1333. Kusunoki Masashige est un tacticien brillant, loyal à l'Empereur, qui lui
a permis de revenirau pouvoir (quoique brièvement), en défendant avec succès
deux forteresses clefs. Quand Ashikaga Takauji trahit la cause impériale et se
retourne contre ses anciens alliés pour récupérer le pouvoir, Masashige
conseille à l'Empereur de se réfugier dans le temple bouddhique du mont Hiei,
colline sacrée surplombant Kyoto, et de laisser son ennemi reprendre la
capitale pour mieux fondre sur lui ensuite. Go-Daigo, orgueilleux, totalement
réactionnaire, impatient dans ses réformes, cruel envers les paysans et
méprisant envers les guerriers, refuse, et ordonne à son lieutenant de
confronter ses forces épuisées, affamées et largement inférieures à celles de
Takauji, dans les plaines, en dehors de Kyoto. Masashige, convaincu de sa
défaite~\incise~et de sa mort~\incise~obéit dans un acte ultime de loyauté
envers son suzerain, et rencontre Takauji à côté de la future Kobe, en laissant
son poème de mort à son jeune fils. La légende dit que son frère lui aurait
lancé un cri qui sera maintes fois repris dans l'histoire, \guil{Shichisei
hokoku !} (\guil{Que j'ai sept vies à donner pour mon pays !}), ce à quoi
Masashige, fidèlement entêté, et donc totalement en opposition à la
réincarnation karmique des bouddhistes, aurait répondu par une charge
désespérée. Son armée totalement encerclée, lui restant moins de 50 cavaliers
sur les 700 initiaux, il meurt sabre à la main, sans jamais avoir eu aucune
chance de succès.

Aucun espoir de victoire, servant un seigneur réactionnaire, réfractaire à la
moindre idée de progrès, et sans avoir accompli la moindre chose\dots\ et
pourtant, c'est Kusunoki Masashige qui représente le plus grand héros dans la
psyché japonaise. De son côté, Ashikaga Takauji est vu comme un vil
traître~\incise~quand ce fut, historiquement, un tacticien excellent, un lettré
raffiné et un politicien brillant. C'est là à mon sens la différence
fondamentale entre l'archétype du héros japonais et son pendant occidental, et
celle qui personnellement me touche au plus haut point. En Europe, un héros va
être un preux chevalier, moralement irréprochable, qui fonce dans un combat
quasi-impossible, mais qui ressort victorieux, grâce à son talent et, souvent,
à un peu de chance, en dépit des probabilités qui le donnent perdant. Quand
hélas il meurt, c'est toujours en donnant naissance à quelque chose de plus
grand, afin que ses vertus lui survivent. Il se sacrifie, d'une certaine
manière, pour que ce en quoi il croyait, forcément moralement irréprochable, se
réalise. Au Japon, au contraire, c'est celui qui échoue qui est glorifié. En
effet, celui qui réussit n'a pu le faire qu'en accommodant ses valeurs, pour
les mettre en adéquation avec la réalité, forcément imparfaite.  Alors que
celui qui meurt dans l'échec le plus total, c'est parce qu'il est resté têtu,
juste, sincère, véridique, fidèle~\incise~et ce, quelles que soient ses
croyances. Celui qui échoue, c'est celui qui n'a jamais compromis, qui est
resté pur. C'est, moins l'efficacité ou le bien-fondé rationnel des valeurs
d'une personne que l'expression de son intensité qui en fait la qualité.

Deux héros occidentaux possèdent, à mon goût, ces caractéristiques : don
Quichotte et Cyrano (\guil{Que dites-vous ? C'est inutile ? Je le sais ! Mais
on ne se bat pas dans l'espoir du succès ! Non, non ! C'est bien plus beau
lorsque c'est inutile !}), et ce n'est pas bien surprenant que les westerns,
notamment de Leone, s'inspirent des histoires japonaises, pleines de ces coups
d'éclat stupides et beaux : les duels à mort, entre deux experts,
irrationnellement têtus, vivant ce qu'ils croient plutôt que de professer,
nobles dans leurs vertus, mais des dinosaures bientôt oubliés dans les faits,
le six-coups à la ceinture, totalement incongru à l'âge où le train à vapeur
arrive\dots\ Voilà une image que les bretteurs de Muromachi n'auraient pas
renié. Les Japonais appellent cette vertu-là \emph{makoto}, habituellement
traduit par \guil{sincérité}. C'est une des sept vertus du guerrier, vertus qui
sont, comme le veut la tradition, symbolisées par les sept plis du
hakama~\incise~et vous comprenez pourquoi il faut s'efforcer de soigneusement
ranger sa jupette à la fin du cours.
