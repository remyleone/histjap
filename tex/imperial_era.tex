\chapter{Ère impériale}

\guil{He shrugged his shoulders. \guiluk{I have known many gods. He who denies
them is as blind as he who trusts them too deeply. I seek not beyond death. It
may be the blackness averred by the Nemedian skeptics, or Crom's realm of ice
and cloud, or the snowy plains and vaulted halls of the Nordheimer's Valhalla.
I know not, nor do I care. Let me live deep while I live; let me know the rich
juices of red meat and stinging wine on my palate, the hot embrace of white
arms, the mad exultation of battle when the blue blades flame and crimson, and
I am content. Let teachers and priests and philosophers brood over questions of
reality and illusion. I know this: if life is illusion, then I am no less an
illusion, and being thus, the illusion is real to me. I live, I burn with life,
I love, I slay, and am content.}}

Le XIX\up{e} siècle se finit : le Japon sort tout juste d'une période de
changements profonds, qui ressemble cependant plus à un dernier baroud
d'honneur de dinosaures conservateurs démodés qu'à une révolution sanglante. La
modernisation du pays, encore une féodalité médiévale en 1850, est fulgurante :
grands chantiers publics, signature d'une Constitution nationale, création d'un
parlement, refonte des institutions politiques, émergence d'une presse
importante, création d'une monnaie fiduciaire unique et d'une économie de
marché, montée en puissance de grands groupes industriels\dots\ Ces derniers
ont à leur tête d'anciens chefs de clans, et deviendront plus tard les grandes
firmes \emph{zaibatsu}, comme Mitsui ou Mitsubishi. En trente ans, l'Empire du
Japon possède une force que les Empires occidentaux ont acquis en plus d'un
siècle.

Ce formidable développement du Japon se mue peu à peu en complexe de
supériorité, puis franchement en impérialisme. Devant l'humiliation des nations
voisines, en particulier celle de la Chine~\incise~cf. les guerres de
l'opium~\incise~des intellectuels du Japon proposent le concept de \emph{zone
d'influence}, dans laquelle le Japon sortirait de ses frontières pour se garder
une zone \guil{tampon} dans le cas d'une attaque étrangère. Une série de
tensions entre la Corée et la Mandchourie culmine en 1894, quand le Japon en
profite pour intervenir. Sous prétexte de vouloir aider la Corée, l'Empire du
Soleil Levant envoie son armée pousser ses frontières contre celle de l'Empire
du Milieu. En un an, la flotte chinoise est détruite, et une série de défaite
militaire contraint la Chine à signer un traité de paix, concédant au Japon
plusieurs territoires (Corée, Taïwan, Port-Arthur en Mandchourie\dots). Ce sont
les tensions impérialistes autour de ce port qui conduisent le Japon à déclarer
la guerre à la Russie tsariste, dix ans plus tard. Là encore, le conflit ne
dure pas longtemps, et la Russie, pas aidée par la révolution rouge d'octobre
de la même année, signe un traité de paix au bout d'un an, cédant l'île
Sakhaline au Japon. En quelques anénes, l'Empire du Japon gagne ses galons de
grande puissance : la \guil{race jaune} a en effet battu, dans un combat
régulier, un gouvernement de la \guil{race blanche}, pourtant fondée à
gouverner le monde. Cette surprise viendra conforter la dangereuse évolution du
climat impérialiste au Japon, légitimant ses ambitions coloniales en se
qualifiant de \guil{pays des dieux}. Le Japon s'estime alors fondé à étendre
son influence colonialiste sur l'ensemble du continent asiatique. La spirale
mortelle est amorcée, et la suite n'est que trop bien connue : du massacre de
Nankin aux bombardements nucléaires, la course à l'abîme va se poursuivre avec
une ponctualité métronomique.

Mais nous n'avons toujours par parlé de l'aikido, pour l'instant\dots\
Revenons un peu en arrière : 1880, le Japon est en pleine modernisation. Les
connaissances occidentales sont considérées comme préférables aux antiques
voies. Aux arts martiaux sont préférés, notamment à haut niveau, les sports
\guil{civilisés} comme le rugby, le football ou le baseball. Parallèlement, un
fort sentiment patriotique nait au Japon. Le nouvel Empire a besoin de
symboles unificateurs forts, et va donc les chercher dans son passé. Ainsi, la
figure du héros loyal jusqu'à la mort est personnifié par Kusunoki Masashige,
le général du XIV\up{e} siècle dont nous avions parlé tantôt. Le
\emph{kamikaze}, le typhon providentiel qui a(urait) mis en déroute les
envahisseurs Mongols en 1280 est vu comme une preuve du caractère sacré et
inviolable du Japon. Nitobe Inazo, un universitaire formé par les Jésuites,
théorise un code moral dans \emph{Bushido, l'Âme du Japon}. Ce livre, à peu
d'être un outil de propagande impérialiste, décrit les qualités qu'auraient
possédées les guerriers d'antan, et expose comment les appliquer aujourd'hui.
La loyauté envers l'Empereur, issue d'une vision confucéenne du monde, et le
sacrifice de soi sont particulièrement mis en avant~\incise~ce qui permettra
aux militaires ensuite de justifier un paquet de morts\dots

C'est dans cet état d'esprit bicéphale, modernisation \emph{vs.} traditions,
toujours terriblement présent de nos jours, qu'un petit enseignant de jiujitsu
arrive sur le devant de la scène. Je crois que Jean-Louis a lu extensivement
sur le sujet, et je le laisserai donc me corriger et compléter. Kano Jigoro,
1m56, 40 kilos tout mouillé, est un haut-fonctionnaire du gouvernement, chargé
de superviser les sports et leur éducation dans l'Empire. Il a étudié
longuement avec différents maîtres de différentes écoles, et possède un style,
qu'il a nommé \emph{judo} (la discipline de la souplesse). En dépit de
l'aspect résolument japonais de son art, ses méthodes d'enseignement, issues
des États- Unis, et sa philosophie sous-jacente, lui amènent un soutien de
l'État. Plutôt que les sempiternels exercices de gymnastique effectués en
masse dans les cours de récré, le judo permet un développement physique
\emph{et} mental, une forme d'enseignement moral étant en effet au coeur du
judo. Kano accepte cependant difficilement la récupération de son art par les
militaires, qui voient là une manière d'avoir aisément des recrues loyales et
en bonne santé, et se place en opposant majeur au fascisme japonais. En tous
les cas, le point est fait : les arts martiaux ancestraux peuvent évoluer. Ce
qu'on appelait les \emph{koryu} (\guil{anciennes écoles}) deviennent des
\emph{gendai budo} (\guil{arts martiaux modernes}). Notons qu'un art martial a
conservé tous ses aspects traditionnels, et a (relativement) peu changé depuis
deux mille ans, le sumo.

Là-dessus, un jeune homme maigrichon et pas bien grand étudie dès 1900 de
multiples écoles de jujitsu. Cinq ans plus tard, il décide de s'engager dans
l'infanterie, mais sa petite taille le lui interdit. À 1m56, il ne lui manque
que quelques centimètres. Qu'à cela ne tienne, il passe des heures suspendu à
un arbre avec des poids au pied, et participe à la guerre russo-japonaise en
Mandchourie. 1912 est une révélation : parti fonder un village à Hokkaido, il
rencontre le grand maître de l'école Daito de jujutsu~\incise~l'ancienne école
du clan Takeda, les cavaliers féroces du Kai. Il rencontre ensuite un des
fondateurs d'une secte shinto, qui donnera à son art une dimension spirituelle
forte. Dans la brèche ouverte par Kano Jigoro, il construit petit à petit une
discipline qui adapte les techniques guerrières ancestrales en réussissant à
les retourner, pour que les volontés de chacun, fussent-elles de nuire,
s'unissent plutôt que s'opposent. Cet homme, c'est O Ueshiba Morihei Sensei
(\guil{le grand professeur Ueshiba Morihei}), le vieillard à la peau
parcheminée et aux regard doux, perçant et impassible, que nous saluons à
chaque fois que nous rentrons dans le dojo.
