\chapter{Une anecdote pour finir}

Désolé, je sais que je n'ai pas tenu mon engagement d'être court et
synthétique, mais je ne résiste décidément pas à vous proposer une histoire et
énigme, à propos de \textcolor{red}{Tanaka Shinbei} :

Une nuit sans lune, le 20 mai 1863, \textcolor{blue}{Anenokoji Kintomo}, un
chef radical à la cour de \textcolor{blue}{l'Empereur}, et un champion des
forces contre le \textcolor{red}{Shogun}~\incise~alors l'actuel centre de
pouvoir~\incise~rentra chez lui, marchant du \textcolor{blue}{palais impérial
de Kyoto} à sa résidence. La route était sombre, et ses pas étaient simplement
éclairés par une lanterne de papier portée par un de ces cinq pages.
Soudainement, un attaquant vêtu de noir, alors caché dans des fourrés, sauta
devant lui, dégaina et lui coupa le visage de haut en bas, avant de
disparaître aussi vite. Quatre des cinq domestiques du noble s'enfuirent, et
le cinquième courut à la recherche de l'assassin alors que le sang coulait du
front de son maître. Un autre assaillant arriva par derrière, et planta son
sabre profondément dans le dos \textcolor{blue}{d'Anenokoji}. Un troisième
agresseur le coupa au niveau du torse. Incapable de crier, le jeune noble
s'écroula à terre. Aidé par son page qui revint, \textcolor{blue}{Anenokoji}
boita une centaine de mètres avant de s'effondrer totalement. Ce leader du
mouvement d'expulsions des barbares étrangers et de renverser le Shogunat
était mort, à vingt-cinq ans.

Les discussions pour savoir qui était le meurtrier allaient bon train. Les
rues de Kyoto n'étaient pas sûres depuis quelques années, et franchement
sanglantes depuis quelques mois. L'anarchie, qui dans l'Histoire a toujours
précédé aux changements de régime, grondait farouchement, et en particulier
ici, comme nous venons de le voir. \textcolor{blue}{Anenokoji} revenait
d'ailleurs d'une discussion animée avec d'autres officiels de la
\textcolor{blue}{cour impériale}, dont le consensus était d'émettre un décret
scellé par l'Empereur obligeant le clan de \textcolor{blue}{Choshu} à
renverser \textcolor{red}{le pouvoir en place}. \textcolor{blue}{Choshu}, ce
sont les mecs de la bataille de Shimonoseki, si vous vous souvenez bien.

Tout le monde pointait le doigt vers les autres : Satsuma (le fief du grand
Saigo), un autre clan, voisin de Choshu et son grand rival politique, était
moins extrême que les gros tarés de Choshu : il préférait une alliance entre
la cour (\textcolor{blue}{impériale}) et le pouvoir en place (le
\textcolor{red}{shogun}), sous la forme d'un mariage entre la soeur de
l'Empereur, l'Impératrice Kazu, et \textcolor{red}{l'actuel shogun, Iemochi} ;
le daimyo de Choshu, au contraire, était partisan d'une guerre ouverte contre
le shogun d'une part et les étrangers de l'autre (d'où la bataille de
Shimonoseki). Satsuma accusait donc Choshu d'avoir buté un type qui risquait
d'argumenter en faveur d'un gouvernement hybride, de cohabitation. Choshu,
pour sa part, suspectait Satsuma d'avoir zigouillé un mec qui jouissait à la
cour de l'Empereur d'un réel pouvoir, et qui risquait de pencher au contraire
vers une guerre totale. D'autant plus que les Choshu ignoraient les intentions
profondes de Satsuma : si la rumeur disait vrai, le daimyo de Satsuma, Shimazu
Hisamitsu, allait mener un contingent de ses troupes dans Kyoto, la capitale
impériale, afin de tordre le bras à l'Empereur pour le forcer à déclarer la
guerre contre le shogunat. Ce dont rêvait Choshu, mais ça les faisait chier
que ce soit quelqu'un d'autre qu'eux qui mène l'armée.  Et, finalement, tout
le monde soupçonnait le shogunat d'avoir éliminé un dangereux ennemi.  Bon, et
y'a plein d'autres trucs (une vingtaine de jours avant sa mort,
\textcolor{blue}{Anenokoji} avait passé du temps à bord d'un cuirassé du
shogun, en compagnie de \textcolor{red}{Katsu Kaishu}, le brillant futur
ministre de la marine, et le plus fervent défenseur de l'ouverture du pays).
Mais bon, l'un dans l'autre\dots\ L'important : c'était le bordel.

Là où l'affaire se corse, c'est quand on apprend qu'un wakizashi, le sabre
court, avait été retrouvé sur les lieux du crime. Sur le fourreau du sabre
était peint, comme c'est la coutume, le blason du clan dont appartenait le
samurai qui possédait le sabre\dots\ Une croix dans un rond~\incise~Satsuma !
Les attaques du domaine de Choshu se firent plus insistantes, tandis que
Satsuma accusait Choshu d'avoir volé le sabre d'un de leurs guerriers, pour le
poser là comme preuve. Ceci était d'ailleurs assez crédible, dans la mesure où
c'était Choshu qui était actuellement chargé de garder le Palais Impérial, et
donc, subséquemment, gérait des troupes provenant de plusieurs contrées.

Le possesseur du sabre a d'ailleurs été identifié, via les gravures dans le
manche : un certain \textcolor{red}{Tanaka Shinbei}, samurai de Satsuma, et
surtout, assassin notoire. Assassin, oui, membre des Quatre
\guil{\textcolor{red}{Hitokiri}}, les trancheurs de gens, et sa brutalité
était bien connue~\incise~et il avait démontré sa maîtrise du sabre en de
nombreuses occasions~\incise~principalement des contrats pour des
\textcolor{red}{partisans du shogun}. Ok, sauf que :
\textcolor{blue}{Anenokoji} a été attaqué par trois hommes~\incise~ce qui
n'est pas le mode opératoire de \textcolor{red}{Tanaka}~\incise~et surtout,
pas un seul n'a été capable de lui délivrer un seul coup fatal ! D'ailleurs,
un assassin du calibre de Tanaka n'aurait jamais laissé son sabre
derrière~\incise~un sabre encore dans son fourreau, pas taché de sang.
Ajoutons à ça que tous les amis et camarades de Tanaka ont tous juré l'avoir
vu dans un maison de geisha à une centaine de kilomètres de Kyoto cette nuit-
là~\incise~une promesse de samurai, quelque chose à ne pas prendre à la
légère.

Et pourtant, la preuve circonstancielle demeurait : le sabre de
\textcolor{red}{Tanaka Shinbei} avait été trouvé là, et il fut arrêté quelques
jours plus tard.

Nous avons l'avantage de l'Histoire avec nous : dans son cachot, attendant son
procès imminent, Tanaka a écrit une lettre. Elle enjoignait un de ces proches
de ne pas informer les autorités du vol de son waikikizashi, survenu dans un
bordel de Kyoto une semaine avant le meurtre. Il lui demandait par écrit de ne
pas soulever la preuve irréfutable, aujourd'hui retrouvée d'ailleurs, de ce
vol, puisqu'il en avait rapporté à son supérieur (afin de récupérer un autre
sabre, partie absolue de l'uniforme de la caste des samurai). Le procès-verbal
de son procès nous apprend qu'il s'est déroulé ainsi : Tanaka Shinbei s'est
présenté, en blanc et non menotté (c'est la coutume : on ne lie pas un
guerrier, ce serait mettre sa parole en doute, et donc, l'insulter gravement)
devant ses interrogateurs. La première question fut : \guil{est-ce votre sabre
?}, demanda un des enquêteurs, le sabre à la main. \guil{Je ne sais pas, je ne
peux pas le voir convenablement d'ici. Je dois le voir de plus près.} répondit
Tanaka. Alors que l'on lui tendit son sabre, Tanaka mit lentement la lame à nu
pour l'examiner. \guil{Oui, c'est bien mon sabre}, déclara-t-il.
\guil{Cependant}, ajouta-t-il avec force, \guil{je n'ai pas tué Anenokoji.} --
\guil{Si vous ne l'avez pas tué, qui\dots} Avant que l'interrogateur ne puisse
finir sa question, et à la surprise des autorités restées interdites, Tanaka,
d'un geste décidé, retourna son sabre vers lui, plongea la lame en son flanc
gauche, et, sa main gauche rejoignant sa main droite, poussa le tranchant de
l'autre côté de son ventre. Puis, dans la même rafale de volonté fanatique,
ramena la pointe de sa lame sur son cou, et s'ouvrit la carotide, tombant vers
l'avant sans le moindre cri.

\emph{On a side note}, les enquêteurs conclurent à la culpabilité de Tanaka Shinbei.
Satsuma tomba en disgrâce (temporairement), et ce fut Choshu et ses radicaux
qui régnèrent (un temps) en maître sur l'ancienne capitale de l'Empereur,
devenue lieu de toutes les agitations les plus frénétiques. Alors,
question\dots\ Pourquoi, ce geste absurde et déraisonné de Shinbei, alors qu'il
pouvait prouver son innocence ? Quel gâchis ! Et si ça peut vous aider, sachez
que Mishima Yukio, l'écrivain génial, avait fait des pieds et des mains pour
pouvoir jouer ce rôle-là dans le \emph{Hitokiri} de Hideo Gosha.
