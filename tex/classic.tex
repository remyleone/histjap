\chapter{Ère classique}

\guil{\guiluk{Barbarism is the natural state of mankind,} the borderer said,
still staring somberly at the Cimmerian. \guiluk{Civilization is unnatural. It
is a whim of circumstance. And barbarism must always ultimately triumph.}}

Notre voyage commence aux tous premiers siècles de notre ère, vers 450. En
Europe, Rome n'est plus depuis peu, les barbares (Goths, Vandales\dots)
déferlent un peu partout. À quelques milliers de kilomètres de là, l'écriture
par idéogrammes est arrivée au Japon \emph{via} la Chine. Une sorte de
paysannerie clanique émerge alors, exploitant les rares terres cultivables
(faut se rappeler que le Japon est majoritairement constitué de montagnes),
concentrées sur les deux plaines très fertiles du Kinai (autour d'Osaka et de
Kyoto) et du Kanto (autour de l'actuelle Tokyo)~\incise~voir
\textcolor{blue}{\underline{\texttt{CARTE DU JAPON}}} par exemple pour avoir
une idée de quoi que j'cause. Petit à petit, les clans se rassemblent, et une
sorte de gouvernement se forme, à l'image des usages chinois, avec une
bureaucratie civile centralisée. Son unité se cristallise autour de la figure
de l'Empereur, issu d'une lignée héréditaire qui continue encore de nos jours,
et qui prétend remonter aux temps mythologiques de la déesse du soleil
Amaterasu.

Là, on voit une logique qui va perdurer encore pendant un bon millénaire,
quoique sous différentes formes : ce proto-gouvernement demande effectivement à
certains chefs de clan d'aller guerroyer contre les peuples aborigènes (Aïnus,
Emishis), notamment vers le nord et l'est, en leur promettant les terres qu'ils
auront réussi à conquérir. Parallèlement, d'autres chefs de clans guerriers,
dépossédés par l'évolution administrative, sont attirés par la capitale, et
contribuent à son évolution sociale, religieuse, culturelle~\incise~mais
laissent petit à petit leur pouvoir entre les sabres des autres.

Mais nous n'en sommes pas là\dots\ Pour l'instant, vers 700, la capitale est à
Nara, avec un empereur fort, gouvernant de manière absolue une structure
organisée rationnellement, en divisant les terres de manière (plutôt) équitable
entre chaque sujet. L'aristocratie, rendue puissante par des impôts
efficacement collectés, se lance dans des grands travaux
publics~\incise~routes, irrigations, temples\dots\ La culture y est d'ailleurs
florissante, avec les deux chroniques majeures, relatant les origines
mythologiques du Japon (\emph{Kojiki} et \emph{Nihon Shoki}) écrites en ce
temps-là, et le \emph{Man'yoshu}, une collection de 5\,000
poèmes~\incise~principalement des tankas, les ancètres des haikus (les poèmes
courts en 5-7-5 pieds). Tiens, autre événement qui augure une logique
récurrente : la capitale se déplace de Nara à Kyoto, vers 800, pour échapper à
la grogne des prêtres bouddhiques. Au Japon en effet, au moins jusqu'en 1600,
les prêtres vont posséder un réel pouvoir politique, au même titre qu'un clan
guerrier. Le bouddhisme ne rime pas (encore) avec non-violence, et les ordres
religieux entretiennent des milices privées de taille conséquente, avec des
milliers de mercenaires et de moines-soldats, réputés pour leur habileté
notamment à la hallebarde (\emph{naginata}).

Les terres étaient donc divisées en parcelles entre les sujets de l'empereur.
La contrepartie était double : des impôts, en nature ou en corvées à payer, et
la possibilité d'être mobilisé sur ordre du pouvoir exécutif. En théorie, cette
idée du service militaire aurait pu marcher ; en pratique, les paysans mal
entraînés et indisciplinés cherchent par tous les moyens à se dérober devant la
conscription. Ainsi, les plus fortunés vont en payer d'autres pour se
soustraire aux trois années de service, et, petit à petit, c'est une armée
quasi-professionnelle qui se met en place. L'intérieur du Japon étant quasiment
entièrement pacifié, les craintes viennent d'une invasion extérieure. Et le
gouvernement perd petit à petit de l'influence au niveau local, ce qui augmente
l'instabilité sociale, notamment dans les provinces éloignées de la capitale.
Les fonctions de maintien de l'ordre, puis judiciaire se \guil{privatise}, au
profit d'archers montés~\incise~les futurs samouraïs.

Un point étymologique, tiens : \guil{samurai} viendrait du verbe
\guil{saburau}, qui signifie \guil{servir} ou \guil{attendre à côté de}, et
était utilisé, dans cette situation, pour désigner, par glissement, les
fonctionnaires qui servaient sur le terrain les intérêts des magistrats restés
à la cour.

Mais Kyoto est le fief de la famille Fujiwara, qui, par un jeu d'intrigues
politiques largement plus complexes que dans Game of Thrones, de mariages
arrangés et de régences dans l'ombre, réussit à placer ses protégés au sommet
de l'État, décidant de la politique du pays à la place de l'Empereur. À partir
de ce moment-là, vers 700, le pouvoir politique réel ne sera plus entre les
mains du Fils du Ciel, mais sera tenu par la noblesse de cour, puis par des
seigneurs de guerre (les shoguns), puis l'armée (entre 1880 et 1945) et, plus
récemment, par le Premier Ministre.

Fast-forward vers 1100 : le pouvoir impérial est entièrement dépendant des
familles martiales (les \emph{kuge}), hissées à la tête du gouvernement à la
suite de réussites et récompenses militaires locales ; Dame Shikibu Murasaki
écrit \emph{Le Dit du Genji}, un des plus vieux romans au monde ; le bouddhisme
continue de se développer dans les îles. La plus importante des familles
contrôlant le Trône reste les Fujiwara, mais un paqueton d'autres gravitent
autour, notamment les Taira et les Minamoto\dots\ qui, à la suite d'une flambée
d'actes de sédition, se foutent allégrement sur la tronche en 1180. C'est la
guerre du Genpei, nom formé des initiales des deux clans. En effet, il y a deux
manières de lire les idéogrammes japonais : la lecture japonaise,
\emph{kun'yomi}, et celle, chinoise traditionnelle, \emph{on'yomi}, dans
laquelle Minamoto se lit \guil{Genji} et Taira \guil{Heike}.

Chose intéressante : c'est pendant la première bataille de la guerre du Genpei
que l'on assiste pour la première fois au \emph{seppuku} (ou \guil{hara-kiri},
si on lit les mêmes idéogrammes en kun-yomi) d'un guerrier, le suicide rituel
tant ancré dans l'imaginaire qu'il n'a pas quitté l'histoire, des 47 Ronin du
XVIII\up{e} siècle à Mishima en 1970. La guerre, le domaine de nobles qui
cherchent à récupérer le pouvoir d'autres nobles, est une affaire de
professionnels : les combats ne sont pas des batailles rangées, mais des duels
codifiés d'archerie montée, ressemblant plus aux tournois médiévaux qu'aux
plages de Normandie. Les deux armées se retrouvent, de part et d'autre du
champ de bataille, et une flèche sifflante à tête creuse est lancée au début
de la confrontation, pour attirer l'attention des dieux (les \emph{kamis},
comme dans \guil{kamikaze}). Puis, un combattant s'avance, déclame son
pédigree et attend que quelqu'un de son rang accepte son invitation. S'ensuit
alors un ballet galopant où les guerriers en armure lourde vont tenter de
placer une flêche dans leur adversaire. Ces cavaliers sont accompagnés
d'écuyers et de fantassins, armés de lances.

Bon, ok, je n'ai pas été très honnête, là. Les historiens s'accordent de plus
en plus pour dire que cette description des conflits est celle,
hagiographique, des écrivains de l'époque, embelissant pas mal leurs
chroniques. Mais, à la limite, peu importe~\incise~c'est ainsi que les choses
sont restées en mémoire. En tous les cas, l'important reste que la guerre est
là éminément une affaire de prestige, entre gens d'un certain rang. On notera
également que l'arme de prédilection est l'archerie montée (la \guil{voie du
guerrier} s'appelait alors \emph{kyuba no michi}~\incise~\guil{l'arc et le
cheval}), et pas du tout le sabre. Celui-ci était long, très courbe, et
pendait à la ceinture tranchant vers le bas (comme un sabre de cavalerie
européen, si je ne m'abuse), et servait plus de symbole d'autorité que réel
outil belliqueux.

Au final, les Minamoto sortent vainqueurs de ce conflit, et les Taira sont
détruits. Là où ce conflit est historique (parce que des nobles qui se
bastonnent, il y en a toujours eu), c'est que les Minamoto, au sortir de la
guerre du Genpei, dépossèdent officiellement l'Empereur de tout pouvoir
politique, et le récupèrent entièrement. Minamoto no Yoritomo, le chef du
clan, se proclame \emph{seii tai-shogun}, littéralement \guil{le grand général
qui soumet les barbares de l'Est} (anciennement, c'était un titre donné
temporairement à un noble chargé de pacifier une région~\incise~comme les
\guil{tyrans} romains), raccourci en Shogun. Il déplace le centre du
gouvernement à Kamakura, à deux pas d'un village marécageux qui s'appellera,
d'ici 700 ans, Tokyo. Ce centre politique s'appelle le \emph{bakufu}, qui
était le nom de la tente de commandement dans les campements des armées en
campagne, montrant l'origine martiale. Chose primordiale : si, dans les faits,
l'Empereur est relégué aux rites religieux, il reste cependant,
traditionnellement, le vrai maître du Japon, de par son origine divine, le
shogun n'ayant qu'un rôle de régent. On voit là toute l'ambivalence
passionnante dans la logique japonaise : les choses dans la réalité ne sont ni
ce qui est dit, ni même ce qui est écrit. Il y a toujours un aspect de
tradition important à considérer, et ce, même si les faits vont en sens
contraire.

Attendez, ça va se compliquer encore un peu\dots\ Mais pour l'instant, closons
ce premier chapitre : l'ère féodale commence.
