\documentclass[11pt,dvipsnames,a4paper,svgnames]{book} % Classe du document
%\usepackage[utf8]{inputenc} % Codage du texte dans le source (là on fait genre on est sous X. Genre... Je sais.)
\usepackage[francais]{babel} % Typographie française
\usepackage{xspace} %sert à babel pour mieux gérer après "", toussa.
%\usepackage[T1]{fontenc} % Codage de la fonte. Permet d'avoir une césure française complète (même avec des lettres accentuées).

\usepackage{fontspec}
\usepackage{xunicode}

% Pour avoir de la couleur
\usepackage[usenames,svgnames,table]{xcolor}
% \definecolor{newgreen}{rgb}{0,0.5,0}
% \definecolor{newred}{rgb}{0.6,0,0}
% Gestion des images
\usepackage{graphicx} %\DeclareGraphicsExtensions{.jpg,.png,.gif}

\usepackage{alltt}

% En-tête et pied de page joli
\usepackage{fancyhdr}
\pagestyle{fancy}

% Normal :
\fancyhf{}
\fancyfoot[LE,RO]{\bfseries\thepage}
\renewcommand{\headrulewidth}{0pt} %tailles des traits entre les en-tête et le corps du texte
\renewcommand{\footrulewidth}{0.4pt}

% TYPOGRAPHY
\usepackage{fontspec}
\defaultfontfeatures{Mapping=tex-text} % converts LaTeX specials (``quotes'' --- dashes etc.) to unicode
\setromanfont [BoldFont={Gentium Basic Bold}, ItalicFont={Gentium Basic Italic}]{Gentium Basic}

% Pour les débuts de chapitres(après \chapter)
\fancypagestyle{plain}{
\fancyhf{} % clear all header and footer fields
\fancyfoot[LE,RO]{\bfseries\thepage}
\fancyfoot[LO,RE]{}
\renewcommand{\headrulewidth}{0pt}
\renewcommand{\footrulewidth}{0.4pt}}


% Pour la table des matières
\fancypagestyle{toc}{%
\fancyhf{} % clear all header and footer fields
%\rfoot{\thepage}
\renewcommand{\headrulewidth}{0pt}
\renewcommand{\footrulewidth}{0pt}
}

% Nouvelles commandes

\newcommand{\todo}[1]{\Large TODO : #1 \normalsize}
\newcommand{\guil}[1]{\guillemotleft~#1~\guillemotright}
\newcommand{\guiluk}[1]{\textquotedblleft\,#1\,\textquotedblright}
\newcommand{\titreun}[1]{\large \textbf{#1} : \normalsize}
\newcommand{\sstitreun}[2]{\large \textbf{#1} (\emph{#2}) \textbf{:} \normalsize}
\newcommand{\titredeux}[1]{\noindent \large \textit{#1} \normalsize}
\newcommand{\incise}{\textendash}
\newcommand{\doubleparskip}{\setlength{\parskip}{2.5ex plus 0.7ex minus 0.4ex}}

\newcounter{numeroapho}
\setcounter{numeroapho}{1} %Hop, on commence 
%\newcommand{\titretrois}[1]{\stepcounter{numeroapho} \large \textbf{\arabic{numeroapho}}~\textendash~\textsc{#1} \normalsize}
\newcommand{\titretrois}[1]{\stepcounter{numeroapho} \ \linebreak \linebreak \noindent \large \arabic{numeroapho}~\textemdash~\textsc{#1} \normalsize}
\newcommand{\stitretrois}[1]{\stepcounter{numeroapho} \noindent \large \arabic{numeroapho}~\textemdash~\textsc{#1} \normalsize}
\newcommand{\sautparagraphetrois}{\\}

\title{Historiette du Japon}
\author{Manfred}
\date{\today}

\raggedbottom %Pour éviter les paragraphes qui s'étalent sur une page entière

\begin{document}

%\maketitle


% \begin{flushleft}
% \end{flushleft}
%\end{titlepage}

% \thispagestyle{empty}


% \newpage
% \null
% \thispagestyle{empty}
% %\begin{comment}
% \frontmatter
% \tableofcontents
% \thispagestyle{toc}
% \pagebreak
% \thispagestyle{empty}

\mainmatter
\pagenumbering{arabic}


\begin{center}
  \Large
   \ 
    \linebreak
    \linebreak
    \linebreak
    \linebreak
    \textbf{Historiette du Japon}
  \normalsize
  \flushright{Michael Z.~\incise~Dernière révision : 19/06/2013}

\end{center}

\doubleparskip

Salut à tous, chers amis en pyjama blanc et jupette pour certains !

Jérôme nous avait parlé il y a quelques semaines d'un livre d'histoire
japonaise, écrit par un de ses amis~\incise~lequel donne d'ailleurs une
conférence fin juin traitant du même thème que son second bouquin, les
invasions mongoles au Japon. Il en avait parlé sur le forum du club, et
l'avait ramené un soir pour le prêter, mais il n'a pas eu le succès escompté.
Comme tout sujet, l'histoire japonaise peut paraître difficile d'accès au
début~\incise~des noms pas franchement faciles à se rappeler, des dates à
foison, d'autres considérations, d'autres valeurs morales\dots~\incise~et je
me suis dit que ça pourrait être intéressant d'en proposer une introduction.
Et puis, ça me permettra de réviser un peu, ce qui n'est pas plus mal.

Mon but est donc d'écrire quelques paragraphes que j'espère assez simples à
lire, à tout le moins pas chiants, qui survoleront 1500 ans d'histoire
japonaise. Je vais essayer de restreindre les dates et les noms, et de laisser
tomber certains détails, quitte à raconter l'histoire plutôt que l'Histoire. À
chacun ensuite, si le virus a pris, d'approfondir telle ou telle période. Je
vais également essayer de montrer, par les (r)évolutions techniques et
sociales successives, comment les batailles rangées de samouraïs ont donné
naissance aux arts martiaux que nous pratiquons chaque semaine. Il y aura six
chapitres à notre voyage. Vais essayer de tout faire de tête, histoire de voir
ce que ça donne\dots

Pour que ce soit plus facile à lire, je l'ai mis en forme, histoire d'éviter
de se ruiner les yeux sur l'écran. Bon, et maintenant que c'est en \LaTeX{},
je pourrais même rajouter des macrons sur les \={o}\dots

\textbf{EDIT, écrit à la fin :}

Comme je remarque que j'ai un peu dépassé la limite d'un paragraphe par
chapitre que je m'étais fixée, je vous propose donc une historiettette du
Japon :

\begin{enumerate}

\item 500 : des clans d'origine paysanne se regroupent, et forment un pouvoir
centré sur la figure de l'Empereur. Petit à petit, les gens à la cour se
détachent des affaires locales, tenues par les guerriers, qui récupèrent alors
le pouvoir. 1180 : un clan en particulier se révèle victorieux, et dépouille
l'Empereur de toute influence, sinon symbolique. Le pouvoir est alors tenu par
un \guil{généralissime}, le Shogun.

\item 1280 : le petit-fils de Genghis Khan essaie de débarquer au Japon, mais
n'y arrive pas. 1333 : l'Empereur du moment essaie de re-récupérer le pouvoir,
mais n'y arrive pas. Cependant, l'influence du shogun diminue de plus en plus,
et en 1450, le Japon se morcelle en une multitude de cité-états indépendantes,
qui se font constamment la guerre.

\item Tout fout l'camp ma bonne dame. 1450 $\rightarrow$ 1600, guerre civile
au Japon. Petit à petit cependant, le pouvoir se concentre dans les mains
d'unificateurs successifs, le dernier restant étant du clan Tokugawa. Aidé
notamment par l'introduction révolutionnaire des arquebuses, \emph{via}
l'arrivée fortuite d'un navire portugais en 1550, la dynastie des Tokugawa est
fondée en 1600.

\item Le pays se cloisonne et se ferme à toute influence
extérieure. Cela ouvre la porte à 250 ans de paix, permettant aux anciennes
casernes militaires de se transformer en écoles d'arts martiaux. Mais\dots

\item 1860 : des navires occidentaux, notamment ceux du Commodore Perry
(\textsc{USA}) forcent, par leur supériorité technologique, le Japon à sortir
de son isolement. Le shogun accepte les traités inégaux, et le Japon
s'industrialise en sortant de sa période féodale. Cela provoque du ressentiment
de la part des anciens clans guerriers, qui voient là une occasion de renverser
le Shogun. C'est l'Empereur qui cristallise cette rébellion, et, après une
guerre civile \guil{contrôlée}, les anciens samourais meurent avec le shogun,
et l'Empereur est de nouveau installé sur le Trône.

\item Face aux puissances occidentales, le Japon a besoin de symboles
nationaux forts. Poussée de l'impérialisme, avec les guerres contre la Chine
(1895) et la Russie (1905) qui sont deux victoires éclatantes pour le Japon.
Parallèlement, renouveau des arts martiaux traditionnels grâce à Jigoro Kano
(fondateur du judo). Quelques années plus tard, l'aikido d'O Sensei naît.

\end{enumerate}
\input{tex/classic}
\input{tex/feodal}
\input{tex/war_provinces}
\input{tex/edo}
\chapter{Restauration Meiji}

\guil{Someday, when all your civilization and science are likewise swept away,
your kind will pray for a man with a sword.}

1853~\incise~le début de la fin pour les samurai. Préparez vos mouchoirs, c'est
le moment tragique. En même pas vingt ans, le pouvoir du shogun, et, par là,
les samurais, va s'éteindre, et l'Empereur va de nouveau être à la tête du
pays. C'est une période que je trouve proprement fascinante, j'espère que vous
m'excuserez si je ne résiste pas à la tentation de rentrer dans certains
détails. L'Histoire a retenu de cette transition qu'elle a été douce, sans
heurts et pacifique, louant les capacités d'adaptation des Japonais~\incise~le
nom même est \guil{Restauration Meiji}, pas \guil{Révolution} ou \guil{Guerre
civile}~\incise~et rien ne serait moins faux.

Un matin de juillet, le Commodore Perry, de la marine américaine, débarque avec
quatre cannonières à vapeur dans la baie d'Edo, la capitale. Il ouvre les
négociations visant à faire sortir le Japon de son isolation, limitant le
commerce avec l'extérieur (et, notamment, avec les puissances occidentales), en
faisant preuve de la puissance de son artillerie embarquée, qui terrorise les
Japonais. Autour du shogun, deux écoles s'opposent : \textcolor{blue}{ceux qui
refusent entièrement de traiter avec les étrangers}, au risque d'aller à la
guerre, et \textcolor{red}{ceux qui sont prêts à les accueillir}. Dans ce
camp-là d'ailleurs, il ne faut voir aucune forme d'amitié particulière. La
plupart, conscients du danger que pose la technologie étrangère supérieure sur
le pays, cherchent à compromettre afin d'apprendre pour mettre le Japon au même
niveau que le reste du monde. Des intellectuels synthétisent tout cela en
théorisant un Japon armé de \guil{connaissances de l'ouest} et de la
\guil{morale de l'est}, afin de \guil{contrôler les barbares avec leurs propres
méthodes}. Le \textcolor{red}{shogunat}, séduit par cette option, et contraint
par la \guil{politique de la canonnière} de Perry et de ses successeurs, signe
avec les États-Unis plusieurs traités de paix. Ces traités sont cependant dans
les faits des conventions très inégales, semblables d'ailleurs au traité de
Tianjin, signé entre la Grande-Bretagne et la Chine après la seconde guerre de
l'opium, pendant la même année. Le Japon signe bien vite d'autres traités
similaires avec les autres puissances (Angleterre, France, Russie,
Pays-Bas\dots). Ces traités ouvrent les grands ports japonais aux étrangers,
les laissent libres de commercer, avec des taxes à l'import et à l'export
extrêmement faibles.

L'ouverture du Japon à ce commerce extérieur incontrôlé entraîne une forte
instabilité économique. Parallèlement, des tremblements de terre et de
mauvaises récoltes à répétition font énormément augmenter le prix de la
nourriture, qui n'en avait pas besoin vu l'inflation effrayante qui avait
cours. Si le \textcolor{red}{shogunat} envoie de nombreuses missions,
notamment en Allemagne et en France (une délégation participe à l'Exposition
Universelle de 1867 à Paris) et tente de faire réviser les traités
inégalitaires, son autorité décroit rapidement. On assiste de nouveau à un
morcellement du Japon, les daimyos reprenant peu à peu leur indépendance.
Schématiquement, ceux qui étaient dans le camp des vainqueurs à la bataille de
Sekigahara en 1600, \textcolor{red}{les fudai-daimyo}, restent loyaux au
shogun ; les autres, \textcolor{blue}{les tozama-daimyo }qui avaient été
placés dans les fiefs les plus lointains de la capitale, s'opposent au
contraire résolument au shogunat. Les camps de \textcolor{blue}{Choshu}, au
sud-ouest de Honshu (l'île principale), \textcolor{blue}{Tosa} sur l'île de
Shikoku et \textcolor{blue}{Satsuma}, à l'extrémité sud de l'île de Kyushu,
sont les plus virulents dans leur opposition, et vont jouer un rôle
déterminant dans les événements qui vont suivre.

Comme d'hab', une crise de succession au shogun vient rajouter de l'huile sur
un feu crépitant. Si les \textcolor{red}{fudai daimyo} gagnent cette lutte de
pouvoir, une purge brutale et sanglante des \textcolor{blue}{opposants}
accentue le ressentiment contre les \textcolor{red}{Tokugawa}, qui en sortent
encore plus affaiblis. Le mouvement \emph{sonno joi}, \guil{Révérer l'Empereur,
expulser les barbares !}, devient un slogan de ralliement dans les provinces de
\textcolor{blue}{Choshu et de Satsuma}. Cette philosophie, outre d'être un
totale opposition avec la politique du \textcolor{red}{shogunat}, contient
également une idée nationaliste forte~\incise~le caractère pour \guil{barbare}
comportant une notion péjorative de \guil{race}. Elle rassemble autour d'elle
la plupart des castes guerrières, qui n'appréciaient pas de signer sous la
contrainte, et qui voyaient en outre leurs pouvoirs et leurs prérogatives
diminuer. La violence grandit contre les étrangers et ceux qui commercent avec
eux. L'ambassade britannique se fait attaquer, le \textcolor{red}{signataire}
qui avait signé le \guil{Traité d'amitié et de commerce} avec les américains se
fait assassiner en pleine rue, l'ambassade américaine est incendiée\dots\ C'est
également à cette occasion que les occidentaux découvrent le rituel du seppuku
: en 1868, onze marins d'une frégate française qui mouillait dans la baie
d'Osaka, sont tués par des guerriers de \textcolor{blue}{Tosa}. L'ambassadeur
de France au Japon proteste et exige un châtiment exemplaire : le daimyo de
\textcolor{blue}{Tosa}, défiant, présente dès le lendemain vingt samourais
jugés coupables, et condamnés à mort par seppuku. La magnanimité du capitaine
de vaisseau, assistant à l'exécution volontaire, permet à neuf d'entre eux
d'être graciés. D'autres témoignages insistent au contraire sur son effroi, les
samourais, dans une ultime bravade morbide, lui jetant leurs intestins à ses
pieds. On peut lire d'ailleurs ses mémoires décrivant la scène sur le net.

L'opposition armée à l'influence occidentale dégénère alors en réelle guerre
civile lorsque \textcolor{blue}{l'Empereur Komei}, rompant avec des siècles de
tradition~\incise~l'Empereur n'avait pas reigné effectivement sur le Japon
depuis 800 ans !~\incise~prend un rôle actif dans les affaires d'État. Il
reprend le slogan \guil{sonno joi} en proclamant en 1863 l'ordre (divin)
d'expulser les barbares, défiant par là ouvertement la politique du
\textcolor{red}{shogunat}. Le clan \textcolor{blue}{Choshu} suit cet ordre à la
lettre et fait tirer sans avertissement sur tous les navires étrangers qui
tentaient de traverser le détroit de Shimonoseki, entre l'île de Kyushu et
l'île d'Honshu. Petite anecdote : \textcolor{blue}{Choshu} tire avec d'antiques
canons (certains en bois), datant d'avant 1600, construits pendant l'ère
Sengoku. En quelques semaines, des navires américains, français, hollandais et
britanniques se font tirer dessus, avec une artillerie antédiluvienne. Quelques
jours plus tard, l'ordre est donné d'aller châtier ces insulaires têtus et
sous-développés : c'est la bataille de Shimonoseki, un combat peu
connu~\incise~et plutôt inégal~\incise~dans lequel des frégates d'un peu toutes
les nations vont bombarder les forteresses japonaises. La portée des canons
occidentaux dépasse de loin celle des pétoires japonaises vieilles de trois
siècles. Les français débarquent d'ailleurs quelques hommes pour sécuriser le
détroit, et capturent les pièces d'artillerie japonaise. Vous pouvez voir ces
canons exposés devant la porte nord des Invalides, devant le Musée de la
Guerre, et vous amuser à discerner le blason de Choshu gravé à côté de leurs
bouches. En passant, un diplomate britannique brillant, Sir Ernest Satow voyage
dans une des frégates anglaises. Il deviendra par la suite et jusqu'en 1900
l'ambassadeur de Londres au Japon, et, par son journal (\emph{A Diplomat in
Japan}), donne un témoignage de première main passionnant sur cette période
trouble de transition.

Le Japon est divisé en deux groupes : l'un, \textcolor{red}{pro-shogun},
centré sur la capitale Edo (l'actuelle Tokyo) et l'autre, \textcolor{blue}{anti-shogun},
donc \textcolor{blue}{pro-empereur}, unissant les clans de
l'ouest et du sud du Japon. Les deux partis prennent des mesures fortes pour
accroître leurs influences respectives. Le \textcolor{red}{bakufu} ainsi
envoie des marins étudier dans les écoles navales occidentales, et s'allie
avec la France (qui, l'histoire le montrera, aura parié sur le mauvais cheval)
pour construire notamment des arsenaux et moderniser son armée. Tiens, autre
anecdote\dots\ Vous souvenez-vous de cette purge du \emph{Dernier Samourai}
avec Tomtom Cruise ? La vérité n'est pas totalement loin de ce qui est montré
dans le flim, et le personnage joué par le beau gosse aux yeux bleus a
réellement existé~\incise~sauf qu'il était français. Il s'appelait
\textcolor{red}{Jules Brunet}, un officier artilleur de l'armée de Napoléon
III en charge de moderniser l'armée du shogun, dont nous parlerons un peu plus
bas. La \textcolor{blue}{coalition impériale} de son côté n'était pas en reste
: poussée par des visionnaires comme \textcolor{blue}{Sakamoto Ryoma} ou
\textcolor{blue}{Saigo Takamori}, des experts américains et britanniques
aident à former une nouvelle armée, issue de toutes les strates de la
population, et achètent des mitrailleuses, des obusiers et des navires de
guerre, notamment au marchant écossais Thomas Glover. C'est cet écossais qui a
aidé à fonder ce qui deviendra la compagnie Mitsubishi et la Brasserie
Japonaise~\incise~laquelle arborera en hommage à Glover une moustache sur ses
bouteilles de Kirin. C'est lui aussi qui, à la suite d'une affaire avec une
courtisane, a inspiré Puccini pour écrire \emph{Madame Butterfly}.

Un mot sur \textcolor{blue}{Sakamoto Ryoma}. Né dans le fief de Tosa, épéiste
confirmé, il s'engage vite dans le patriotisme anti-étrangers et choisit
d'assassiner \textcolor{red}{Katsu Kaishu} (dans quelle couleur l'écrire ?),
un officiel haut-gradé du shogunat, et fervent partisan de l'occidentalisation
et de la modernisation du Japon. Il entre dans sa maison au crépuscule armé
simplement d'un sabre court, et le surprend en train de lire un traité de
droit maritime néerlandais. \guil{Je ne suis pas armé, et j'aimerai finir ce
livre avant de mourir. Quelle différence cela fait-il pour toi, que je meure
maintenant ou dans quelques heures ? Je te jure, sur mes ancêtres, de ne pas
fuir quand le soleil se lèvera}, lui lance le conseiller. Si la légende dit
vrai, \textcolor{red}{Katsu Kaishu} réussit à persuader
\textcolor{blue}{Ryoma} de la futilité qu'il avait à combattre les pouvoirs
occidentaux au vu de l'état du Japon de l'époque, et de la nécessité pour le
pays d'avoir un projet à long terme. À l'aube, \textcolor{blue}{Ryoma} devient
l'assistant et le protégé de \textcolor{red}{Kai}\textcolor{blue}{shu} (qui
deviendra d'ailleurs, en 1873, le ministre de la Marine Impériale).
\textcolor{blue}{Ryoma} meurt hélas avant d'avoir vu l'impact de ses actions
et de ses convitions, à 31 ans, dans une auberge à Kyoto, assassiné par une
milice shogunale bien connue par les amateurs de flims de sabre et de mangas,
le \textcolor{red}{Shinsengumi}.

En effet, devant l'agitation dans sa capitale, créée notamment par d'anciens
samourais, le \textcolor{red}{shogunat} fonde de nombreuses milices
indépendantes, chargées de maintenir l'ordre. Ces milices, extrêmement
organisées et régies sur un code d'honneur strict, sont constituées de
samourais très compétents, l'entrée se faisant sur une évaluation du niveau de
\emph{kenjutsu} (de sabre). Dans les faits, ces épéistes d'élite se comportent
de la même manière qu'une mafia : assassinats, extorsions, chantages,
intimidations, actes terroristes\dots\ étaient fréquents. C'est cependant là
une autre preuve de la différence fondamentale qui existe entre les morales
japonaise et occidentale : la plus connue de ces milices, le
\textcolor{red}{Shinsengumi}, arborait en effet sur les vestes de chacun de
ses membres l'idéogramme \emph{makoto}, la \guil{sincérité} dont nous parlions
tantôt, écrit en traits d'or sur fond rouge. Une autre de ses milices, les
\emph{\textcolor{red}{hitokiri}} (mot-à-mot, \guil{coupeurs de gens}),
contenait quatre escrimeurs d'élite dont un \textcolor{red}{Tanaka Shinbei}
dont je raconte une anecdote captivante tout à la fin.

C'est \textcolor{blue}{Saigo Takamori}, un colosse taciturne et colérique, du
domaine de \textcolor{blue}{Satsuma}, qui prend la tête des troupes impériales
nouvellement formées. Menacé par une action militaire imminente, le shogun
abdique en 1867 et rend le pouvoir à \textcolor{blue}{l'Empereur Komei}. Dans
la foulée, l'Empereur Komei meurt (probablement assassiné d'ailleurs par des
extrémistes de \textcolor{blue}{Choshu}, craignant une alliance entre
l'Empereur et le shogun) et est remplacé par son jeune (15 ans) fils qui sera
connu sous le nom \textcolor{blue}{d'Empereur Meiji}. L'Empereur Meiji se
suffit de ce transfert d'autorité, mais \textcolor{blue}{Saigo} tempête et
exige que que le \textcolor{red}{shogun} soit privé de son armée, de ses terres
et de son titre, et devienne un simple daimyo. Le shogun refuse, et le Japon
plonge dans une guerre civile, la guerre du Boshin.

En janvier 1868, les forces du \textcolor{red}{shogunat} attaquent les forces
de \textcolor{blue}{Choshu et de Satsuma}, menées par \textcolor{blue}{Saigo},
à l'entrée de Kyoto. Même si l'armée shogunale compte 15\,000 hommes, certains
formés par des experts français, la majorité d'entre elle reste d'anciens
samurai. Les forces impériales, comptant moins de 5\,000 hommes, posséde
cependant des fusils, des mitrailleuses et une artillerie technologiquement
supérieurs, et une armée globalement mieux disciplinée~\incise~et moins
encline aux charges héroïques et honorables, quoique tactiquement peu
efficaces\dots\ C'est \textcolor{red}{Kastu Kaishu} (le \guil{Bismarck
japonais}) qui négocie avec \textcolor{blue}{Saigo} la capitulation d'Edo, en
mai. Désorganisée, démotivée et sans ressources, l'armée shogunale essuie les
défaites et fuit toujours plus au nord. Quelques centaines de soldats fondent
la \textcolor{red}{République d'Ezo }à Hokkaido (l'île tout au nord du Japon).
Parmi ces hommes, l'ancien commandant-en-chef du \textcolor{red}{Shinsengumi},
et cinq capitaines français, dont \textcolor{red}{Jules Brunet} dont nous
parlions plus haut. Anecdote amusante, au début du conflit, les Français
envoyèrent à Napoléon III une lettre. La France, comme les autres puissances
occidentales, s'étant déclarée comme neutre, ils devaient quitter l'armée
française pour y participer. Cette lettre de démission indique explicitement
qu'ils estimaient que les qualités d'honneur, de valeur et de dignité, si
importantes pour un militaires, étaient plus prononcées ici au Japon que dans
l'armée française. À un contre trois, les dernières forces moribondes rebelles
se font décimer, en dépit de la forteresse en étoile, construite sur des plans
de Vauban. Certains conseillers français restent au Japon, Brunet rentre et,
en dépit de la requète japonaise d'être jugé comme traître, le soutien
populaire dont il jouit force l'armée française de le cacher pendant quelques
années, le temps que l'affaire se tasse.

La victoire sur tout le territoire acquise, le nouveau gouvernement, avec
\textcolor{blue}{l'Empereur Meiji} à sa tête (quoiqu'à 15 ans, la tête était
plus symbolique qu'autre chose) s'efforce de moderniser le pays.
L'indépendance des domaines est petit à petit supprimée, et la classe des
samourai est abolie, par des édits successifs : interdiction de porter le
chignon, les sabres, suppression de leurs privilèges, autorisation des classes
populaires d'avoir un nom de famille\dots\ \textcolor{blue}{Saigo Takamori},
qui avait mené les troupes impériales pendant la guerre du Boshin, rejoint un
poste-clef dans le gouvernement, et demande la clémence pour les anciens
partisans au shogun. Le gouvernement impérial fait petit à petit glisser sa
politique initiale d'\guil{expulser les barbares !} en cherchant de plus en
plus à construire une nation forte, similaire aux puissances occidentales.
Mais la notion de \guil{nation forte} divise le gouvernement naissant :
certains considèrent qu'il faut moderniser le pays, avec des projets publics
de grande ampleur (télégraphe, chemin de fer\dots), d'autres, comme
\textcolor{blue}{Saigo}, insistent qu'il s'agit en premier lieu de posséder
une armée puissante, afin que le pays soit reconnu par les autres puissances
mondiales. Ce dernier propose d'ailleurs d'envahir la Corée, suite à son refus
de reconnaître \textcolor{blue}{l'Empereur Meiji} comme chef d'État. nt se
propose même d'aller en Corée en temps qu'ambassadeur, et de se comporter de
manière si outrancière que les Coréens n'aient d'autre choix que de le tuer,
provocant un \emph{casus belli} donnant au Japon une raison de lui déclarer la
guerre. Une telle entreprise aurait été cependant désastreuse, notamment sur
les finances balbutiantes du pays, et est refusée. \textcolor{green}{Saigo}
démissionne de toutes ses fonctions et rentre dans sa ville natale, Kagoshima.

Il fonde une école militaire privée, afin surtout de canaliser les énergies
des anciens samourais qui l'aidèrent à combattre le shogunat. Craignant une
rebellion de ces \textcolor{green}{forces nostalgiques}, le
\textcolor{blue}{gouvernement impérial} envoie un navire de guerre à Kagoshima
pour prendre toutes les armes et munitions des arsenaux. Ironiquement, cela
conduit les tensions déjà importantes (en cette même année de 1877, le
\textcolor{blue}{gouvernement} supprime les rentes des samourais) à éclater en
conflit armé. \textcolor{green}{Un millier d'étudiants} armés tout autant
d'armes à feu moderne que d'armes traditionnelles (sabres, lances, arcs)
attaquent les ports et arsenaux. Devant ce fait accompli,
\textcolor{green}{Saigo}, écartelé entre la manifestaion vivante de ses
morales qui disparaissent et son désir pacifique loyal au Japon, est poussé à
mener la rebellion. À 60 contre 1 environ, les mathématiques ne sont pas du
côté des insurgés. Quelques mois plus tard, \textcolor{green}{Saigo} lance une
charge héroïque et stupide, à la tête d'une \textcolor{blue}{armée bigarrée},
entre fusiliers sans munition et sabreurs anachroniques, contre une compagnie
de \textcolor{green}{mitrailleuses modernes}. Blessé à la cuisse, il se retire
du champ de bataille pour s'ouvrir le ventre comme l'exige son rang : ainsi
meurt le dernier samourai. 

Bien que le grand Saigo restera dans l'inconscient collectif comme étant un
grand héros, le gouvernement a maintenant toute liberté de révolutionner le
pays, et de le modeler comme une puissance occidentale. Il adopte une
constitution, inspirée de la constitution de l'Empire Germanique, et fait des
réformes de fond afin de moderniser le pays dans tous les domaines : sciences
modernes, langues, technologies, constructions, agriculture, éducation,
finance\dots\ et sports. Aux arts martiaux, associés à la tradition
rétrograde, sont préférés les sports occidentaux, comme le base-ball (encore
très en vogue au Japon aujourd'hui) et le rugby, par exemple. Mais alors, que
diable s'est-il donc passé pour que nous étudions l'aikido chaque semaine ?
Nous verrons cela dans notre prochain et dernier (ouf !) chapitre !

\chapter{Ère impériale}

\guil{He shrugged his shoulders. \guiluk{I have known many gods. He who denies
them is as blind as he who trusts them too deeply. I seek not beyond death. It
may be the blackness averred by the Nemedian skeptics, or Crom's realm of ice
and cloud, or the snowy plains and vaulted halls of the Nordheimer's Valhalla.
I know not, nor do I care. Let me live deep while I live; let me know the rich
juices of red meat and stinging wine on my palate, the hot embrace of white
arms, the mad exultation of battle when the blue blades flame and crimson, and
I am content. Let teachers and priests and philosophers brood over questions of
reality and illusion. I know this: if life is illusion, then I am no less an
illusion, and being thus, the illusion is real to me. I live, I burn with life,
I love, I slay, and am content.}}

Le XIX\up{e} siècle se finit : le Japon sort tout juste d'une période de
changements profonds, qui ressemble cependant plus à un dernier baroud
d'honneur de dinosaures conservateurs démodés qu'à une révolution sanglante. La
modernisation du pays, encore une féodalité médiévale en 1850, est fulgurante :
grands chantiers publics, signature d'une Constitution nationale, création d'un
parlement, refonte des institutions politiques, émergence d'une presse
importante, création d'une monnaie fiduciaire unique et d'une économie de
marché, montée en puissance de grands groupes industriels\dots\ Ces derniers
ont à leur tête d'anciens chefs de clans, et deviendront plus tard les grandes
firmes \emph{zaibatsu}, comme Mitsui ou Mitsubishi. En trente ans, l'Empire du
Japon possède une force que les Empires occidentaux ont acquis en plus d'un
siècle.

Ce formidable développement du Japon se mue peu à peu en complexe de
supériorité, puis franchement en impérialisme. Devant l'humiliation des nations
voisines, en particulier celle de la Chine~\incise~cf. les guerres de
l'opium~\incise~des intellectuels du Japon proposent le concept de \emph{zone
d'influence}, dans laquelle le Japon sortirait de ses frontières pour se garder
une zone \guil{tampon} dans le cas d'une attaque étrangère. Une série de
tensions entre la Corée et la Mandchourie culmine en 1894, quand le Japon en
profite pour intervenir. Sous prétexte de vouloir aider la Corée, l'Empire du
Soleil Levant envoie son armée pousser ses frontières contre celle de l'Empire
du Milieu. En un an, la flotte chinoise est détruite, et une série de défaite
militaire contraint la Chine à signer un traité de paix, concédant au Japon
plusieurs territoires (Corée, Taïwan, Port-Arthur en Mandchourie\dots). Ce sont
les tensions impérialistes autour de ce port qui conduisent le Japon à déclarer
la guerre à la Russie tsariste, dix ans plus tard. Là encore, le conflit ne
dure pas longtemps, et la Russie, pas aidée par la révolution rouge d'octobre
de la même année, signe un traité de paix au bout d'un an, cédant l'île
Sakhaline au Japon. En quelques anénes, l'Empire du Japon gagne ses galons de
grande puissance : la \guil{race jaune} a en effet battu, dans un combat
régulier, un gouvernement de la \guil{race blanche}, pourtant fondée à
gouverner le monde. Cette surprise viendra conforter la dangereuse évolution du
climat impérialiste au Japon, légitimant ses ambitions coloniales en se
qualifiant de \guil{pays des dieux}. Le Japon s'estime alors fondé à étendre
son influence colonialiste sur l'ensemble du continent asiatique. La spirale
mortelle est amorcée, et la suite n'est que trop bien connue : du massacre de
Nankin aux bombardements nucléaires, la course à l'abîme va se poursuivre avec
une ponctualité métronomique.

Mais nous n'avons toujours par parlé de l'aikido, pour l'instant\dots\
Revenons un peu en arrière : 1880, le Japon est en pleine modernisation. Les
connaissances occidentales sont considérées comme préférables aux antiques
voies. Aux arts martiaux sont préférés, notamment à haut niveau, les sports
\guil{civilisés} comme le rugby, le football ou le baseball. Parallèlement, un
fort sentiment patriotique nait au Japon. Le nouvel Empire a besoin de
symboles unificateurs forts, et va donc les chercher dans son passé. Ainsi, la
figure du héros loyal jusqu'à la mort est personnifié par Kusunoki Masashige,
le général du XIV\up{e} siècle dont nous avions parlé tantôt. Le
\emph{kamikaze}, le typhon providentiel qui a(urait) mis en déroute les
envahisseurs Mongols en 1280 est vu comme une preuve du caractère sacré et
inviolable du Japon. Nitobe Inazo, un universitaire formé par les Jésuites,
théorise un code moral dans \emph{Bushido, l'Âme du Japon}. Ce livre, à peu
d'être un outil de propagande impérialiste, décrit les qualités qu'auraient
possédées les guerriers d'antan, et expose comment les appliquer aujourd'hui.
La loyauté envers l'Empereur, issue d'une vision confucéenne du monde, et le
sacrifice de soi sont particulièrement mis en avant~\incise~ce qui permettra
aux militaires ensuite de justifier un paquet de morts\dots

C'est dans cet état d'esprit bicéphale, modernisation \emph{vs.} traditions,
toujours terriblement présent de nos jours, qu'un petit enseignant de jiujitsu
arrive sur le devant de la scène. Je crois que Jean-Louis a lu extensivement
sur le sujet, et je le laisserai donc me corriger et compléter. Kano Jigoro,
1m56, 40 kilos tout mouillé, est un haut-fonctionnaire du gouvernement, chargé
de superviser les sports et leur éducation dans l'Empire. Il a étudié
longuement avec différents maîtres de différentes écoles, et possède un style,
qu'il a nommé \emph{judo} (la discipline de la souplesse). En dépit de
l'aspect résolument japonais de son art, ses méthodes d'enseignement, issues
des États- Unis, et sa philosophie sous-jacente, lui amènent un soutien de
l'État. Plutôt que les sempiternels exercices de gymnastique effectués en
masse dans les cours de récré, le judo permet un développement physique
\emph{et} mental, une forme d'enseignement moral étant en effet au coeur du
judo. Kano accepte cependant difficilement la récupération de son art par les
militaires, qui voient là une manière d'avoir aisément des recrues loyales et
en bonne santé, et se place en opposant majeur au fascisme japonais. En tous
les cas, le point est fait : les arts martiaux ancestraux peuvent évoluer. Ce
qu'on appelait les \emph{koryu} (\guil{anciennes écoles}) deviennent des
\emph{gendai budo} (\guil{arts martiaux modernes}). Notons qu'un art martial a
conservé tous ses aspects traditionnels, et a (relativement) peu changé depuis
deux mille ans, le sumo.

Là-dessus, un jeune homme maigrichon et pas bien grand étudie dès 1900 de
multiples écoles de jujitsu. Cinq ans plus tard, il décide de s'engager dans
l'infanterie, mais sa petite taille le lui interdit. À 1m56, il ne lui manque
que quelques centimètres. Qu'à cela ne tienne, il passe des heures suspendu à
un arbre avec des poids au pied, et participe à la guerre russo-japonaise en
Mandchourie. 1912 est une révélation : parti fonder un village à Hokkaido, il
rencontre le grand maître de l'école Daito de jujutsu~\incise~l'ancienne école
du clan Takeda, les cavaliers féroces du Kai. Il rencontre ensuite un des
fondateurs d'une secte shinto, qui donnera à son art une dimension spirituelle
forte. Dans la brèche ouverte par Kano Jigoro, il construit petit à petit une
discipline qui adapte les techniques guerrières ancestrales en réussissant à
les retourner, pour que les volontés de chacun, fussent-elles de nuire,
s'unissent plutôt que s'opposent. Cet homme, c'est O Ueshiba Morihei Sensei
(\guil{le grand professeur Ueshiba Morihei}), le vieillard à la peau
parcheminée et aux regard doux, perçant et impassible, que nous saluons à
chaque fois que nous rentrons dans le dojo.

\chapter{Une anecdote pour finir}

Désolé, je sais que je n'ai pas tenu mon engagement d'être court et
synthétique, mais je ne résiste décidément pas à vous proposer une histoire et
énigme, à propos de \textcolor{red}{Tanaka Shinbei} :

Une nuit sans lune, le 20 mai 1863, \textcolor{blue}{Anenokoji Kintomo}, un
chef radical à la cour de \textcolor{blue}{l'Empereur}, et un champion des
forces contre le \textcolor{red}{Shogun}~\incise~alors l'actuel centre de
pouvoir~\incise~rentra chez lui, marchant du \textcolor{blue}{palais impérial
de Kyoto} à sa résidence. La route était sombre, et ses pas étaient simplement
éclairés par une lanterne de papier portée par un de ces cinq pages.
Soudainement, un attaquant vêtu de noir, alors caché dans des fourrés, sauta
devant lui, dégaina et lui coupa le visage de haut en bas, avant de
disparaître aussi vite. Quatre des cinq domestiques du noble s'enfuirent, et
le cinquième courut à la recherche de l'assassin alors que le sang coulait du
front de son maître. Un autre assaillant arriva par derrière, et planta son
sabre profondément dans le dos \textcolor{blue}{d'Anenokoji}. Un troisième
agresseur le coupa au niveau du torse. Incapable de crier, le jeune noble
s'écroula à terre. Aidé par son page qui revint, \textcolor{blue}{Anenokoji}
boita une centaine de mètres avant de s'effondrer totalement. Ce leader du
mouvement d'expulsions des barbares étrangers et de renverser le Shogunat
était mort, à vingt-cinq ans.

Les discussions pour savoir qui était le meurtrier allaient bon train. Les
rues de Kyoto n'étaient pas sûres depuis quelques années, et franchement
sanglantes depuis quelques mois. L'anarchie, qui dans l'Histoire a toujours
précédé aux changements de régime, grondait farouchement, et en particulier
ici, comme nous venons de le voir. \textcolor{blue}{Anenokoji} revenait
d'ailleurs d'une discussion animée avec d'autres officiels de la
\textcolor{blue}{cour impériale}, dont le consensus était d'émettre un décret
scellé par l'Empereur obligeant le clan de \textcolor{blue}{Choshu} à
renverser \textcolor{red}{le pouvoir en place}. \textcolor{blue}{Choshu}, ce
sont les mecs de la bataille de Shimonoseki, si vous vous souvenez bien.

Tout le monde pointait le doigt vers les autres : Satsuma (le fief du grand
Saigo), un autre clan, voisin de Choshu et son grand rival politique, était
moins extrême que les gros tarés de Choshu : il préférait une alliance entre
la cour (\textcolor{blue}{impériale}) et le pouvoir en place (le
\textcolor{red}{shogun}), sous la forme d'un mariage entre la soeur de
l'Empereur, l'Impératrice Kazu, et \textcolor{red}{l'actuel shogun, Iemochi} ;
le daimyo de Choshu, au contraire, était partisan d'une guerre ouverte contre
le shogun d'une part et les étrangers de l'autre (d'où la bataille de
Shimonoseki). Satsuma accusait donc Choshu d'avoir buté un type qui risquait
d'argumenter en faveur d'un gouvernement hybride, de cohabitation. Choshu,
pour sa part, suspectait Satsuma d'avoir zigouillé un mec qui jouissait à la
cour de l'Empereur d'un réel pouvoir, et qui risquait de pencher au contraire
vers une guerre totale. D'autant plus que les Choshu ignoraient les intentions
profondes de Satsuma : si la rumeur disait vrai, le daimyo de Satsuma, Shimazu
Hisamitsu, allait mener un contingent de ses troupes dans Kyoto, la capitale
impériale, afin de tordre le bras à l'Empereur pour le forcer à déclarer la
guerre contre le shogunat. Ce dont rêvait Choshu, mais ça les faisait chier
que ce soit quelqu'un d'autre qu'eux qui mène l'armée.  Et, finalement, tout
le monde soupçonnait le shogunat d'avoir éliminé un dangereux ennemi.  Bon, et
y'a plein d'autres trucs (une vingtaine de jours avant sa mort,
\textcolor{blue}{Anenokoji} avait passé du temps à bord d'un cuirassé du
shogun, en compagnie de \textcolor{red}{Katsu Kaishu}, le brillant futur
ministre de la marine, et le plus fervent défenseur de l'ouverture du pays).
Mais bon, l'un dans l'autre\dots\ L'important : c'était le bordel.

Là où l'affaire se corse, c'est quand on apprend qu'un wakizashi, le sabre
court, avait été retrouvé sur les lieux du crime. Sur le fourreau du sabre
était peint, comme c'est la coutume, le blason du clan dont appartenait le
samurai qui possédait le sabre\dots\ Une croix dans un rond~\incise~Satsuma !
Les attaques du domaine de Choshu se firent plus insistantes, tandis que
Satsuma accusait Choshu d'avoir volé le sabre d'un de leurs guerriers, pour le
poser là comme preuve. Ceci était d'ailleurs assez crédible, dans la mesure où
c'était Choshu qui était actuellement chargé de garder le Palais Impérial, et
donc, subséquemment, gérait des troupes provenant de plusieurs contrées.

Le possesseur du sabre a d'ailleurs été identifié, via les gravures dans le
manche : un certain \textcolor{red}{Tanaka Shinbei}, samurai de Satsuma, et
surtout, assassin notoire. Assassin, oui, membre des Quatre
\guil{\textcolor{red}{Hitokiri}}, les trancheurs de gens, et sa brutalité
était bien connue~\incise~et il avait démontré sa maîtrise du sabre en de
nombreuses occasions~\incise~principalement des contrats pour des
\textcolor{red}{partisans du shogun}. Ok, sauf que :
\textcolor{blue}{Anenokoji} a été attaqué par trois hommes~\incise~ce qui
n'est pas le mode opératoire de \textcolor{red}{Tanaka}~\incise~et surtout,
pas un seul n'a été capable de lui délivrer un seul coup fatal ! D'ailleurs,
un assassin du calibre de Tanaka n'aurait jamais laissé son sabre
derrière~\incise~un sabre encore dans son fourreau, pas taché de sang.
Ajoutons à ça que tous les amis et camarades de Tanaka ont tous juré l'avoir
vu dans un maison de geisha à une centaine de kilomètres de Kyoto cette nuit-
là~\incise~une promesse de samurai, quelque chose à ne pas prendre à la
légère.

Et pourtant, la preuve circonstancielle demeurait : le sabre de
\textcolor{red}{Tanaka Shinbei} avait été trouvé là, et il fut arrêté quelques
jours plus tard.

Nous avons l'avantage de l'Histoire avec nous : dans son cachot, attendant son
procès imminent, Tanaka a écrit une lettre. Elle enjoignait un de ces proches
de ne pas informer les autorités du vol de son waikikizashi, survenu dans un
bordel de Kyoto une semaine avant le meurtre. Il lui demandait par écrit de ne
pas soulever la preuve irréfutable, aujourd'hui retrouvée d'ailleurs, de ce
vol, puisqu'il en avait rapporté à son supérieur (afin de récupérer un autre
sabre, partie absolue de l'uniforme de la caste des samurai). Le procès-verbal
de son procès nous apprend qu'il s'est déroulé ainsi : Tanaka Shinbei s'est
présenté, en blanc et non menotté (c'est la coutume : on ne lie pas un
guerrier, ce serait mettre sa parole en doute, et donc, l'insulter gravement)
devant ses interrogateurs. La première question fut : \guil{est-ce votre sabre
?}, demanda un des enquêteurs, le sabre à la main. \guil{Je ne sais pas, je ne
peux pas le voir convenablement d'ici. Je dois le voir de plus près.} répondit
Tanaka. Alors que l'on lui tendit son sabre, Tanaka mit lentement la lame à nu
pour l'examiner. \guil{Oui, c'est bien mon sabre}, déclara-t-il.
\guil{Cependant}, ajouta-t-il avec force, \guil{je n'ai pas tué Anenokoji.} --
\guil{Si vous ne l'avez pas tué, qui\dots} Avant que l'interrogateur ne puisse
finir sa question, et à la surprise des autorités restées interdites, Tanaka,
d'un geste décidé, retourna son sabre vers lui, plongea la lame en son flanc
gauche, et, sa main gauche rejoignant sa main droite, poussa le tranchant de
l'autre côté de son ventre. Puis, dans la même rafale de volonté fanatique,
ramena la pointe de sa lame sur son cou, et s'ouvrit la carotide, tombant vers
l'avant sans le moindre cri.

\emph{On a side note}, les enquêteurs conclurent à la culpabilité de Tanaka Shinbei.
Satsuma tomba en disgrâce (temporairement), et ce fut Choshu et ses radicaux
qui régnèrent (un temps) en maître sur l'ancienne capitale de l'Empereur,
devenue lieu de toutes les agitations les plus frénétiques. Alors,
question\dots\ Pourquoi, ce geste absurde et déraisonné de Shinbei, alors qu'il
pouvait prouver son innocence ? Quel gâchis ! Et si ça peut vous aider, sachez
que Mishima Yukio, l'écrivain génial, avait fait des pieds et des mains pour
pouvoir jouer ce rôle-là dans le \emph{Hitokiri} de Hideo Gosha.


\end{document}
