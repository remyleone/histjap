\documentclass[11pt,dvipsnames,a4paper,svgnames]{book} % Classe du document
%\usepackage[utf8]{inputenc} % Codage du texte dans le source (là on fait genre on est sous X. Genre... Je sais.)
\usepackage[francais]{babel} % Typographie française
\usepackage{xspace} %sert à babel pour mieux gérer après "", toussa.
%\usepackage[T1]{fontenc} % Codage de la fonte. Permet d'avoir une césure française complète (même avec des lettres accentuées).

\usepackage{fontspec}
\usepackage{xunicode}

% Pour avoir de la couleur
\usepackage[usenames,svgnames,table]{xcolor}
% \definecolor{newgreen}{rgb}{0,0.5,0}
% \definecolor{newred}{rgb}{0.6,0,0}
% Gestion des images
\usepackage{graphicx} %\DeclareGraphicsExtensions{.jpg,.png,.gif}

\usepackage{alltt}

% En-tête et pied de page joli
\usepackage{fancyhdr}
\pagestyle{fancy}

% Normal :
\fancyhf{}
\fancyfoot[LE,RO]{\bfseries\thepage}
\renewcommand{\headrulewidth}{0pt} %tailles des traits entre les en-tête et le corps du texte
\renewcommand{\footrulewidth}{0.4pt}

% TYPOGRAPHY
\usepackage{fontspec}
\defaultfontfeatures{Mapping=tex-text} % converts LaTeX specials (``quotes'' --- dashes etc.) to unicode
\setromanfont [BoldFont={Gentium Basic Bold}, ItalicFont={Gentium Basic Italic}]{Gentium Basic}

% Pour les débuts de chapitres(après \chapter)
\fancypagestyle{plain}{
\fancyhf{} % clear all header and footer fields
\fancyfoot[LE,RO]{\bfseries\thepage}
\fancyfoot[LO,RE]{}
\renewcommand{\headrulewidth}{0pt}
\renewcommand{\footrulewidth}{0.4pt}}


% Pour la table des matières
\fancypagestyle{toc}{%
\fancyhf{} % clear all header and footer fields
%\rfoot{\thepage}
\renewcommand{\headrulewidth}{0pt}
\renewcommand{\footrulewidth}{0pt}
}

% Nouvelles commandes

\newcommand{\todo}[1]{\Large TODO : #1 \normalsize}
\newcommand{\guil}[1]{\guillemotleft~#1~\guillemotright}
\newcommand{\guiluk}[1]{\textquotedblleft\,#1\,\textquotedblright}
\newcommand{\titreun}[1]{\large \textbf{#1} : \normalsize}
\newcommand{\sstitreun}[2]{\large \textbf{#1} (\emph{#2}) \textbf{:} \normalsize}
\newcommand{\titredeux}[1]{\noindent \large \textit{#1} \normalsize}
\newcommand{\incise}{\textendash}
\newcommand{\doubleparskip}{\setlength{\parskip}{2.5ex plus 0.7ex minus 0.4ex}}

\newcounter{numeroapho}
\setcounter{numeroapho}{1} %Hop, on commence 
%\newcommand{\titretrois}[1]{\stepcounter{numeroapho} \large \textbf{\arabic{numeroapho}}~\textendash~\textsc{#1} \normalsize}
\newcommand{\titretrois}[1]{\stepcounter{numeroapho} \ \linebreak \linebreak \noindent \large \arabic{numeroapho}~\textemdash~\textsc{#1} \normalsize}
\newcommand{\stitretrois}[1]{\stepcounter{numeroapho} \noindent \large \arabic{numeroapho}~\textemdash~\textsc{#1} \normalsize}
\newcommand{\sautparagraphetrois}{\\}

\title{Historiette du Japon}
\author{Manfred}
\date{\today}

\raggedbottom %Pour éviter les paragraphes qui s'étalent sur une page entière

\begin{document}

%\maketitle


% \begin{flushleft}
% \end{flushleft}
%\end{titlepage}

% \thispagestyle{empty}


% \newpage
% \null
% \thispagestyle{empty}
% %\begin{comment}
% \frontmatter
% \tableofcontents
% \thispagestyle{toc}
% \pagebreak
% \thispagestyle{empty}

\mainmatter
\pagenumbering{arabic}


\begin{center}
  \Large
   \ 
    \linebreak
    \linebreak
    \linebreak
    \linebreak
    \textbf{Historiette du Japon}
  \normalsize
  \flushright{Michael Z.~\incise~Dernière révision : 19/06/2013}

\end{center}

\doubleparskip

Salut à tous, chers amis en pyjama blanc et jupette pour certains !

Jérôme nous avait parlé il y a quelques semaines d'un livre d'histoire
japonaise, écrit par un de ses amis~\incise~lequel donne d'ailleurs une
conférence fin juin traitant du même thème que son second bouquin, les
invasions mongoles au Japon. Il en avait parlé sur le forum du club, et
l'avait ramené un soir pour le prêter, mais il n'a pas eu le succès escompté.
Comme tout sujet, l'histoire japonaise peut paraître difficile d'accès au
début~\incise~des noms pas franchement faciles à se rappeler, des dates à
foison, d'autres considérations, d'autres valeurs morales\dots~\incise~et je
me suis dit que ça pourrait être intéressant d'en proposer une introduction.
Et puis, ça me permettra de réviser un peu, ce qui n'est pas plus mal.

Mon but est donc d'écrire quelques paragraphes que j'espère assez simples à
lire, à tout le moins pas chiants, qui survoleront 1500 ans d'histoire
japonaise. Je vais essayer de restreindre les dates et les noms, et de laisser
tomber certains détails, quitte à raconter l'histoire plutôt que l'Histoire. À
chacun ensuite, si le virus a pris, d'approfondir telle ou telle période. Je
vais également essayer de montrer, par les (r)évolutions techniques et
sociales successives, comment les batailles rangées de samouraïs ont donné
naissance aux arts martiaux que nous pratiquons chaque semaine. Il y aura six
chapitres à notre voyage. Vais essayer de tout faire de tête, histoire de voir
ce que ça donne\dots

Pour que ce soit plus facile à lire, je l'ai mis en forme, histoire d'éviter
de se ruiner les yeux sur l'écran. Bon, et maintenant que c'est en \LaTeX{},
je pourrais même rajouter des macrons sur les \={o}\dots

\textbf{EDIT, écrit à la fin :}

Comme je remarque que j'ai un peu dépassé la limite d'un paragraphe par
chapitre que je m'étais fixée, je vous propose donc une historiettette du
Japon :

\begin{enumerate}

\item 500 : des clans d'origine paysanne se regroupent, et forment un pouvoir
centré sur la figure de l'Empereur. Petit à petit, les gens à la cour se
détachent des affaires locales, tenues par les guerriers, qui récupèrent alors
le pouvoir. 1180 : un clan en particulier se révèle victorieux, et dépouille
l'Empereur de toute influence, sinon symbolique. Le pouvoir est alors tenu par
un \guil{généralissime}, le Shogun.

\item 1280 : le petit-fils de Genghis Khan essaie de débarquer au Japon, mais
n'y arrive pas. 1333 : l'Empereur du moment essaie de re-récupérer le pouvoir,
mais n'y arrive pas. Cependant, l'influence du shogun diminue de plus en plus,
et en 1450, le Japon se morcelle en une multitude de cité-états indépendantes,
qui se font constamment la guerre.

\item Tout fout l'camp ma bonne dame. 1450 $\rightarrow$ 1600, guerre civile
au Japon. Petit à petit cependant, le pouvoir se concentre dans les mains
d'unificateurs successifs, le dernier restant étant du clan Tokugawa. Aidé
notamment par l'introduction révolutionnaire des arquebuses, \emph{via}
l'arrivée fortuite d'un navire portugais en 1550, la dynastie des Tokugawa est
fondée en 1600.

\item Le pays se cloisonne et se ferme à toute influence
extérieure. Cela ouvre la porte à 250 ans de paix, permettant aux anciennes
casernes militaires de se transformer en écoles d'arts martiaux. Mais\dots

\item 1860 : des navires occidentaux, notamment ceux du Commodore Perry
(\textsc{USA}) forcent, par leur supériorité technologique, le Japon à sortir
de son isolement. Le shogun accepte les traités inégaux, et le Japon
s'industrialise en sortant de sa période féodale. Cela provoque du ressentiment
de la part des anciens clans guerriers, qui voient là une occasion de renverser
le Shogun. C'est l'Empereur qui cristallise cette rébellion, et, après une
guerre civile \guil{contrôlée}, les anciens samourais meurent avec le shogun,
et l'Empereur est de nouveau installé sur le Trône.

\item Face aux puissances occidentales, le Japon a besoin de symboles
nationaux forts. Poussée de l'impérialisme, avec les guerres contre la Chine
(1895) et la Russie (1905) qui sont deux victoires éclatantes pour le Japon.
Parallèlement, renouveau des arts martiaux traditionnels grâce à Jigoro Kano
(fondateur du judo). Quelques années plus tard, l'aikido d'O Sensei naît.

\end{enumerate}
\chapter{Ère classique}

\guil{\guiluk{Barbarism is the natural state of mankind,} the borderer said,
still staring somberly at the Cimmerian. \guiluk{Civilization is unnatural. It
is a whim of circumstance. And barbarism must always ultimately triumph.}}

Notre voyage commence aux tous premiers siècles de notre ère, vers 450. En
Europe, Rome n'est plus depuis peu, les barbares (Goths, Vandales\dots)
déferlent un peu partout. À quelques milliers de kilomètres de là, l'écriture
par idéogrammes est arrivée au Japon \emph{via} la Chine. Une sorte de
paysannerie clanique émerge alors, exploitant les rares terres cultivables
(faut se rappeler que le Japon est majoritairement constitué de montagnes),
concentrées sur les deux plaines très fertiles du Kinai (autour d'Osaka et de
Kyoto) et du Kanto (autour de l'actuelle Tokyo)~\incise~voir
\textcolor{blue}{\underline{\texttt{CARTE DU JAPON}}} par exemple pour avoir
une idée de quoi que j'cause. Petit à petit, les clans se rassemblent, et une
sorte de gouvernement se forme, à l'image des usages chinois, avec une
bureaucratie civile centralisée. Son unité se cristallise autour de la figure
de l'Empereur, issu d'une lignée héréditaire qui continue encore de nos jours,
et qui prétend remonter aux temps mythologiques de la déesse du soleil
Amaterasu.

Là, on voit une logique qui va perdurer encore pendant un bon millénaire,
quoique sous différentes formes : ce proto-gouvernement demande effectivement à
certains chefs de clan d'aller guerroyer contre les peuples aborigènes (Aïnus,
Emishis), notamment vers le nord et l'est, en leur promettant les terres qu'ils
auront réussi à conquérir. Parallèlement, d'autres chefs de clans guerriers,
dépossédés par l'évolution administrative, sont attirés par la capitale, et
contribuent à son évolution sociale, religieuse, culturelle~\incise~mais
laissent petit à petit leur pouvoir entre les sabres des autres.

Mais nous n'en sommes pas là\dots\ Pour l'instant, vers 700, la capitale est à
Nara, avec un empereur fort, gouvernant de manière absolue une structure
organisée rationnellement, en divisant les terres de manière (plutôt) équitable
entre chaque sujet. L'aristocratie, rendue puissante par des impôts
efficacement collectés, se lance dans des grands travaux
publics~\incise~routes, irrigations, temples\dots\ La culture y est d'ailleurs
florissante, avec les deux chroniques majeures, relatant les origines
mythologiques du Japon (\emph{Kojiki} et \emph{Nihon Shoki}) écrites en ce
temps-là, et le \emph{Man'yoshu}, une collection de 5\,000
poèmes~\incise~principalement des tankas, les ancètres des haikus (les poèmes
courts en 5-7-5 pieds). Tiens, autre événement qui augure une logique
récurrente : la capitale se déplace de Nara à Kyoto, vers 800, pour échapper à
la grogne des prêtres bouddhiques. Au Japon en effet, au moins jusqu'en 1600,
les prêtres vont posséder un réel pouvoir politique, au même titre qu'un clan
guerrier. Le bouddhisme ne rime pas (encore) avec non-violence, et les ordres
religieux entretiennent des milices privées de taille conséquente, avec des
milliers de mercenaires et de moines-soldats, réputés pour leur habileté
notamment à la hallebarde (\emph{naginata}).

Les terres étaient donc divisées en parcelles entre les sujets de l'empereur.
La contrepartie était double : des impôts, en nature ou en corvées à payer, et
la possibilité d'être mobilisé sur ordre du pouvoir exécutif. En théorie, cette
idée du service militaire aurait pu marcher ; en pratique, les paysans mal
entraînés et indisciplinés cherchent par tous les moyens à se dérober devant la
conscription. Ainsi, les plus fortunés vont en payer d'autres pour se
soustraire aux trois années de service, et, petit à petit, c'est une armée
quasi-professionnelle qui se met en place. L'intérieur du Japon étant quasiment
entièrement pacifié, les craintes viennent d'une invasion extérieure. Et le
gouvernement perd petit à petit de l'influence au niveau local, ce qui augmente
l'instabilité sociale, notamment dans les provinces éloignées de la capitale.
Les fonctions de maintien de l'ordre, puis judiciaire se \guil{privatise}, au
profit d'archers montés~\incise~les futurs samouraïs.

Un point étymologique, tiens : \guil{samurai} viendrait du verbe
\guil{saburau}, qui signifie \guil{servir} ou \guil{attendre à côté de}, et
était utilisé, dans cette situation, pour désigner, par glissement, les
fonctionnaires qui servaient sur le terrain les intérêts des magistrats restés
à la cour.

Mais Kyoto est le fief de la famille Fujiwara, qui, par un jeu d'intrigues
politiques largement plus complexes que dans Game of Thrones, de mariages
arrangés et de régences dans l'ombre, réussit à placer ses protégés au sommet
de l'État, décidant de la politique du pays à la place de l'Empereur. À partir
de ce moment-là, vers 700, le pouvoir politique réel ne sera plus entre les
mains du Fils du Ciel, mais sera tenu par la noblesse de cour, puis par des
seigneurs de guerre (les shoguns), puis l'armée (entre 1880 et 1945) et, plus
récemment, par le Premier Ministre.

Fast-forward vers 1100 : le pouvoir impérial est entièrement dépendant des
familles martiales (les \emph{kuge}), hissées à la tête du gouvernement à la
suite de réussites et récompenses militaires locales ; Dame Shikibu Murasaki
écrit \emph{Le Dit du Genji}, un des plus vieux romans au monde ; le bouddhisme
continue de se développer dans les îles. La plus importante des familles
contrôlant le Trône reste les Fujiwara, mais un paqueton d'autres gravitent
autour, notamment les Taira et les Minamoto\dots\ qui, à la suite d'une flambée
d'actes de sédition, se foutent allégrement sur la tronche en 1180. C'est la
guerre du Genpei, nom formé des initiales des deux clans. En effet, il y a deux
manières de lire les idéogrammes japonais : la lecture japonaise,
\emph{kun'yomi}, et celle, chinoise traditionnelle, \emph{on'yomi}, dans
laquelle Minamoto se lit \guil{Genji} et Taira \guil{Heike}.

Chose intéressante : c'est pendant la première bataille de la guerre du Genpei
que l'on assiste pour la première fois au \emph{seppuku} (ou \guil{hara-kiri},
si on lit les mêmes idéogrammes en kun-yomi) d'un guerrier, le suicide rituel
tant ancré dans l'imaginaire qu'il n'a pas quitté l'histoire, des 47 Ronin du
XVIII\up{e} siècle à Mishima en 1970. La guerre, le domaine de nobles qui
cherchent à récupérer le pouvoir d'autres nobles, est une affaire de
professionnels : les combats ne sont pas des batailles rangées, mais des duels
codifiés d'archerie montée, ressemblant plus aux tournois médiévaux qu'aux
plages de Normandie. Les deux armées se retrouvent, de part et d'autre du
champ de bataille, et une flèche sifflante à tête creuse est lancée au début
de la confrontation, pour attirer l'attention des dieux (les \emph{kamis},
comme dans \guil{kamikaze}). Puis, un combattant s'avance, déclame son
pédigree et attend que quelqu'un de son rang accepte son invitation. S'ensuit
alors un ballet galopant où les guerriers en armure lourde vont tenter de
placer une flêche dans leur adversaire. Ces cavaliers sont accompagnés
d'écuyers et de fantassins, armés de lances.

Bon, ok, je n'ai pas été très honnête, là. Les historiens s'accordent de plus
en plus pour dire que cette description des conflits est celle,
hagiographique, des écrivains de l'époque, embelissant pas mal leurs
chroniques. Mais, à la limite, peu importe~\incise~c'est ainsi que les choses
sont restées en mémoire. En tous les cas, l'important reste que la guerre est
là éminément une affaire de prestige, entre gens d'un certain rang. On notera
également que l'arme de prédilection est l'archerie montée (la \guil{voie du
guerrier} s'appelait alors \emph{kyuba no michi}~\incise~\guil{l'arc et le
cheval}), et pas du tout le sabre. Celui-ci était long, très courbe, et
pendait à la ceinture tranchant vers le bas (comme un sabre de cavalerie
européen, si je ne m'abuse), et servait plus de symbole d'autorité que réel
outil belliqueux.

Au final, les Minamoto sortent vainqueurs de ce conflit, et les Taira sont
détruits. Là où ce conflit est historique (parce que des nobles qui se
bastonnent, il y en a toujours eu), c'est que les Minamoto, au sortir de la
guerre du Genpei, dépossèdent officiellement l'Empereur de tout pouvoir
politique, et le récupèrent entièrement. Minamoto no Yoritomo, le chef du
clan, se proclame \emph{seii tai-shogun}, littéralement \guil{le grand général
qui soumet les barbares de l'Est} (anciennement, c'était un titre donné
temporairement à un noble chargé de pacifier une région~\incise~comme les
\guil{tyrans} romains), raccourci en Shogun. Il déplace le centre du
gouvernement à Kamakura, à deux pas d'un village marécageux qui s'appellera,
d'ici 700 ans, Tokyo. Ce centre politique s'appelle le \emph{bakufu}, qui
était le nom de la tente de commandement dans les campements des armées en
campagne, montrant l'origine martiale. Chose primordiale : si, dans les faits,
l'Empereur est relégué aux rites religieux, il reste cependant,
traditionnellement, le vrai maître du Japon, de par son origine divine, le
shogun n'ayant qu'un rôle de régent. On voit là toute l'ambivalence
passionnante dans la logique japonaise : les choses dans la réalité ne sont ni
ce qui est dit, ni même ce qui est écrit. Il y a toujours un aspect de
tradition important à considérer, et ce, même si les faits vont en sens
contraire.

Attendez, ça va se compliquer encore un peu\dots\ Mais pour l'instant, closons
ce premier chapitre : l'ère féodale commence.

\chapter{Ère féodale}

\guil{Civilized men are more discourteous than savages because they know they
can be impolite without having their skulls split, as a general thing.}

Nous avions laissé notre voyage en 1185, alors que le Japon n'est plus gouverné
par un Empereur et sa cour, mais par un ancien chef de guerre, le shogun. Les
administrateurs et fonctionnaires ne comptent plus parmi les plus puissants ;
une société féodale se met en place, avec la caste du guerrier à sa proue. Ces
guerriers (les \emph{bushi}) jurent fidélité à leur clan, ce qui, par
transitivité de loyauté, revient à jurer fidélité à l'Empereur. Ils demandent
la même loyauté aux paysans, lesquels louent leurs terres à l'Empereur (donc
aux clans), et payaient leurs baux en nature.

Dans la pure tradition bien japonaise des régents et du pouvoir entre les mains
de celui qui n'est pas au pouvoir, à la mort du premier shogun, c'est le clan
de sa femme, les Hojo, qui règnent dans les faits, et occupent le rôle de
\emph{shikken}, comme régents du shogun. La capitale impériale reste à Kyoto,
et est considérée comme plus raffinée et cultivée que Kamakura, vue comme la
ville des gros bourrins de guerriers~\incise~mais, à tout le moins,
indépendante des intrigues bureaucratiques de la cour.

Un courant du bouddhisme, alors minoritaire, s'attire bientôt les faveurs des
samouraïs : le zen, avec l'importance donnée à l'ascèse martiale, à la
méditation disciplinée, au renoncement et à ce lien intime et mystérieux avec
la mort, séduit vite les guerriers. Cette importance du zen se voit encore dans
virtuellement tous les arts martiaux japonais~\incise~j'entends encore mon
premier prof de kendo me dire, après avoir enchaîné une dizaine de coupes sur
mon casque sans que je ne puisse ni les voir, ni encore moins faire grand-chose
pour les éviter : \guil{Évidemment que j'arrive à te toucher aussi
facilement\dots\ Tu es ailleurs ! Il faut être \emph{ici et maintenant} !}

Un événement exogène vient cependant perturber la situation actuelle : en 1274,
puis en 1281, Kubilai Khan, le petit-fils de Gengis Khan, pose ses yeux sur ces
petites îles. C'est le thème du second livre de Julien Peltier (\emph{Le Sabre
et le Typhon}), et je vous le conseille pour tout savoir de ces invasions.

En deux mots, cependant : les Mongols débarquent (un peu) par surprise, avec
une force supérieure en nombre, très disciplinée, mais surtout, totalement
étrangère aux Japonais. Aux défis lancés noblement par des cavaliers s'avançant
solitaires sur le champ de bataille, ces enflures répondent par des volées de
flèches lancées par de la piétaille populacière bigarrée. Scandale ! De même,
ils ne laissent pas tranquilles les écuyers chargés de récupérer les têtes des
ennemis tués par leur maître, afin de prouver, de fait, leur valeur, et de
demander une récompense en conséquence de ces trophées. Résultat, les Japonais
sont forcés de changer de tactique, et c'est une série d'escarmouches héroïques
qui leur permet de garder les envahisseurs bloqués dans leurs bateaux,
attendant des renforts. Là, un typhon providentiel ravage la flotte mongole qui
mouillait près des côtes japonaises : c'est le \emph{kamikaze}~\incise~mot à
mot, le \guil{vent divin}~\incise~qui reviendra d'ailleurs sept cent ans plus
tard, portant les chasseurs Zéro à Pearl Harbor.

Ces invasions mongoles, bien qu'une victoire japonaise, est un désastre pour le
shogun. D'une, la peur d'une troisième invasion subsistait, et le shogunat doit
racler le fond de ses caisses pour maintenir une armée prête à défendre les
côtes. De deux, les samouraïs n'ont rien conquis, et il est impossible pour le
shogunat de les récompenser avec des richesses ou des terres. De trois enfin,
un siècle après en 1278, à la mort de l'Empereur du moment, la famille
impériale tente de se rebiffer en se plaçant en désaccord sur les affaires de
succession avec le shogunat affaibli.

Ce sursaut de volonté politique cultmine en 1333, où le shogunat de Kamakura
est renversé par un coup d'état, mené par l'Empereur en date, Go-Daigo, et ses
alliés / vassaux Nitta Yoshisada, Kusunoki Masashige et Ashikaga Takauji. Je
reviendrai à la fin ces trois lascars, mais, pour la faire courte, le coup
d'état se révèle être un échec pour la lignée impériale. Le shogunat de
Kamakura, totalement dépassé par les événements, s'éteint avec lui, et c'est
Ashikaga Takauji qui fait volte-face pour prendre le pouvoir. Il écrase la
coalition de Go-Daigo et se fait proclamer shogun à la place du shogun. La
capitale shogunale revient à Kyoto, au même endroit que la cour impériale. Le
shogunat des Ashikaga est globalement similaire à celui des Minamoto d'avant, à
une exception de taille près : contrairement au bakufu de Kamakura, les
Ashikaga n'ont jamais réussi à centraliser leur pouvoir, reposant plutôt sur
une coalition vaguement stable d'une multitude pyramidale de seigneurs de
guerre régionaux. Ces seigneurs de guerre se font appeler les \emph{daimyo}
(les \guil{grands noms}), et vont avoir une importance prépondérante dans notre
prochain chapitre.

En tous les cas, la culture flamboie, le no se développe, la cérémonie du thé
se codifie, le zen explose (pour autant que le vide puisse exploser,
s'entend)~\incise~le petit fils de Takauji construit d'ailleurs le Pavillon
d'Or à Kyoto, et les shoguns, petit à petit, perdent leur influence unifiante,
notamment dans les contrées éloignées de la capitale. Les conflits de petite
envergure entre tel ou tel chef régional sont de plus en plus légion, et cette
situation atteint son apogée en 1467, à l'occasion, vous aurez deviné, d'une
dispute de succession. Le 8\up{e} shogun de la dynastie de Ashikaga n'a pas
d'héritier, demande à son frère cadet, alors moine, de quitter les ordres,
puis, merde, un fils lui naît. Deux clans ennemis sautent sur l'occasion, et en
viennent même à se battre dans les rues de la capitale, pendant dix
ans~\incise~c'est dire à quel point l'autorité du shogun avait décliné, lequel
shogun d'ailleurs, un peu comme Néron, était en train de dessiner les plans
d'un Pavillon d'Argent pendant que sa ville brûlait et que son pays plongeait
dans la guerre civile. C'est la Guerre d'Onin, qui sert classiquement de point
de départ de la prochaine étape de notre voyage, l'époque Sengoku : l'âge des
provinces en guerre.

J'ouvre une parenthèse en revenant sur la restauration ratée de l'Empereur en
1333. Kusunoki Masashige est un tacticien brillant, loyal à l'Empereur, qui lui
a permis de revenirau pouvoir (quoique brièvement), en défendant avec succès
deux forteresses clefs. Quand Ashikaga Takauji trahit la cause impériale et se
retourne contre ses anciens alliés pour récupérer le pouvoir, Masashige
conseille à l'Empereur de se réfugier dans le temple bouddhique du mont Hiei,
colline sacrée surplombant Kyoto, et de laisser son ennemi reprendre la
capitale pour mieux fondre sur lui ensuite. Go-Daigo, orgueilleux, totalement
réactionnaire, impatient dans ses réformes, cruel envers les paysans et
méprisant envers les guerriers, refuse, et ordonne à son lieutenant de
confronter ses forces épuisées, affamées et largement inférieures à celles de
Takauji, dans les plaines, en dehors de Kyoto. Masashige, convaincu de sa
défaite~\incise~et de sa mort~\incise~obéit dans un acte ultime de loyauté
envers son suzerain, et rencontre Takauji à côté de la future Kobe, en laissant
son poème de mort à son jeune fils. La légende dit que son frère lui aurait
lancé un cri qui sera maintes fois repris dans l'histoire, \guil{Shichisei
hokoku !} (\guil{Que j'ai sept vies à donner pour mon pays !}), ce à quoi
Masashige, fidèlement entêté, et donc totalement en opposition à la
réincarnation karmique des bouddhistes, aurait répondu par une charge
désespérée. Son armée totalement encerclée, lui restant moins de 50 cavaliers
sur les 700 initiaux, il meurt sabre à la main, sans jamais avoir eu aucune
chance de succès.

Aucun espoir de victoire, servant un seigneur réactionnaire, réfractaire à la
moindre idée de progrès, et sans avoir accompli la moindre chose\dots\ et
pourtant, c'est Kusunoki Masashige qui représente le plus grand héros dans la
psyché japonaise. De son côté, Ashikaga Takauji est vu comme un vil
traître~\incise~quand ce fut, historiquement, un tacticien excellent, un lettré
raffiné et un politicien brillant. C'est là à mon sens la différence
fondamentale entre l'archétype du héros japonais et son pendant occidental, et
celle qui personnellement me touche au plus haut point. En Europe, un héros va
être un preux chevalier, moralement irréprochable, qui fonce dans un combat
quasi-impossible, mais qui ressort victorieux, grâce à son talent et, souvent,
à un peu de chance, en dépit des probabilités qui le donnent perdant. Quand
hélas il meurt, c'est toujours en donnant naissance à quelque chose de plus
grand, afin que ses vertus lui survivent. Il se sacrifie, d'une certaine
manière, pour que ce en quoi il croyait, forcément moralement irréprochable, se
réalise. Au Japon, au contraire, c'est celui qui échoue qui est glorifié. En
effet, celui qui réussit n'a pu le faire qu'en accommodant ses valeurs, pour
les mettre en adéquation avec la réalité, forcément imparfaite.  Alors que
celui qui meurt dans l'échec le plus total, c'est parce qu'il est resté têtu,
juste, sincère, véridique, fidèle~\incise~et ce, quelles que soient ses
croyances. Celui qui échoue, c'est celui qui n'a jamais compromis, qui est
resté pur. C'est, moins l'efficacité ou le bien-fondé rationnel des valeurs
d'une personne que l'expression de son intensité qui en fait la qualité.

Deux héros occidentaux possèdent, à mon goût, ces caractéristiques : don
Quichotte et Cyrano (\guil{Que dites-vous ? C'est inutile ? Je le sais ! Mais
on ne se bat pas dans l'espoir du succès ! Non, non ! C'est bien plus beau
lorsque c'est inutile !}), et ce n'est pas bien surprenant que les westerns,
notamment de Leone, s'inspirent des histoires japonaises, pleines de ces coups
d'éclat stupides et beaux : les duels à mort, entre deux experts,
irrationnellement têtus, vivant ce qu'ils croient plutôt que de professer,
nobles dans leurs vertus, mais des dinosaures bientôt oubliés dans les faits,
le six-coups à la ceinture, totalement incongru à l'âge où le train à vapeur
arrive\dots\ Voilà une image que les bretteurs de Muromachi n'auraient pas
renié. Les Japonais appellent cette vertu-là \emph{makoto}, habituellement
traduit par \guil{sincérité}. C'est une des sept vertus du guerrier, vertus qui
sont, comme le veut la tradition, symbolisées par les sept plis du
hakama~\incise~et vous comprenez pourquoi il faut s'efforcer de soigneusement
ranger sa jupette à la fin du cours.

\chapter{Ère des province en guerre}

\guil{I remember days like this when my father took me to the forest and we ate
wild blueberries. More than 20 years ago\dots\ I was just a boy of four or
five. The leaves were so dark and green then. The grass smelled sweet with the
spring wind\dots\ Almost twenty years of pitiless combat ! No rest, no sleep
like other men. And yet the spring wind blows, Subotai. But for us, there is no
spring. Just the wind that smells fresh before the storm.}

Nous nous étions quittés tantôt alors que l'influence du shogun déclinait
grandement au Japon.  Rappelons que, tout mandaté par les Cieux qu'il soit,
l'Empereur n'avait aucun pouvoir politique, et restait cantonné dans sa
capitale. Le conflit initial dont nous parlions, la Guerre d'Onin, annonce la
prochaine période de notre voyage : \emph{Sengoku jidai}, l'âge des provinces
en guerre. C'est cette période qui est largement décrite dans le premier livre
de Julien Peltier que Jérôme avait ramené au dojo (\emph{Le Crépuscule des
Samouraïs} si ma mémoire est bonne).

Cette période de conflits virtuellement ininterrompus peut se modéliser en deux
phases : une phase de morcellement du Japon, qui se transforme petit à petit en
une multitude de petits seigneurs de guerre locaux autonomes, les
\emph{daimyo}, qui cherchaient à annexer leurs voisins ; puis, une phase
d'unification, les daimyos restants devenant de plus en plus puissants, qui
culmine en 1600, quand Tokugawa Ieyasu, victorieux de tous ses adversaires,
fonde le shogunat qui porte son nom, et qui durera plus de 250 ans.

En passant, outre le livre de Peltier, et à condition d'avoir une babasse
conséquente pour le faire tourner, je vous conseille très chaudement le jeu de
stratégie / gestion \emph{Total War : Shogun 2}, sorti l'an dernier, qui vous
place dans le kimono d'un de ces daimyos. À vous de monter des alliances, de
lever des armées et de déferler sur un Japon en plein chaos pour vous hisser à
la tête du pays. À quelques détails près, le jeu est historiquement solide, et
permet d'apprendre sans trop de difficultés la multitude de noms des gens
d'importance de l'époque. Et, très honnêtement, lancer ses ashigarus à l'assaut
des pentes douces en pierre des châteaux japonais, pour les voir ensuite
échanger des passes d'armes martialement réalistes avec les sabreurs ennemis,
sashimonos flottant au vent dans leur dos, ça n'a pas de prix. De même, c'est
cette période qui est souvent mise en scène dans les films de sabre, comme
\emph{Les Sept Samurai}, \emph{Ran} ou \emph{Kagemusha} de Kurosawa, ou la
série-fleuve \emph{Shogun}, qui, à défaut d'être historiquement exacte, permet
de voir le château d'Osaka.

Une autre appellation de cette \guil{période des provinces en guerre} est
\emph{Gekokujo} : le monde à l'envers.  Effectivement, pendant près de 150 ans,
le Japon va être le théâtre d'une guerre permanente, pleine de trahisons, de
rivalités, d'alliances oubliées, de mariages arrangés pour consolider un
avantage stratégique ou enterrer des querelles, de familles qui se brisent, de
clans qui se scindent puis s'allient avec leurs anciens ennemis\dots\ Le plus
bas des fantassins peut devenir, en quelques années, le seigneur le plus
puissant du pays, puis être assassiné par son lieutenant ; les fils
s'entretuent à la mort du père, les vassaux oublient leurs v\oe{}ux de fidélité
en l'absence de leurs seigneurs, les moines fondent des ligues de soldats, et
se soulèvent contre leurs maîtres. Même l'Empereur est obligé de vendre des
estampes pour manger !

C'était donc une période ou l'efficacité brute prime sur la pureté morale. La
fin~\incise~principalement, la survie de son clan~\incise~justifie tout, ou
presque. J'en veux pour preuve le nombre de codes et préceptes, écrits par les
chefs de clan à l'adresse de leurs fils et subordonnés, les exhortant à la plus
grande fidélité~\incise~un peu comme les codes de chevalerie insistait sur les
vertus que devaient posséder ceux qui portaient l'armure. La figure du samourai
comme parangon de nobles valeurs n'existe pas encore\dots\ Pourtant, c'est dans
ce climat inconstant de trahisons que l'on peut trouver les plus grands actes
de bravoure et les plus beaux exemples de loyauté désespérée qui préfigurent
l'archétype vertueux du samouraï. Ainsi, en 1600, Torii Mototada, vassal de
Tokugawa Ieyasu, resta avec seulement 2\,000 hommes dans le chateau de Fushimi,
afin de permettre à son seigneur de regrouper ses forces, et stoppa une colonne
ennemie de 40\,000 hommes pendant une semaine~\incise~au bout de laquelle,
implacable arithmétique !, les assiégés auront tous péri. Cet acte permit à
Ieyasu de remporter la bataille décisive de Sekigahara (nous y reviendrons). La
veille de l'assaut, les deux hommes échangèrent une dernière coupe de sake, et
Mototada proposa même de réduire la garnison du château, pour accroître les
forces de Ieyasu, certain et lucide qu'il tomberait quel qu'en soit le nombre.

De même, Takeda Shingen (le chef de clan à la cavalerie puissante, mis en scène
dans \emph{Kagemusha}) et Uesugi Kenshin, nous offrent de jolies anecdotes. Ils
guerroyèrent l'un contre l'autre pendant plus de quatorze ans, au point de se
rencontrer sur les mêmes rives d'une rivière frontalière quatre fois durant
leur règne. Sur son lit de mort, Shingen aurait instruit son fils de s'appuyer
sur Kenshin ; ce même Kenshin qui, en apprenant la mort de son ennemi, aurait
pleuré la disparition d'un adversaire aussi valeureux. En 1568, une coalition
de daimyos coupent l'accès au sel à la province des Takeda~\incise~le sel était
alors une denrée précieuse, notamment pour préserver la nourriture. Uesugi
Kenshin envoya secrètement du sel aux Takeda, arguant qu'un tel acte n'était
pas honorable, ajoutant que \guil{les guerres doivent être gagnées au sabre et
à la lance, pas avec du riz et du sel}.

Les armes, armées et techniques de combat évoluent~\incise~simple
darwinisme~\incise~extrêmement vite. En grossissant un peu le trait, d'une
succession de duels d'archerie montée entre deux nobles, on passe petit à petit
à des batailles de plus en plus importantes, mobilisant de plus en plus
d'hommes. Les mentalités changent, et les généraux font appel, de plus en plus,
à des soldats issus du peuple, les \emph{ashigaru} (\guil{pieds légers}, comme
dans \emph{tsugi ashi}, les \guil{pieds chassés} que nous apprenons en aiki),
plaçant certains samouraïs à la tête de ces fantassins. Ceux-ci sont légèrement
protégés et armés principalement d'une lance. Simultanément, l'arc ne reste pas
l'apanage des aristocrates, mais les généraux comprennent bien vite l'avantage
de larges volées de flèches, quitte à ce qu'elles soient tirées par de la
piétaille. Le sabre sert là toujours d'arme d'apparat, ou
d'appoint~\incise~notamment lors du rituel, maintenant codifié, socialement
valorisé et largement utilisé, du seppuku. Les récompenses obtenues dépendant
directement de la valeur prouvée sur le champ de bataille, c'est aussi le temps
du flamboiement baroque des armures et des habits : étendards de couleurs,
bannières individuelles, décorations sur les casques\dots

Les châteaux évoluent également : l'impossibilité géographique (tremblements de
terre) et l'inutilité stratégique (pays trop montagneux pour que l'artillerie
se déplace) d'avoir des hautes murailles, alliée à la stratégie en vogue du
moment (se reposer sur une infanterie nombreuse) fait que les chateaux forts
japonais n'ont pas du tout la même tête que nos châteaux médiévaux. Plutôt que
des murailles larges et hautes, ils sont souvent bâtis en haut d'une colline
surplombant un intérêt stratégique, avec un système de forteresses secondaires,
permettant aux défenseurs de se replier au fur et à mesure, et d'enceintes
successives, formant un labyrinthe mortel pour les assaillants. Ainsi, un
attaquant devait parcourir plus de quatre kilomètres pour couvrir les cinq cent
mètres à vol d'oiseau séparant l'enceinte extérieure du donjon principal du
château de Himeji (si ma mémoire est bonne) !

Des conscrits à la lance, en masse ; des archers, au début du combat ; des
samourais de petit rang, qui se battaient parfois à pieds, à la lance ou au
glaive ; des samourais de haut rang, commandant tout ce petit monde\dots\ Et en
1543, un événement de taille va bouleverser tout ça. Sur l'île Tanegashima, au
sud de Kyushu, un bateau portugais, en route vers la Chine mais qui avait dévié
de sa route, s'échoue.  Il contenait deux choses révolutionnaires : un
missionaire catholique, et une arquebuse. Extrêmement vite, les armes à feu
envahissent tous les champs de bataille, aidé par le savoir-faire japonais de
fabrication de l'acier (cf., si besoin était, la qualité de leurs lames). Moins
de quinze ans plus tard, Date Masamune, le daimyo contrôlant la pointe nord du
Japon, à 3\,000 kilomètres de Tanegashima, en utilise. Le Japon a été tellement
enthousiaste de ces nouvelles armes qu'il aurait dépassé, en nombre, tous les
pays européens à la même date.  Et, bien que les fusils restaient primitifs et
encombrants, ils se révèlent bien souvent décisifs. Ainsi, en 1575, à la
bataille de Nagashino, opposant le fils de Takeda Shingen à Oda Nobunaga, et
mis en scène dramatiquement à la fin de \emph{Kagemusha}, c'est par une
tactique de tirs en série disciplinés que la charge effrayante de la cavalerie
lourde des Takeda fut cassée. On voit d'ailleurs là l'ambivalence morale
japonaise : d'un côté, une arme techniquement très efficace ; mais, bien peu
honorable, puisqu'un paysan boueux et crotté pouvait, en quelques jours de
formation, tuer un noble aristocrate, lettré et raffiné, qui s'entraînait aux
armes depuis trente ans, quel scandale ! Quant au christianisme, il s'est
également diffusé comme une trainée de poudre dans tous le pays. Les historiens
estiment qu'en une cinquantaine d'années, 300 000 Japonais étaient baptisés.
Nous y reviendrons dans notre prochain chapitre d'ailleurs. 

On distingue habituellement trois grandes figures unificatrices du Japon : Oda
Nobunaga, Toyotomi Hideyoshi puis Tokugawa Ieyasu. Un dicton prétend que
\guil{Nobunaga confectionne le gâteau de riz, Hideyoshi le pétrit, et Ieyasu
s'assoit et le mange}. Si Oda Nobunaga n'a pas pu, avant sa mort, unifier
toutes les provinces du Japon, c'est lui le premier qui décide de ré-établir un
semblant d'autorité dans le pays, en forçant l'Empereur à nommer un de ses
pantins comme shogun. En 1582, alors qu'il part rejoindre un allié en
difficulté~\incise~un certain Toyotomi Hideyoshi~\incise~il est trahi par un de
ses lieutenants, qui le force à se suicider. Hideyoshi apprend la mort de son
seigneur, et rentre d'urgence pour châtier le traître. De fait, il récupère
alors les troupes d'Oda et se retrouve catapulté à la tête d'un des clans les
plus puissants. Notons que Hideyoshi est fils de fermier, a commencé comme
domestique d'un vague lieutenant, et est laid comme un cul de cynocéphale (on
le surnomme, dans son dos bien sûr, \guil{face de singe}~\incise~d'où le titre
de sa biographie écrite par Shiba Ryotaryo, \emph{Seigneur-Singe}). Ambitieux,
il soumet clan par clan, bataille par bataille, chaque famille, débarque en
1587 sur l'île de Kyushu avec 150\,000 hommes, et assoit sa domination sur la
totalité du Japon en 1590. Dans une poussée d'orgueil, il pose même ses yeux
vers la Chine, et débarque, en 1592 en Corée, avec 200\,000 hommes. À la suite
d'une conquête facile, grâce notamment à des combattants entraînés et
disciplinés, équipés d'armes à feu, face à une armée prise par surprise et peu
nombreuse, la Corée parvient cependant à bouter les envahisseurs grâce à sa
flotte. Celle-ci est extrêmement moderne~\incise~le bateau-tortue, outre une
puissante artillerie embarquée, était le premier navire cuirassé au monde, 275
ans avant le premier cuirassé occidental !~\incise~et commandée par un maître
stratège, véritable héros en Corée, Yi Sun-Sin. Résultat, les Japonais font
marche arrière, évacuent en ne laissant guère que quelques fortifications
derrière eux~\incise~et un paquet d'exactions qui préfigurent celles qui
viendront trois cent ans plus tard.

En tous les cas, Toyotomi Hideyoshi meurt en 1598, et son ami, Tokugawa Ieyasu,
également un ancien lieutenant (et otage !) d'Oda qui prend la tête des
troupes. Mais tous les daimyos ne sont pas d'accord ! Le Japon est encore
divisé en deux clans, l'un favorisant Ieyasu, l'autre préférant le fils
héritier de Hideyoshi.  La titanesque bataille de Sekigahara, en plein centre
du pays, décide du vainqueur : plus de 24 heures de bataille, qui s'ouvrent sur
un brouillard d'octobre épais, 200\,000 combattants, équitablement répartis,
des charges héroïques, des trahisons en plein milieu, des arrivées
providentielles, des éclairs de génie tactique qui finalement ne marchent
pas\dots\ Y'aurait moyen d'en faire un flim sacrément épique. Un certain
Musashi Miyamoto, d'ailleurs, y participe~\incise~et est dans le camp des
vaincus. Ieyasu remporte de manière décisive, face aux autres forces féodales
qui s'opposaient à sa prise de pouvoir, se fait nommer shogun, place sa
capitale à Edo, un petit village (qui s'appellera Tokyo dans 250 ans), et fonde
une dynastie qui garantira enfin la paix à son pays. Nous verrons ça la
prochaine fois\dots

\chapter{Ère d'Edo}

\guil{The secret of steel has always carried with it a mystery. You must learn
its riddle, Conan. You must learn its discipline. For no one, no one in this
world can you trust. Not men, not women, not beasts. [pointant l'épée du doigt]
This you can trust.}

Arrivé au pouvoir par une série de circonstances incertaines, Tokugawa Ieyasu,
et, par la suite, ses héritiers successifs, vont tout mettre en oeuvre pour
rendre cette situation impossible. Le shogunat va en effet cloisonner en castes
quasi-imperméables toute la société du Japon : les daimyos (environ 250) tout
d'abord, qui sont les anciens chefs de clan, alliés ou ennemis, qui règnent sur
leurs provinces avec une certaine autonomie ; les samurai ensuite (environ
400\,000, entre 5\% et 10\% de la population), les seuls à avoir le droit de
porter les armes, qui servent de police aux ordres du shogun et de leur clan,
auquel ils devaient fidélité, sous peine de mort ; les paysans ensuite (un bon
80\% de la population) qui, s'ils jouissent d'un certain prestige social, vont
être de plus en plus écrasés sous les impôts ; et puis, tout en bas, les
artisans et les marchants, qui pourtant vont accumuler des richesses dépassant
largement celles des autres strates. Tout cela était contrôlé de manière
étroite par des officiels, et maintenu par une série de cérémonials et une
étiquette rigoureusement définie, empêchant les gens d'un certain rang d'aller
fricoter avec ceux d'un autre rang.

Tout est prétexte à être codifié, pour accroître le cloisonnement formel de la
société : on écrit des lois limitant l'habillement de tel ou tel rang, les
coiffures et les accessoires deviennent des marques d'appartenance à un groupe,
certains mots sont même rayés du vocabulaire des castes inférieures.
Parallèlement, dès le début du XVII\up{e} siècle, les Tokugawa interdisent le
christianisme, afin de garantir la loyauté des paysans envers leur daimyo,
persécutant violemment les convertis. Craignant que les étrangers soient à
l'origine d'une conquête européenne du Japon, le shogunat instaure une
politique d'isolement de l'île, nommée \emph{Sakoku} (\guil{fermeture du
pays}). Cette politique interdit, sous peine de mort, à tout Japonais de sortir
ou ré-entrer du territoire, puis expulse tous les étrangers du sol japonais, à
l'exception ultra-régulée, de négociants Hollandais (qui réussirent à
convaincre le shogun qu'ils étaient protestants et pas catholiques comme les
Portugais) dans le port de Nagasaki.  C'est \emph{via} ce contact que le Japon
se tient au courant des avancées technologiques occidentales (planches de
médecine, astronomie, horlogerie, électricité, automates\dots).

Le shogunat met également très vite en place un système de mesures empêchant
les daimyos de se rebeller. Même dans ce groupe de 250 personnes, on trouve
trois rangs bien distincts : la famille étendue des Tokugawa ; les daimyo
\guil{fudai}, qui sont ceux qui étaient alliés à Ieyasu avant la bataille de
Sekigahara, et qui récupèrent des fiefs stratégiques, proches de la capitale et
sur le long des routes ; puis les daimyo \guil{tozama}, vaincus à la bataille
de Sekigahara, qui récupèrent les fiefs les plus éloignés d'Edo. Nous y
reviendrons au prochain chapitre, d'ailleurs.  Conformément à leur rang, tous
les daimyos se doivent de maintenir dans la capitale une grande résidence
richement décorée, dans laquelle doit séjourner une partie (souvent la femme et
les fils) de leur famille~\incise~une prise d'otages à peine déguisée.
Parallèlement, les daimyos sont tenus de venir présenter, six mois par an,
leurs respects au shogun, ce qui leur impose d'organiser de vastes processions
ruineuses, particulièrement pour les fiefs les plus éloignés (\emph{sankin
kotai}, \guil{service en alternance}). Louis XIV en son temps exigeait
d'ailleurs la même chose de la noblesse française.  Enfin, ils doivent veiller
à l'entretien des temples et des routes de leur fief, et doivent demander la
permission au shogun d'effectuer des réparations sur leur chateau.

Dans cette société de castes fermée, bétonnée, le Japon jouit pourtant d'une
paix durable. En 1860, plus d'un tiers de la population était lettrée, et ce
sans que l'éducation ne soit obligatoire. Le Japon publiait plus de livres que
toute l'Europe réunie. Un visiteur allemand dont le nom m'échappe s'étonne de
voir les gens, du plus noble au plus populaire, sortir un mouchoir en papier
pour se moucher, plutôt que de le faire dans sa manche comme il était coutume
chez lui. Les arts foisonnent, notamment la poésie et l'écriture de haikus, la
peinture avec les estampes (Hokusai, Hiroshige, Utamaro\dots), le théâtre avec
le kabuki et le bunraku (théâtre de marionnettes). À propos des estampes, c'est
notamment \emph{via} le système d'astreinte du \emph{sankin kotai} et tout le
monde qui gravite que Hiroshige peint ses \emph{53 Stations du Tokaido} ou
Hokusai ses \emph{36 Vues du mont Fuji} (contenant notamment la célèbre vague),
un peu comme un Guide du Routard à l'attention des voyageurs voyageant entre
Edo et Kyoto. Petite parenthèse personnelle, la finance aussi se met en place,
avec les instruments de crédit, et les premières options, utilisées par les
marchands autour d'Osaka.

Et côté armes ? Déjà, le shogun, connaissant la puissance des armes à feu,
interdit la création d'arquebuses et les fait saisir, prétextant avoir besoin
de bronze pour fondre une statue de Bouddha, afin de remercier les dieux de
cette période de paix. Mais à quoi peuvent servir la masse énorme de soldats
qui avaient servi dans ces batailles gigantesques de l'ère précédente ? Si la
majorité retourne à ses champs, un nombre conséquent d'anciens samourais,
devenus sans maître ni cause à servir, deviennent des \guil{ronin}, des
\guil{hommes (ballotés par les) vagues}. Parallèlement, sans guerre à mener,
les samourais encore employés comme soldats perdent leur fonction première, et
sont alors des bureaucrates au service de leur daimyo. La paire de sabres, le
grand katana et le petit wakizashi sert de marque de statut social, tout comme
le chignon relevé sur la tonsure. Certains cependant déplacent la recherche de
l'efficacité martiale brute, si désirée pendant la période de tumultes
précédente vers une recherche de connaissance et de développement de soi.
Ainsi, poussé par une logique néo-confucéenne, qui veut qu'une société prospère
quand les supérieurs donnent l'exemple vertueux à suivre aux inférieurs,
certains idéaux du guerrier, qui préfigurent le Bushido (la voie du samouraï,
si l'on veut), commencent à être formalisés. Les anciens forgerons, plutôt que
d'équiper en masse des armées qui n'existent plus, vont chercher l'excellence
de leur art. De même, privés de leur utilité première, certains anciens
guerriers vont utiliser les techniques martiales pour les transformer en
discipline de vie~\incise~et notamment avec ce qu'ils ont sous la main,
c'est-à-dire leur corps et leurs sabres~\incise~et donnent naissance aux arts
martiaux. Les \emph{jutsu}, les pratiques guerrières, deviennent des \emph{do},
des disciplines, des voies.

C'est dans cette atmosphère que Yamamoto Jocho écrit le \emph{Hagakure}, un des
livres fondateurs du mythe moderne du samourai. C'est ce livre que Jarmusch
fait lire à Forest Whitaker dans \emph{Ghost Dog}, par exemple, et c'est le
\emph{Hagakure} que Mishima considérait comme son livre de chevet, au point
d'en avoir écrit un commentaire. On notera (ironiquement ? Ce serait ne pas
avoir compris le principe d'un mythe\dots) que Jocho décrit l'idéal du samourai
en ayant vécu un bon siècle après la dernière bataille que le Japon ait connu.
C'est également dans cette atmosphère que Musashi, le bretteur légendaire,
écrit son \emph{Traité des Cinq Roues}. À ce sujet justement, pendant l'ère
d'Edo, les écoles de disciplines de combat divers se multiplient. Pour le sabre
par exemple, on en recense plus de 2\,000 différentes en 1750, avec chacune un
\guil{style} qui lui est propre. C'est vers cette période d'ailleurs que le
shinai (le sabre en bambou) et le bogu (l'armure), encore utilisés quasiment
tels quels en kendo, sont développés, histoire d'arrêter d'utiliser un bokken
pour s'entraîner au combat. Les écoles se défient fréquemment, afin de prouver
au public, et aux élites en particulier, la supériorité de leurs styles. Les
élèves sont d'ailleurs encouragés à mener un pèlerinage (\emph{musha shugyo}),
arpentant seuls le pays pour chercher à parfaire leurs connaissances.

Tout pourrait donc aller plus ou moins pour le mieux dans le meilleur des
mondes\dots\ mais le souci est double : d'une, ce monde est fermé, et
l'extérieur se développe sacrément vite (l'Europe entre en pleine révolution
industrielle) ; de deux, ce monde est dramatiquement mal géré. La strate des
samurai, interdite de cultiver ou, pire, de commercer, sous peine d'être
destituée, voit ses revenus, indexés aux mêmes niveaux de riz qu'à l'avènement
du shogunat, donc soumis à une inflation énorme depuis 200 ans, s'évaporer
comme rosée au soleil d'été. Résultat, ils s'endettent auprès des commerçants ;
tout le monde s'endette, même le shogunat, qui n'a jamais taxé le commerce, et
voilà qui donne aux marchands une importance capitale, qu'ils traduisent en
influence politique. La grogne monte, l'autorité du shogunat est remise en
question, certains villages se soulèvent\dots\ Et, que vois-je au loin ? Quatre
navires, chacun peut-être dix fois plus large que le plus gros de nos
\emph{senki bune}, avec des colonnes énormes qui crachent une fumée noire, et
battant un pavillon étoilé inconnu\dots\ Voilà qui n'augure rien de bon, et
annonce la fin de ce quatrième chapitre !

\chapter{Restauration Meiji}

\guil{Someday, when all your civilization and science are likewise swept away,
your kind will pray for a man with a sword.}

1853~\incise~le début de la fin pour les samurai. Préparez vos mouchoirs, c'est
le moment tragique. En même pas vingt ans, le pouvoir du shogun, et, par là,
les samurais, va s'éteindre, et l'Empereur va de nouveau être à la tête du
pays. C'est une période que je trouve proprement fascinante, j'espère que vous
m'excuserez si je ne résiste pas à la tentation de rentrer dans certains
détails. L'Histoire a retenu de cette transition qu'elle a été douce, sans
heurts et pacifique, louant les capacités d'adaptation des Japonais~\incise~le
nom même est \guil{Restauration Meiji}, pas \guil{Révolution} ou \guil{Guerre
civile}~\incise~et rien ne serait moins faux.

Un matin de juillet, le Commodore Perry, de la marine américaine, débarque
avec quatre cannonières à vapeur dans la baie d'Edo, la capitale. Il ouvre les
négociations visant à faire sortir le Japon de son isolation, limitant le
commerce avec l'extérieur (et, notamment, avec les puissances occidentales),
en faisant preuve de la puissance de son artillerie embarquée, qui terrorise
les Japonais. Autour du shogun, deux écoles s'opposent : \textcolor{blue}{ceux
qui refusent entièrement de traiter avec les étrangers}, au risque d'aller à
la guerre, et \textcolor{red}{ceux qui sont prêts à les accueillir}. Dans ce
camp-là d'ailleurs, il ne faut voir aucune forme d'amitié particulière. La
plupart, conscients du danger que pose la technologie étrangère supérieure sur
le pays, cherchent à compromettre afin d'apprendre pour mettre le Japon au
même niveau que le reste du monde. Des intellectuels synthétisent tout cela en
théorisant un Japon armé de \guil{connaissances de l'ouest} et de la
\guil{morale de l'est}, afin de \guil{contrôler les barbares avec leurs
propres méthodes}. Le \textcolor{red}{shogunat}, séduit par cette option, et
contraint par la \guil{politique de la canonnière} de Perry et de ses
successeurs, signe avec les États-Unis plusieurs traités de paix. Ces traités
sont cependant dans les faits des conventions très inégales, semblables
d'ailleurs au traité de Tianjin, signé entre la Grande-Bretagne et la Chine
après la seconde guerre de l'opium, pendant la même année. Le Japon signe bien
vite d'autres traités similaires avec les autres puissances (Angleterre,
France, Russie, Pays-Bas\dots). Ces traités ouvrent les grands ports japonais
aux étrangers, les laissent libres de commercer, avec des taxes à l'import et
à l'export extrêmement faibles.

L'ouverture du Japon à ce commerce extérieur incontrôlé entraîne une forte
instabilité économique. Parallèlement, des tremblements de terre et de
mauvaises récoltes à répétition font énormément augmenter le prix de la
nourriture, qui n'en avait pas besoin vu l'inflation effrayante qui avait
cours. Si le \textcolor{red}{shogunat} envoie de nombreuses missions,
notamment en Allemagne et en France (une délégation participe à l'Exposition
Universelle de 1867 à Paris) et tente de faire réviser les traités
inégalitaires, son autorité décroit rapidement. On assiste de nouveau à un
morcellement du Japon, les daimyos reprenant peu à peu leur indépendance.
Schématiquement, ceux qui étaient dans le camp des vainqueurs à la bataille de
Sekigahara en 1600, \textcolor{red}{les fudai-daimyo}, restent loyaux au
shogun ; les autres, \textcolor{blue}{les tozama-daimyo }qui avaient été
placés dans les fiefs les plus lointains de la capitale, s'opposent au
contraire résolument au shogunat. Les camps de \textcolor{blue}{Choshu}, au
sud-ouest de Honshu (l'île principale), \textcolor{blue}{Tosa} sur l'île de
Shikoku et \textcolor{blue}{Satsuma}, à l'extrémité sud de l'île de Kyushu,
sont les plus virulents dans leur opposition, et vont jouer un rôle
déterminant dans les événements qui vont suivre.

Comme d'hab', une crise de succession au shogun vient rajouter de l'huile sur
un feu crépitant. Si les \textcolor{red}{fudai daimyo} gagnent cette lutte de
pouvoir, une purge brutale et sanglante des \textcolor{blue}{opposants}
accentue le ressentiment contre les \textcolor{red}{Tokugawa}, qui en sortent
encore plus affaiblis. Le mouvement \emph{sonno joi}, \guil{Révérer l'Empereur,
expulser les barbares !}, devient un slogan de ralliement dans les provinces de
\textcolor{blue}{Choshu et de Satsuma}. Cette philosophie, outre d'être un
totale opposition avec la politique du \textcolor{red}{shogunat}, contient
également une idée nationaliste forte~\incise~le caractère pour \guil{barbare}
comportant une notion péjorative de \guil{race}. Elle rassemble autour d'elle
la plupart des castes guerrières, qui n'appréciaient pas de signer sous la
contrainte, et qui voyaient en outre leurs pouvoirs et leurs prérogatives
diminuer. La violence grandit contre les étrangers et ceux qui commercent avec
eux. L'ambassade britannique se fait attaquer, le \textcolor{red}{signataire}
qui avait signé le \guil{Traité d'amitié et de commerce} avec les américains se
fait assassiner en pleine rue, l'ambassade américaine est incendiée\dots\ C'est
également à cette occasion que les occidentaux découvrent le rituel du seppuku
: en 1868, onze marins d'une frégate française qui mouillait dans la baie
d'Osaka, sont tués par des guerriers de \textcolor{blue}{Tosa}. L'ambassadeur
de France au Japon proteste et exige un châtiment exemplaire : le daimyo de
\textcolor{blue}{Tosa}, défiant, présente dès le lendemain vingt samourais
jugés coupables, et condamnés à mort par seppuku. La magnanimité du capitaine
de vaisseau, assistant à l'exécution volontaire, permet à neuf d'entre eux
d'être graciés. D'autres témoignages insistent au contraire sur son effroi, les
samourais, dans une ultime bravade morbide, lui jetant leurs intestins à ses
pieds. On peut lire d'ailleurs ses mémoires décrivant la scène sur le net.

L'opposition armée à l'influence occidentale dégénère alors en réelle guerre
civile lorsque \textcolor{blue}{l'Empereur Komei}, rompant avec des siècles de
tradition~\incise~l'Empereur n'avait pas reigné effectivement sur le Japon
depuis 800 ans !~\incise~prend un rôle actif dans les affaires d'État. Il
reprend le slogan \guil{sonno joi} en proclamant en 1863 l'ordre (divin)
d'expulser les barbares, défiant par là ouvertement la politique du
\textcolor{red}{shogunat}. Le clan \textcolor{blue}{Choshu} suit cet ordre à la
lettre et fait tirer sans avertissement sur tous les navires étrangers qui
tentaient de traverser le détroit de Shimonoseki, entre l'île de Kyushu et
l'île d'Honshu. Petite anecdote : \textcolor{blue}{Choshu} tire avec d'antiques
canons (certains en bois), datant d'avant 1600, construits pendant l'ère
Sengoku. En quelques semaines, des navires américains, français, hollandais et
britanniques se font tirer dessus, avec une artillerie antédiluvienne. Quelques
jours plus tard, l'ordre est donné d'aller châtier ces insulaires têtus et
sous-développés : c'est la bataille de Shimonoseki, un combat peu
connu~\incise~et plutôt inégal~\incise~dans lequel des frégates d'un peu toutes
les nations vont bombarder les forteresses japonaises. La portée des canons
occidentaux dépasse de loin celle des pétoires japonaises vieilles de trois
siècles. Les français débarquent d'ailleurs quelques hommes pour sécuriser le
détroit, et capturent les pièces d'artillerie japonaise. Vous pouvez voir ces
canons exposés devant la porte nord des Invalides, devant le Musée de la
Guerre, et vous amuser à discerner le blason de Choshu gravé à côté de leurs
bouches. En passant, un diplomate britannique brillant, Sir Ernest Satow voyage
dans une des frégates anglaises. Il deviendra par la suite et jusqu'en 1900
l'ambassadeur de Londres au Japon, et, par son journal (\emph{A Diplomat in
Japan}), donne un témoignage de première main passionnant sur cette période
trouble de transition.

Le Japon est divisé en deux groupes : l'un, \textcolor{red}{pro-shogun},
centré sur la capitale Edo (l'actuelle Tokyo) et l'autre, \textcolor{blue
}{anti-shogun}, donc \textcolor{blue}{pro-empereur}, unissant les clans de
l'ouest et du sud du Japon. Les deux partis prennent des mesures fortes pour
accroître leurs influences respectives. Le \textcolor{red}{bakufu} ainsi
envoie des marins étudier dans les écoles navales occidentales, et s'allie
avec la France (qui, l'histoire le montrera, aura parié sur le mauvais cheval)
pour construire notamment des arsenaux et moderniser son armée. Tiens, autre
anecdote\dots\ Vous souvenez-vous de cette purge du \emph{Dernier Samourai}
avec Tomtom Cruise ? La vérité n'est pas totalement loin de ce qui est montré
dans le flim, et le personnage joué par le beau gosse aux yeux bleus a
réellement existé~\incise~sauf qu'il était français. Il s'appelait
\textcolor{red}{Jules Brunet}, un officier artilleur de l'armée de Napoléon
III en charge de moderniser l'armée du shogun, dont nous parlerons un peu plus
bas. La \textcolor{blue}{coalition impériale} de son côté n'était pas en reste
: poussée par des visionnaires comme \textcolor{blue}{Sakamoto Ryoma} ou
\textcolor{blue}{Saigo Takamori}, des experts américains et britanniques
aident à former une nouvelle armée, issue de toutes les strates de la
population, et achètent des mitrailleuses, des obusiers et des navires de
guerre, notamment au marchant écossais Thomas Glover. C'est cet écossais qui a
aidé à fonder ce qui deviendra la compagnie Mitsubishi et la Brasserie
Japonaise~\incise~laquelle arborera en hommage à Glover une moustache sur ses
bouteilles de Kirin. C'est lui aussi qui, à la suite d'une affaire avec une
courtisane, a inspiré Puccini pour écrire \emph{Madame Butterfly}.

Un mot sur \textcolor{blue}{Sakamoto Ryoma}. Né dans le fief de Tosa, épéiste
confirmé, il s'engage vite dans le patriotisme anti-étrangers et choisit
d'assassiner \textcolor{red}{Katsu Kaishu} (dans quelle couleur l'écrire ?),
un officiel haut-gradé du shogunat, et fervent partisan de l'occidentalisation
et de la modernisation du Japon. Il entre dans sa maison au crépuscule armé
simplement d'un sabre court, et le surprend en train de lire un traité de
droit maritime néerlandais. \guil{Je ne suis pas armé, et j'aimerai finir ce
livre avant de mourir. Quelle différence cela fait-il pour toi, que je meure
maintenant ou dans quelques heures ? Je te jure, sur mes ancêtres, de ne pas
fuir quand le soleil se lèvera}, lui lance le conseiller. Si la légende dit
vrai, \textcolor{red}{Katsu Kaishu} réussit à persuader
\textcolor{blue}{Ryoma} de la futilité qu'il avait à combattre les pouvoirs
occidentaux au vu de l'état du Japon de l'époque, et de la nécessité pour le
pays d'avoir un projet à long terme. À l'aube, \textcolor{blue}{Ryoma} devient
l'assistant et le protégé de \textcolor{red}{Kai}\textcolor{blue}{shu} (qui
deviendra d'ailleurs, en 1873, le ministre de la Marine Impériale).
\textcolor{blue}{Ryoma} meurt hélas avant d'avoir vu l'impact de ses actions
et de ses convitions, à 31 ans, dans une auberge à Kyoto, assassiné par une
milice shogunale bien connue par les amateurs de flims de sabre et de mangas,
le \textcolor{red}{Shinsengumi}.

En effet, devant l'agitation dans sa capitale, créée notamment par d'anciens
samourais, le \textcolor{red}{shogunat} fonde de nombreuses milices
indépendantes, chargées de maintenir l'ordre. Ces milices, extrêmement
organisées et régies sur un code d'honneur strict, sont constituées de
samourais très compétents, l'entrée se faisant sur une évaluation du niveau de
\emph{kenjutsu} (de sabre). Dans les faits, ces épéistes d'élite se comportent
de la même manière qu'une mafia : assassinats, extorsions, chantages,
intimidations, actes terroristes\dots\ étaient fréquents. C'est cependant là
une autre preuve de la différence fondamentale qui existe entre les morales
japonaise et occidentale : la plus connue de ces milices, le
\textcolor{red}{Shinsengumi}, arborait en effet sur les vestes de chacun de
ses membres l'idéogramme \emph{makoto}, la \guil{sincérité} dont nous parlions
tantôt, écrit en traits d'or sur fond rouge. Une autre de ses milices, les
\emph{\textcolor{red}{hitokiri}} (mot-à-mot, \guil{coupeurs de gens}),
contenait quatre escrimeurs d'élite dont un \textcolor{red}{Tanaka Shinbei}
dont je raconte une anecdote captivante tout à la fin.

C'est \textcolor{blue}{Saigo Takamori}, un colosse taciturne et colérique, du
domaine de \textcolor{blue}{Satsuma}, qui prend la tête des troupes impériales
nouvellement formées. Menacé par une action militaire imminente, le shogun
abdique en 1867 et rend le pouvoir à \textcolor{blue}{l'Empereur Komei}. Dans
la foulée, l'Empereur Komei meurt (probablement assassiné d'ailleurs par des
extrémistes de \textcolor{blue}{Choshu}, craignant une alliance entre
l'Empereur et le shogun) et est remplacé par son jeune (15 ans) fils qui sera
connu sous le nom \textcolor{blue}{d'Empereur Meiji}. L'Empereur Meiji se
suffit de ce transfert d'autorité, mais \textcolor{blue}{Saigo} tempête et
exige que que le \textcolor{red}{shogun} soit privé de son armée, de ses terres
et de son titre, et devienne un simple daimyo. Le shogun refuse, et le Japon
plonge dans une guerre civile, la guerre du Boshin.

En janvier 1868, les forces du \textcolor{red}{shogunat} attaquent les forces
de \textcolor{blue}{Choshu et de Satsuma}, menées par \textcolor{blue}{Saigo},
à l'entrée de Kyoto. Même si l'armée shogunale compte 15\,000 hommes, certains
formés par des experts français, la majorité d'entre elle reste d'anciens
samurai. Les forces impériales, comptant moins de 5\,000 hommes, posséde
cependant des fusils, des mitrailleuses et une artillerie technologiquement
supérieurs, et une armée globalement mieux disciplinée~\incise~et moins
encline aux charges héroïques et honorables, quoique tactiquement peu
efficaces\dots\ C'est \textcolor{red}{Kastu Kaishu} (le \guil{Bismarck
japonais}) qui négocie avec \textcolor{blue}{Saigo} la capitulation d'Edo, en
mai. Désorganisée, démotivée et sans ressources, l'armée shogunale essuie les
défaites et fuit toujours plus au nord. Quelques centaines de soldats fondent
la \textcolor{red}{République d'Ezo }à Hokkaido (l'île tout au nord du Japon).
Parmi ces hommes, l'ancien commandant-en-chef du \textcolor{red}{Shinsengumi},
et cinq capitaines français, dont \textcolor{red}{Jules Brunet} dont nous
parlions plus haut. Anecdote amusante, au début du conflit, les Français
envoyèrent à Napoléon III une lettre. La France, comme les autres puissances
occidentales, s'étant déclarée comme neutre, ils devaient quitter l'armée
française pour y participer. Cette lettre de démission indique explicitement
qu'ils estimaient que les qualités d'honneur, de valeur et de dignité, si
importantes pour un militaires, étaient plus prononcées ici au Japon que dans
l'armée française. À un contre trois, les dernières forces moribondes rebelles
se font décimer, en dépit de la forteresse en étoile, construite sur des plans
de Vauban. Certains conseillers français restent au Japon, Brunet rentre et,
en dépit de la requète japonaise d'être jugé comme traître, le soutien
populaire dont il jouit force l'armée française de le cacher pendant quelques
années, le temps que l'affaire se tasse.

La victoire sur tout le territoire acquise, le nouveau gouvernement, avec
\textcolor{blue}{l'Empereur Meiji} à sa tête (quoiqu'à 15 ans, la tête était
plus symbolique qu'autre chose) s'efforce de moderniser le pays.
L'indépendance des domaines est petit à petit supprimée, et la classe des
samourai est abolie, par des édits successifs : interdiction de porter le
chignon, les sabres, suppression de leurs privilèges, autorisation des classes
populaires d'avoir un nom de famille\dots\ \textcolor{blue}{Saigo Takamori},
qui avait mené les troupes impériales pendant la guerre du Boshin, rejoint un
poste-clef dans le gouvernement, et demande la clémence pour les anciens
partisans au shogun. Le gouvernement impérial fait petit à petit glisser sa
politique initiale d'\guil{expulser les barbares !} en cherchant de plus en
plus à construire une nation forte, similaire aux puissances occidentales.
Mais la notion de \guil{nation forte} divise le gouvernement naissant :
certains considèrent qu'il faut moderniser le pays, avec des projets publics
de grande ampleur (télégraphe, chemin de fer\dots), d'autres, comme
\textcolor{blue}{Saigo}, insistent qu'il s'agit en premier lieu de posséder
une armée puissante, afin que le pays soit reconnu par les autres puissances
mondiales. Ce dernier propose d'ailleurs d'envahir la Corée, suite à son refus
de reconnaître \textcolor{blue}{l'Empereur Meiji} comme chef d'État. nt se
propose même d'aller en Corée en temps qu'ambassadeur, et de se comporter de
manière si outrancière que les Coréens n'aient d'autre choix que de le tuer,
provocant un \emph{casus belli} donnant au Japon une raison de lui déclarer la
guerre. Une telle entreprise aurait été cependant désastreuse, notamment sur
les finances balbutiantes du pays, et est refusée. \textcolor{green}{Saigo}
démissionne de toutes ses fonctions et rentre dans sa ville natale, Kagoshima.

Il fonde une école militaire privée, afin surtout de canaliser les énergies
des anciens samourais qui l'aidèrent à combattre le shogunat. Craignant une
rebellion de ces \textcolor{green}{forces nostalgiques}, le
\textcolor{blue}{gouvernement impérial} envoie un navire de guerre à Kagoshima
pour prendre toutes les armes et munitions des arsenaux. Ironiquement, cela
conduit les tensions déjà importantes (en cette même année de 1877, le
\textcolor{blue}{gouvernement} supprime les rentes des samourais) à éclater en
conflit armé. \textcolor{green}{Un millier d'étudiants} armés tout autant
d'armes à feu moderne que d'armes traditionnelles (sabres, lances, arcs)
attaquent les ports et arsenaux. Devant ce fait accompli,
\textcolor{green}{Saigo}, écartelé entre la manifestaion vivante de ses
morales qui disparaissent et son désir pacifique loyal au Japon, est poussé à
mener la rebellion. À 60 contre 1 environ, les mathématiques ne sont pas du
côté des insurgés. Quelques mois plus tard, \textcolor{green}{Saigo} lance une
charge héroïque et stupide, à la tête d'une \textcolor{blue}{armée bigarrée},
entre fusiliers sans munition et sabreurs anachroniques, contre une compagnie
de \textcolor{green}{mitrailleuses modernes}. Blessé à la cuisse, il se retire
du champ de bataille pour s'ouvrir le ventre comme l'exige son rang : ainsi
meurt le dernier samourai. 

Bien que le grand Saigo restera dans l'inconscient collectif comme étant un
grand héros, le gouvernement a maintenant toute liberté de révolutionner le
pays, et de le modeler comme une puissance occidentale. Il adopte une
constitution, inspirée de la constitution de l'Empire Germanique, et fait des
réformes de fond afin de moderniser le pays dans tous les domaines : sciences
modernes, langues, technologies, constructions, agriculture, éducation,
finance\dots\ et sports. Aux arts martiaux, associés à la tradition
rétrograde, sont préférés les sports occidentaux, comme le base-ball (encore
très en vogue au Japon aujourd'hui) et le rugby, par exemple. Mais alors, que
diable s'est-il donc passé pour que nous étudions l'aikido chaque semaine ?
Nous verrons cela dans notre prochain et dernier (ouf !) chapitre !

\chapter{Ère impériale}

\guil{He shrugged his shoulders. \guiluk{I have known many gods. He who denies
them is as blind as he who trusts them too deeply. I seek not beyond death. It
may be the blackness averred by the Nemedian skeptics, or Crom's realm of ice
and cloud, or the snowy plains and vaulted halls of the Nordheimer's Valhalla.
I know not, nor do I care. Let me live deep while I live; let me know the rich
juices of red meat and stinging wine on my palate, the hot embrace of white
arms, the mad exultation of battle when the blue blades flame and crimson, and
I am content. Let teachers and priests and philosophers brood over questions of
reality and illusion. I know this: if life is illusion, then I am no less an
illusion, and being thus, the illusion is real to me. I live, I burn with life,
I love, I slay, and am content.}}

Le XIX\up{e} siècle se finit : le Japon sort tout juste d'une période de
changements profonds, qui ressemble cependant plus à un dernier baroud
d'honneur de dinosaures conservateurs démodés qu'à une révolution sanglante. La
modernisation du pays, encore une féodalité médiévale en 1850, est fulgurante :
grands chantiers publics, signature d'une Constitution nationale, création d'un
parlement, refonte des institutions politiques, émergence d'une presse
importante, création d'une monnaie fiduciaire unique et d'une économie de
marché, montée en puissance de grands groupes industriels\dots\ Ces derniers
ont à leur tête d'anciens chefs de clans, et deviendront plus tard les grandes
firmes \emph{zaibatsu}, comme Mitsui ou Mitsubishi. En trente ans, l'Empire du
Japon possède une force que les Empires occidentaux ont acquis en plus d'un
siècle.

Ce formidable développement du Japon se mue peu à peu en complexe de
supériorité, puis franchement en impérialisme. Devant l'humiliation des nations
voisines, en particulier celle de la Chine~\incise~cf. les guerres de
l'opium~\incise~des intellectuels du Japon proposent le concept de \emph{zone
d'influence}, dans laquelle le Japon sortirait de ses frontières pour se garder
une zone \guil{tampon} dans le cas d'une attaque étrangère. Une série de
tensions entre la Corée et la Mandchourie culmine en 1894, quand le Japon en
profite pour intervenir. Sous prétexte de vouloir aider la Corée, l'Empire du
Soleil Levant envoie son armée pousser ses frontières contre celle de l'Empire
du Milieu. En un an, la flotte chinoise est détruite, et une série de défaite
militaire contraint la Chine à signer un traité de paix, concédant au Japon
plusieurs territoires (Corée, Taïwan, Port-Arthur en Mandchourie\dots). Ce sont
les tensions impérialistes autour de ce port qui conduisent le Japon à déclarer
la guerre à la Russie tsariste, dix ans plus tard. Là encore, le conflit ne
dure pas longtemps, et la Russie, pas aidée par la révolution rouge d'octobre
de la même année, signe un traité de paix au bout d'un an, cédant l'île
Sakhaline au Japon. En quelques anénes, l'Empire du Japon gagne ses galons de
grande puissance : la \guil{race jaune} a en effet battu, dans un combat
régulier, un gouvernement de la \guil{race blanche}, pourtant fondée à
gouverner le monde. Cette surprise viendra conforter la dangereuse évolution du
climat impérialiste au Japon, légitimant ses ambitions coloniales en se
qualifiant de \guil{pays des dieux}. Le Japon s'estime alors fondé à étendre
son influence colonialiste sur l'ensemble du continent asiatique. La spirale
mortelle est amorcée, et la suite n'est que trop bien connue : du massacre de
Nankin aux bombardements nucléaires, la course à l'abîme va se poursuivre avec
une ponctualité métronomique.

Mais nous n'avons toujours par parlé de l'aikido, pour l'instant\dots\
Revenons un peu en arrière : 1880, le Japon est en pleine modernisation. Les
connaissances occidentales sont considérées comme préférables aux antiques
voies. Aux arts martiaux sont préférés, notamment à haut niveau, les sports
\guil{civilisés} comme le rugby, le football ou le baseball. Parallèlement, un
fort sentiment patriotique nait au Japon. Le nouvel Empire a besoin de
symboles unificateurs forts, et va donc les chercher dans son passé. Ainsi, la
figure du héros loyal jusqu'à la mort est personnifié par Kusunoki Masashige,
le général du XIV\up{e} siècle dont nous avions parlé tantôt. Le
\emph{kamikaze}, le typhon providentiel qui a(urait) mis en déroute les
envahisseurs Mongols en 1280 est vu comme une preuve du caractère sacré et
inviolable du Japon. Nitobe Inazo, un universitaire formé par les Jésuites,
théorise un code moral dans \emph{Bushido, l'Âme du Japon}. Ce livre, à peu
d'être un outil de propagande impérialiste, décrit les qualités qu'auraient
possédées les guerriers d'antan, et expose comment les appliquer aujourd'hui.
La loyauté envers l'Empereur, issue d'une vision confucéenne du monde, et le
sacrifice de soi sont particulièrement mis en avant~\incise~ce qui permettra
aux militaires ensuite de justifier un paquet de morts\dots

C'est dans cet état d'esprit bicéphale, modernisation \emph{vs.} traditions,
toujours terriblement présent de nos jours, qu'un petit enseignant de jiujitsu
arrive sur le devant de la scène. Je crois que Jean-Louis a lu extensivement
sur le sujet, et je le laisserai donc me corriger et compléter. Kano Jigoro,
1m56, 40 kilos tout mouillé, est un haut-fonctionnaire du gouvernement, chargé
de superviser les sports et leur éducation dans l'Empire. Il a étudié
longuement avec différents maîtres de différentes écoles, et possède un style,
qu'il a nommé \emph{judo} (la discipline de la souplesse). En dépit de
l'aspect résolument japonais de son art, ses méthodes d'enseignement, issues
des États- Unis, et sa philosophie sous-jacente, lui amènent un soutien de
l'État. Plutôt que les sempiternels exercices de gymnastique effectués en
masse dans les cours de récré, le judo permet un développement physique
\emph{et} mental, une forme d'enseignement moral étant en effet au coeur du
judo. Kano accepte cependant difficilement la récupération de son art par les
militaires, qui voient là une manière d'avoir aisément des recrues loyales et
en bonne santé, et se place en opposant majeur au fascisme japonais. En tous
les cas, le point est fait : les arts martiaux ancestraux peuvent évoluer. Ce
qu'on appelait les \emph{koryu} (\guil{anciennes écoles}) deviennent des
\emph{gendai budo} (\guil{arts martiaux modernes}). Notons qu'un art martial a
conservé tous ses aspects traditionnels, et a (relativement) peu changé depuis
deux mille ans, le sumo.

Là-dessus, un jeune homme maigrichon et pas bien grand étudie dès 1900 de
multiples écoles de jujitsu. Cinq ans plus tard, il décide de s'engager dans
l'infanterie, mais sa petite taille le lui interdit. À 1m56, il ne lui manque
que quelques centimètres. Qu'à cela ne tienne, il passe des heures suspendu à
un arbre avec des poids au pied, et participe à la guerre russo-japonaise en
Mandchourie. 1912 est une révélation : parti fonder un village à Hokkaido, il
rencontre le grand maître de l'école Daito de jujutsu~\incise~l'ancienne école
du clan Takeda, les cavaliers féroces du Kai. Il rencontre ensuite un des
fondateurs d'une secte shinto, qui donnera à son art une dimension spirituelle
forte. Dans la brèche ouverte par Kano Jigoro, il construit petit à petit une
discipline qui adapte les techniques guerrières ancestrales en réussissant à
les retourner, pour que les volontés de chacun, fussent-elles de nuire,
s'unissent plutôt que s'opposent. Cet homme, c'est O Ueshiba Morihei Sensei
(\guil{le grand professeur Ueshiba Morihei}), le vieillard à la peau
parcheminée et aux regard doux, perçant et impassible, que nous saluons à
chaque fois que nous rentrons dans le dojo.

\chapter{Une anecdote pour finir}

Désolé, je sais que je n'ai pas tenu mon engagement d'être court et
synthétique, mais je ne résiste décidément pas à vous proposer une histoire et
énigme, à propos de \textcolor{red}{Tanaka Shinbei} :

Une nuit sans lune, le 20 mai 1863, \textcolor{blue}{Anenokoji Kintomo}, un
chef radical à la cour de \textcolor{blue}{l'Empereur}, et un champion des
forces contre le \textcolor{red}{Shogun}~\incise~alors l'actuel centre de
pouvoir~\incise~rentra chez lui, marchant du \textcolor{blue}{palais impérial
de Kyoto} à sa résidence. La route était sombre, et ses pas étaient simplement
éclairés par une lanterne de papier portée par un de ces cinq pages.
Soudainement, un attaquant vêtu de noir, alors caché dans des fourrés, sauta
devant lui, dégaina et lui coupa le visage de haut en bas, avant de
disparaître aussi vite. Quatre des cinq domestiques du noble s'enfuirent, et
le cinquième courut à la recherche de l'assassin alors que le sang coulait du
front de son maître. Un autre assaillant arriva par derrière, et planta son
sabre profondément dans le dos \textcolor{blue}{d'Anenokoji}. Un troisième
agresseur le coupa au niveau du torse. Incapable de crier, le jeune noble
s'écroula à terre. Aidé par son page qui revint, \textcolor{blue}{Anenokoji}
boita une centaine de mètres avant de s'effondrer totalement. Ce leader du
mouvement d'expulsions des barbares étrangers et de renverser le Shogunat
était mort, à vingt-cinq ans.

Les discussions pour savoir qui était le meurtrier allaient bon train. Les
rues de Kyoto n'étaient pas sûres depuis quelques années, et franchement
sanglantes depuis quelques mois. L'anarchie, qui dans l'Histoire a toujours
précédé aux changements de régime, grondait farouchement, et en particulier
ici, comme nous venons de le voir. \textcolor{blue}{Anenokoji} revenait
d'ailleurs d'une discussion animée avec d'autres officiels de la
\textcolor{blue}{cour impériale}, dont le consensus était d'émettre un décret
scellé par l'Empereur obligeant le clan de \textcolor{blue}{Choshu} à
renverser \textcolor{red}{le pouvoir en place}. \textcolor{blue}{Choshu}, ce
sont les mecs de la bataille de Shimonoseki, si vous vous souvenez bien.

Tout le monde pointait le doigt vers les autres : Satsuma (le fief du grand
Saigo), un autre clan, voisin de Choshu et son grand rival politique, était
moins extrême que les gros tarés de Choshu : il préférait une alliance entre
la cour (\textcolor{blue}{impériale}) et le pouvoir en place (le
\textcolor{red}{shogun}), sous la forme d'un mariage entre la soeur de
l'Empereur, l'Impératrice Kazu, et \textcolor{red}{l'actuel shogun, Iemochi} ;
le daimyo de Choshu, au contraire, était partisan d'une guerre ouverte contre
le shogun d'une part et les étrangers de l'autre (d'où la bataille de
Shimonoseki). Satsuma accusait donc Choshu d'avoir buté un type qui risquait
d'argumenter en faveur d'un gouvernement hybride, de cohabitation. Choshu,
pour sa part, suspectait Satsuma d'avoir zigouillé un mec qui jouissait à la
cour de l'Empereur d'un réel pouvoir, et qui risquait de pencher au contraire
vers une guerre totale. D'autant plus que les Choshu ignoraient les intentions
profondes de Satsuma : si la rumeur disait vrai, le daimyo de Satsuma, Shimazu
Hisamitsu, allait mener un contingent de ses troupes dans Kyoto, la capitale
impériale, afin de tordre le bras à l'Empereur pour le forcer à déclarer la
guerre contre le shogunat. Ce dont rêvait Choshu, mais ça les faisait chier
que ce soit quelqu'un d'autre qu'eux qui mène l'armée.  Et, finalement, tout
le monde soupçonnait le shogunat d'avoir éliminé un dangereux ennemi.  Bon, et
y'a plein d'autres trucs (une vingtaine de jours avant sa mort,
\textcolor{blue}{Anenokoji} avait passé du temps à bord d'un cuirassé du
shogun, en compagnie de \textcolor{red}{Katsu Kaishu}, le brillant futur
ministre de la marine, et le plus fervent défenseur de l'ouverture du pays).
Mais bon, l'un dans l'autre\dots\ L'important : c'était le bordel.

Là où l'affaire se corse, c'est quand on apprend qu'un wakizashi, le sabre
court, avait été retrouvé sur les lieux du crime. Sur le fourreau du sabre
était peint, comme c'est la coutume, le blason du clan dont appartenait le
samurai qui possédait le sabre\dots\ Une croix dans un rond~\incise~Satsuma !
Les attaques du domaine de Choshu se firent plus insistantes, tandis que
Satsuma accusait Choshu d'avoir volé le sabre d'un de leurs guerriers, pour le
poser là comme preuve. Ceci était d'ailleurs assez crédible, dans la mesure où
c'était Choshu qui était actuellement chargé de garder le Palais Impérial, et
donc, subséquemment, gérait des troupes provenant de plusieurs contrées.

Le possesseur du sabre a d'ailleurs été identifié, via les gravures dans le
manche : un certain \textcolor{red}{Tanaka Shinbei}, samurai de Satsuma, et
surtout, assassin notoire. Assassin, oui, membre des Quatre
\guil{\textcolor{red}{Hitokiri}}, les trancheurs de gens, et sa brutalité
était bien connue~\incise~et il avait démontré sa maîtrise du sabre en de
nombreuses occasions~\incise~principalement des contrats pour des
\textcolor{red}{partisans du shogun}. Ok, sauf que :
\textcolor{blue}{Anenokoji} a été attaqué par trois hommes~\incise~ce qui
n'est pas le mode opératoire de \textcolor{red}{Tanaka}~\incise~et surtout,
pas un seul n'a été capable de lui délivrer un seul coup fatal ! D'ailleurs,
un assassin du calibre de Tanaka n'aurait jamais laissé son sabre
derrière~\incise~un sabre encore dans son fourreau, pas taché de sang.
Ajoutons à ça que tous les amis et camarades de Tanaka ont tous juré l'avoir
vu dans un maison de geisha à une centaine de kilomètres de Kyoto cette nuit-
là~\incise~une promesse de samurai, quelque chose à ne pas prendre à la
légère.

Et pourtant, la preuve circonstancielle demeurait : le sabre de
\textcolor{red}{Tanaka Shinbei} avait été trouvé là, et il fut arrêté quelques
jours plus tard.

Nous avons l'avantage de l'Histoire avec nous : dans son cachot, attendant son
procès imminent, Tanaka a écrit une lettre. Elle enjoignait un de ces proches
de ne pas informer les autorités du vol de son waikikizashi, survenu dans un
bordel de Kyoto une semaine avant le meurtre. Il lui demandait par écrit de ne
pas soulever la preuve irréfutable, aujourd'hui retrouvée d'ailleurs, de ce
vol, puisqu'il en avait rapporté à son supérieur (afin de récupérer un autre
sabre, partie absolue de l'uniforme de la caste des samurai). Le procès-verbal
de son procès nous apprend qu'il s'est déroulé ainsi : Tanaka Shinbei s'est
présenté, en blanc et non menotté (c'est la coutume : on ne lie pas un
guerrier, ce serait mettre sa parole en doute, et donc, l'insulter gravement)
devant ses interrogateurs. La première question fut : \guil{est-ce votre sabre
?}, demanda un des enquêteurs, le sabre à la main. \guil{Je ne sais pas, je ne
peux pas le voir convenablement d'ici. Je dois le voir de plus près.} répondit
Tanaka. Alors que l'on lui tendit son sabre, Tanaka mit lentement la lame à nu
pour l'examiner. \guil{Oui, c'est bien mon sabre}, déclara-t-il.
\guil{Cependant}, ajouta-t-il avec force, \guil{je n'ai pas tué Anenokoji.} --
\guil{Si vous ne l'avez pas tué, qui\dots} Avant que l'interrogateur ne puisse
finir sa question, et à la surprise des autorités restées interdites, Tanaka,
d'un geste décidé, retourna son sabre vers lui, plongea la lame en son flanc
gauche, et, sa main gauche rejoignant sa main droite, poussa le tranchant de
l'autre côté de son ventre. Puis, dans la même rafale de volonté fanatique,
ramena la pointe de sa lame sur son cou, et s'ouvrit la carotide, tombant vers
l'avant sans le moindre cri.

\emph{On a side note}, les enquêteurs conclurent à la culpabilité de Tanaka Shinbei.
Satsuma tomba en disgrâce (temporairement), et ce fut Choshu et ses radicaux
qui régnèrent (un temps) en maître sur l'ancienne capitale de l'Empereur,
devenue lieu de toutes les agitations les plus frénétiques. Alors,
question\dots\ Pourquoi, ce geste absurde et déraisonné de Shinbei, alors qu'il
pouvait prouver son innocence ? Quel gâchis ! Et si ça peut vous aider, sachez
que Mishima Yukio, l'écrivain génial, avait fait des pieds et des mains pour
pouvoir jouer ce rôle-là dans le \emph{Hitokiri} de Hideo Gosha.


\end{document}
